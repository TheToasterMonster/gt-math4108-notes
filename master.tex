\documentclass[12pt, letterpaper, oneside]{book}
\usepackage[margin={0.6in, 0.75in}]{geometry}
\usepackage{microtype}
% \usepackage{kpfonts}
\usepackage{amsmath, amssymb, amsthm}
\usepackage{hyperref}
\usepackage{parskip}
\usepackage[many]{tcolorbox}
\usepackage{footnote}
\usepackage{cancel}
\usepackage{titlesec}
\usepackage{pgffor}
\usepackage[shortlabels]{enumitem}
\usepackage{tikz-cd}

\renewcommand{\chaptername}{Lecture}
\newtheorem{axiom}{Axiom}[chapter]
\newtheorem{theorem}{Theorem}[chapter]
\newtheorem{prop}{Proposition}[chapter]
\newtheorem{corollary}{Corollary}[theorem]
\newtheorem{lemma}{Lemma}[chapter]
\theoremstyle{definition}
\newtheorem{definition}{Definition}[chapter]
\newtheorem{exercise}{Exercise}[chapter]
\newtheorem{example}{Example}[definition]
\newtheorem*{remark}{Remark}

\tcbset{sharp corners, breakable, enhanced, parbox=false}
\newtcolorbox{mybox}[3][]
{
  colframe = #2!150,
  colback  = #2!5,
  coltitle = #2!0!white,  
  title    = {#3},
  #1,
}

\titleformat{\chapter}[display]
    {\normalfont\huge\bfseries}{\chaptertitlename\ \thechapter}{20pt}{\Huge}
\titlespacing*{\chapter}{0pt}{0pt}{40pt}

\newcommand{\R}{\mathbb{R}}
\newcommand{\N}{\mathbb{N}}
\newcommand{\Z}{\mathbb{Z}}
\newcommand{\C}{\mathbb{C}}
\newcommand{\Q}{\mathbb{Q}}
\newcommand{\F}{\mathbb{F}}
\newcommand{\GF}{\mathrm{GF}}
\newcommand{\id}{\mathrm{id}}

\DeclareMathOperator{\Char}{char}
\DeclareMathOperator{\ord}{ord}
\DeclareMathOperator{\lcm}{lcm}
\DeclareMathOperator{\Aut}{Aut}
\DeclareMathOperator{\Gal}{Gal}

\title{MATH 4108: Abstract Algebra II}
\author{Frank Qiang\\Instructor: Jennifer Hom}
\date{Georgia Institute of Technology\\Spring 2024}

\begin{document}
  \maketitle

  \begingroup
  \let\cleardoublepage\clearpage
  \tableofcontents
  \endgroup

  % \foreach \i in {00, 01, 02, 03, 04, ..., 50} {%
  %   \edef\FileName{lectures/lecture\i.tex}%     The % here are necessary to eliminate any
  %   \IfFileExists{\FileName}{%  spurious spaces that may get inserted
  %      \input{\FileName}%       at these points
  %   }
  % }
  \chapter{Jan.~8 --- Rings and Fields}

\section{Lots of Definitions}
Recall the definitions of a ring and a field:
\begin{definition}[Ring]
  A \emph{ring} $R = (R, +, \cdot)$ is a non-empty set
  $R$ together with two binary operations
  $+$ and $\cdot$, called addition and multiplication
  respectively, which satisfy:
  \begin{enumerate}
    \item[(R1)] \textit{Associative law for addition}:
      $(a + b) + c = a + (b + c)$ for all $a, b, c \in R$.
    \item[(R2)] \textit{Commutative law for addition}:
      $a + b = b + a$ for all $a, b \in R$.
    \item[(R3)] \textit{Existence of zero}: There exists $0 \in R$ such that
      $a + 0 = a$ for all $a \in R$.
    \item[(R4)] \textit{Existence of additive inverses}:
      For all $a \in R$, there exists $-a \in R$ such that
      $a + (-a) = 0$.\footnote{Note that we'll usually write $a - b$ in place of $a + (-b)$.}
    \item[(R5)] \textit{Associative law for multiplication}:
      $(ab)c = a(bc)$ for all $a, b, c \in R$.
    \item[(R6)] \textit{Distributive laws}:
      $a(b + c) = ab + ac$ and $(a + b)c = ac + bc$ for all $a, b, c \in R$.
  \end{enumerate}
\end{definition}

\begin{definition}[Commutative ring]
In this class, we will mostly be interested in
\emph{commutative rings}, which satisfy the following
additional property for multiplication:
\begin{enumerate}
  \item[(R7)] \textit{Commutative law for multiplication}:
    $ab = ba$ for all $a, b \in R$.
\end{enumerate}
\end{definition}

\begin{definition}[Ring with unity]
A ring \emph{with unity} satisfies the additional
property that
\begin{enumerate}
  \item[(R8)] \emph{Existence of unity}: There exists
    $1 \ne 0 \in R$ such that and
    $a 1 = 1 a = a$ for $a \in R$.
\end{enumerate}
\end{definition}
Note that a ring need not be commutative to have a unity.

\begin{definition}[Domain]
A commutative ring with unity is called a
\emph{(integral) domain} if it has the following
cancellation property:
\begin{enumerate}
  \item[(R9)] \emph{Cancellation}: For all
    $a, b \in R$ and $c \ne 0$, $ca = cb$ implies
    $a = b$.
  \item[(R9')] \textit{No zero divisors}: For all
    $a, b \in R$, $ab = 0$ implies $a = 0$ or $b = 0$.
\end{enumerate}
\end{definition}

The conditions (R9) and (R9') are equivalent.

\begin{definition}[Field]
A commutative ring with unity is called a \emph{field}
if it has the following additional property for
multiplicative inverses:
\begin{enumerate}
  \item[(R10)] \emph{Existence of multiplicative inverses}:
    For all $a \ne 0 \in R$, there exists
    $a^{-1} \in R$ such that $aa^{-1} = 1$.
\end{enumerate}
\end{definition}

\begin{example}
  Some examples of rings are $\Z / 2\Z$, which
  also happens to be a field. The ring $\Z$ is a domain.
  The set $M_{2 \times 2}(\R)$ is a non-commutative ring
  with unity, and has zero divisors. The ring
  $\Q$ is a field.\footnote{In fact, $\Q$ is somehow the smallest field containing $\Z$.} The real polynomials in a single
  variable $\R[x]$ form a ring, which is a domain but
  not a field. The complex numbers $\C$ and the
  real numbers $\R$ both form a field. The
  even integers $2\Z$ form a commutative ring without
  unity. In general, $\Z / n\Z$ is a commutative ring
  with unity, and is a field if and only if $n$ is prime
  (and has zero divisors otherwise, if $n$ is composite).
\end{example}

\begin{remark}
  If $(R, +, \cdot)$ is a ring, then $(R, +)$ is an
  abelian group. If $(K, +, \cdot)$ is a field,
  then $(K^*, \cdot)$ is an abelian group, where
  $K^* = K \setminus \{0\}$.
\end{remark}

\begin{definition}[Group of units]
  Let $R$ be a commutative ring with unity. The
  \emph{group of units} of $R$ is
  \[U = \{u \in R \mid \text{there exists $v \in R$ such that $uv = 1$}\}.\]
\end{definition}

\begin{exercise}
  Show that $U$ is in fact a group under multiplication.
\end{exercise}

\begin{definition}[Associate]
  If $a, b \in R$ such that $a = ub$ for some
  $u \in U$, then $a$ and $b$ are called
  \emph{associates}, denoted by $a \sim b$.
\end{definition}

\begin{exercise}
  Show that $\sim$ is in fact an equivalence relation.
\end{exercise}

\begin{example}
  The group of units of $\Z$ is $\{1, -1\}$. The group
  of units of a field $K$ is $K^* = K \setminus \{0\}$.
\end{example}

\begin{exercise}
  Let $R = \{a + b \sqrt{2} \mid a, b \in \Z\}$. Check
  the following:
  \begin{enumerate}
    \item $R$ is a commutative ring with unity.
    \item The group of units of $R$ is
      $\{a + b \sqrt{2} \mid a, b \in \Z, |a^2 - 2b^2| = 1\}$.
  \end{enumerate}
\end{exercise}

\begin{definition}[Divisor]
  Let $D$ be an integral domain, $a \in D \setminus \{0\}$,
  $b \in D$. Then $a$ divides $b$, or $a$ is a
  \emph{divisor} or \emph{factor} of $b$, denoted by
  $a | b$, if there exists $z \in D$ such that $az = b$.
  We write $a {\nmid} b$ if $a$ does not divide $b$.
  We say that $a$ is a \emph{proper divisor} or
  that $a$ \emph{properly divides} $b$ if $z$ is not
  a unit.
\end{definition}

\begin{remark}
  Equivalent, $a$ is a proper divisor of $b$ if
  and only if $a | b$ and $b {\nmid} a$.
\end{remark}

\begin{definition}[Subring]
  A \emph{subring} $U$ of a ring $R$ is a non-empty
  subset of $R$ with the property that for all
  $a, b \in R$, $a, b \in U$ implies $a + b \in U$
  and $ab \in U$,
  and $a \in U$ implies $-a \in U$.
\end{definition}

\begin{remark}
  Equivalently, $U$ is a subring of $R$ if and only
  if $a, b \in U$ implies $a - b \in U$ and $ab \in U$.
\end{remark}

\begin{remark}
  We automatically have $0 \in U$ since we can pick
  any $a \in U$, and then $0 = a - a \in U$.
\end{remark}

\begin{definition}[Subfield]
  A \emph{subfield} of a field $K$ is a subset $E$
  containing at least two elements such that
  $a, b \in E$ implies $a - b \in E$ and $a \in E, b \in E \setminus \{0\}$ implies $ab^{-1} \in E$. If $E$ is
  a subfield and $E \ne K$, then we say $E$ is a
  \emph{proper} subfield.
\end{definition}

\begin{remark}
  As before, we can replace the last condition with
  the equivalent statement that
  $a, b \in E$ implies $ab \in E$ and
  $a \in E \setminus \{0\}$ implies $a^{-1} \in E$.
\end{remark}

\begin{definition}[Ideal]
  An \emph{ideal} of $R$ is a non-empty subset $I$ of
  $R$ with the properties that $a, b \in I$ implies
  $a - b \in I$ and $a \in I, r \in R$ implies $ra \in I$.
\end{definition}

\begin{remark}
  All ideals are subrings, but the converse is not
  true in general.
\end{remark}

\begin{example}
  The integers $\Z$ form a subring of $\R$ but not an
  ideal.
\end{example}

\begin{remark}
  We trivially have that $\{0\}$ and $R$ are both
  ideals of $R$. An ideal $I$ is called \emph{proper}
  if $\{0\} \subsetneq I \subsetneq R$.
\end{remark}

\begin{tcolorbox}
\begin{theorem}
  Let $A = \{a_1, \dots, a_n\}$ be a finite subset
  of a commutative ring $R$. Then the set
  \[Ra_1 + \dots + Ra_n = \{x_1 a_1 + \dots + x_n a_n \mid x_i \in R\}\]
  is the smallest ideal of $R$ containing $A$.
\end{theorem}
\end{tcolorbox}

\begin{proof}
  See Howie. Check this is indeed an ideal
  and is contained in any other ideal containing $A$.
\end{proof}

\begin{definition}[Ideals generated by elements of a ring]
  The set $Ra_1 + \dots + Ra_n$ is the
  \emph{ideal generated by} $a_1, \dots, a_n$,
  denoted by $\langle a_1, \dots, a_n \rangle$.
  If the ideal is generated by a single element
  $a \in R$, then we say that $Ra = \langle a \rangle$
  is a \emph{principal ideal}.
\end{definition}

\begin{example}
  In $\Z$, the ideal $\langle 2 \rangle = 2\Z$ are
  the even numbers. We have
  $\langle 2, 3 \rangle = \Z$, but
  $\langle 6, 8 \rangle = \langle 2 \rangle$.
\end{example}

\begin{tcolorbox}
\begin{theorem}
  Let $D$ be an integral domain with group of units
  $U$ and let $a, b \in D \setminus \{0\}$. Then
  \begin{enumerate}
    \item $\langle a \rangle \subseteq \langle b \rangle$
      if and only if $b | a$,
    \item $\langle a \rangle = \langle b \rangle$
      if and only if $a \sim b$,
    \item $\langle a \rangle = D$ if and only if
      $a \in U$.
  \end{enumerate}
\end{theorem}
\end{tcolorbox}

\begin{proof}
  See Howie.
\end{proof}

\begin{definition}[Homomorphism of rings]
  A \emph{homomorphism} from a ring $R$ to a ring $S$
  is a mapping $\varphi : R \to S$ such that
  $\varphi(a +_R b) = \varphi(a) +_S \varphi(b)$ and
  $\varphi(ab) = \varphi(a)\varphi(b)$ for all
  $a, b \in R$.
\end{definition}

\begin{example}
  The zero mapping $\varphi(a) = 0$ is always a
  homomorphism. The inclusion map
  $\iota : 2\Z \to \Z$ or $\iota : \Z \to \Q$ is
  a homomorphism.
\end{example}

\begin{tcolorbox}
\begin{theorem}
  Let $R, S$ be rings and $\varphi : R \to S$ a
  homomorphism. Then
  \begin{enumerate}
    \item $\varphi(0_R) = 0_S$,
    \item $\varphi(-r) = -\varphi(r)$ for all $r \in R$,
    \item the image $\varphi(R)$ is a subring of $S$.
  \end{enumerate}
\end{theorem}
\end{tcolorbox}

\begin{proof}
  See Howie.
\end{proof}

\begin{definition}[Monomorphism]
  Let $\varphi : R \to S$ be a homomorphism. If
  $\varphi$ is injective, we say that $\varphi$ is
  a \emph{monomorphism} or an \emph{embedding}.
\end{definition}

\begin{example}
  The inclusion map $\varphi : \Z \to \R$ given by
  $\varphi(n) = n$ is an embedding.
\end{example}

  \chapter{Jan.~10 --- Field of Fractions, Polynomials}

\section{Isomorphisms}
\begin{definition}[Isomorphism]
If a homomorphism $\varphi : R \to S$ is both one-to-one
and onto,
then $\varphi$ is an \emph{isomorphism} and we say $R$
and $S$ are \emph{isomorphic}, denoted
$R \cong S$.
\end{definition}

\begin{definition}[Automorphism]
  An isomorphism $\varphi : R \to R$ is called an
  \emph{automorphism}.
\end{definition}

\begin{example}
  For any ring $R$, the identity map $\varphi : R \to R$
  with $\varphi = \text{id}$ is an automorphism.
\end{example}

\begin{exercise}
  The complex conjugation $\varphi : \C \to \C$
  with $\varphi(z) = \overline{z}$ is an automorphism.
\end{exercise}

\begin{definition}[Kernel]
  Let $\varphi : R \to S$ be a homomorphism. The
  \emph{kernel} of $\varphi$ is
  \[
    \ker \varphi = \phi^{-1}(0_S) \{ a \in R : \varphi(a) = 0_S \}.
  \]
\end{definition}

\begin{exercise}
  For any homomorphism $\varphi$, $\ker \varphi$ is
  an ideal.
\end{exercise}

\begin{definition}[Residue class]
  Let $I$ be an ideal of a ring $R$ and $a \in R$.
  The set
  \[a + I = \{a + x \mid x \in I\}\]
  is the \emph{residue class} of $a$ modulo $I$.
\end{definition}

\begin{exercise}
  The set $R / I$ of residue classes modulo $I$ forms
  a ring with respect to the operations
  \[(a + I) + (b + I) = (a + b) + I \quad \text{and} \quad (a + I)(b + I) = ab + I.\]
\end{exercise}

\begin{exercise}
  The map $\theta_{I} : R \to R / I$ with
  $\theta_I(a) = a + I$ is a surjective homomorphism
  onto $R / I$ with kernel $I$. This map $\theta_I$ is
  called the \emph{natural homomorphism} from
  $R$ to $R / I$.
\end{exercise}

\begin{example}
  Consider $\Z$ and $I = \langle n \rangle = n \Z$.
  Then $\theta_I : \Z \to \Z / n \Z$ with
  $\theta_I(a) = a + \langle n \rangle$ is the
  natural homomorphism. There are $n$ residue classes,
  which are
  \[\langle n \rangle, \quad 1 + \langle n \rangle, \quad \dots, \quad (n - 1) + \langle n \rangle.\]
\end{example}

\begin{theorem}
  Let $n \in \Z_{> 0}$. Then $\Z / n \Z$ is a field
  if and only if $n$ is prime.
\end{theorem}

\begin{proof}
  See Howie.
\end{proof}

\begin{remark}
  If $n = 0$, then $\Z / 0 \Z \cong \Z$.
\end{remark}

\begin{theorem}
  Let $\varphi : R \to S$ be a surjective homomorphism
  with kernel $K$. Then there is an isomorphism
  $\alpha : R / K \to S$ such that the following diagram
  commutes (i.e. $\varphi = \alpha \circ \theta_K$):
  \[
    \begin{tikzcd}
      R \arrow{r}{\varphi} \arrow[swap]{d}{\theta_K}
      & S \\
      R / K \arrow{ur}{\alpha}
    \end{tikzcd}
  \]
\end{theorem}

\begin{proof}
  See Howie. But the general idea is to define
  $\alpha : R / K \to S$ by $\alpha(a + K) = \varphi(a)$.
  Then need to check that $\alpha$ is well-defined
  and an isomorphism.
\end{proof}

\section{Field of Fractions}
The motivating question is: How do we get from
$\Z$ to $\Q$? Recall that
\[\Q = \{a / b \mid a, b \in \Z, b \ne 0\},\]
where $a / c = b / d$ if $ad = bc$. We add and multiply
fractions by
\[
  \frac{a}{b} + \frac{c}{d} = \frac{ad + bc}{bd} \quad
  \text{and} \quad
  \frac{a}{b} \cdot \frac{c}{d} = \frac{ac}{bd}.
\]
How do we do this more generally (construct a field out
of an arbitrary integral domain)?

\begin{definition}[Field of fractions of a domain]
  Let $D$ be an integral domain and
  \[P = D \times (D \setminus \{0\})
    = \{(a, b) \mid a, b \in D, b \ne 0.\}
  \]
  Define an equivalence relation $\equiv$ on $P$ by
  $(a, b) \equiv (a', b')$ if $ab' = a'b$. Then
  the \emph{field of fractions} of $D$ is
  \[
    Q(D) = P / {\equiv}.
  \]
  We denote the equivalence class $[a, b]$ by
  $a / b$, i.e. $a / b = c / d$ if $ad = bc$. We
  define addition and multiplication on $Q(D)$ by
\[
  \frac{a}{b} + \frac{c}{d} = \frac{ad + bc}{bd} \quad
  \text{and} \quad
  \frac{a}{b} \cdot \frac{c}{d} = \frac{ac}{bd}.
  \]
\end{definition}

\begin{exercise}
  Do the following:
  \begin{enumerate}
    \item Check that $\equiv$ is an equivalence relation.
    \item Check that these operations are well-defined.
    \item Check that $Q(D)$ is a commutative ring with unity.
      \begin{itemize}
        \item The zero element is $0 / b$ for $b \ne 0$.
        \item The unity element is $a / a$ for $a \ne 0$.
        \item The negative of $a / b$ is $(-a) / b$
          or equivalently $a / (-b)$.
        \item The multiplicative inverse of $a / b$ is
          $b / a$ for $a, b \ne 0$.
      \end{itemize}
    \item Complete the previous exercise and check that
      $Q(D)$ is a field.
  \end{enumerate}
\end{exercise}

\begin{exercise}
  The map $\phi : D \to Q(D)$ defined by
  $\phi(a) = a / 1$ is a monomorphism. In particular,
  the field of fractions
  $Q(D)$ contains $D$ as a subring and $Q(D)$ is
  the smallest field containing $D$, in the sense that
  if $K$ is a field with the property that there exists
  a monomorphism $\theta : D \to K$, then there exists
  a monomorphism $\psi : Q(D) \to K$ such that the
  following diagram commutes:
  \[
    \begin{tikzcd}
      D \arrow{r}{\theta} \arrow[swap]{d}{\varphi}
      & K \\
      Q(D) \arrow{ur}{\psi}
    \end{tikzcd}
  \]
\end{exercise}

\section{The Characteristic of a Field}
Note that for $a \in R$, we might write $a + a$ as
$2a$ and $a + a + \dots + a$ ($n$ times) as $na$.
Furthermore, $0a = 0_R$ and $(-n)a = n(-a)$ for
$n \in \Z_{> 0}$. Thus $na$ has meaning for all
$n \in \Z$.\footnote{This is saying that any abelian group is naturally a \emph{module} over the integers $\Z$.}

\begin{exercise}
  For $a, b \in R$ and $m, n \in \Z$, we have
  $(ma)(nb) = (mn)(ab)$.
\end{exercise}

\begin{definition}[Characteristic of a ring]
For an arbitrary ring $R$, there are two possibilities:
\begin{enumerate}
  \item $m 1_R$ for $m \in \Z$ are all distinct. In
    this case, we say that $R$ has \emph{characteristic}
    $0$.
  \item There exists $m, n \in \N$ such that
    $m 1_R = (m + n) 1_R$. In this case, we say that
    $R$ has \emph{characteristic} $n$, where $n$ is
    the least positive $n$ for which this property
    holds.
\end{enumerate}
We denote the characteristic of $R$ by $\Char R$.
If $\Char R = n$, then $na = 0_R$ for all $a \in R$
since $na = (n 1_R) a = 0a = 0$.
\end{definition}

\begin{example}
  We have $\Char \Z / n\Z = n$.
\end{example}

\begin{theorem}
  The characteristic of a field is either $0$ or a prime.
\end{theorem}

\begin{proof}
  Let $K$ be a field and suppose $\Char K = n \ne 0$
  and $n$ is not prime. Then we can write $n = rs$ where
  $1 < r, s < n$. The minimal property of $n$
  implies that $r 1_K \ne 0$ and $s 1_K \ne 0$. But then
  \[
    r 1_K \cdot s 1_K = rs 1_K = n 1_K = 0,
  \]
  which is impossible since $K$ is a field and thus
  has no zero divisors.
\end{proof}

\begin{remark}
  Note the following:
  \begin{enumerate}
    \item If $K$ is a field with $\Char K = 0$, then
      $K$ has a subring isomorphic to $\Z$, i.e.
      elements of the form $n 1_K$ for $n \in \Z$,
      and $K$ has a subfield isomorphic to $\Q$,
      i.e.
      \[P(K) = \{m 1_K / n 1_K \mid m, n \in \Z, n \ne 0\}.\]
      This is the \emph{prime subfield} of $K$,
      and any subfield of $K$ must contain $P(K)$.
    \item If $K$ is a field with $\Char K = p$, then
      the prime subfield of $K$ is
      \[P(K) = \{1_K, 2 \cdot 1_K, \dots, (p - 1) \cdot 1_K\},\]
      which is isomorphic to $\Z / p \Z$.
  \end{enumerate}
\end{remark}

\begin{remark}
  In other words, every field of characteristic $0$
  is an \emph{extension} of $\Q$ (contains $\Q$ as a subfield),
  and every field of characteristic $p$ is an
  \emph{extension} of $\Z / p \Z$ (contains $\Z / p\Z$
  as a subfield).
\end{remark}

\begin{remark}
  If $\Char K = 0$, then writing $a / n 1_K$ as
  $a / n$ is fine. But if $\Char K = p$, then
  $a / n$ does not make sense when $p | n$
  (since $p \cdot 1_K = 0$).
\end{remark}

\begin{theorem}
  If $K$ is a field with $\Char K = p$, then
  for all $x, y \in K$, $(x + y)^p = x^p + y^p$.
\end{theorem}

\begin{proof}
  See Howie. Uses the binomial theorem.
\end{proof}

\section{Polynomials}
Let $R$ be a ring, then we have the polynomial ring
over $R$
\[
  R[X] = \{a_0 + a_1 X + \dots + a_n X^n \mid a_i \in R, n \in \N\}.
\]
If $f \in R[X]$, then it has \emph{degree} $n$ if the
last nonzero element in the sequence
$\{a_0, a_1, \dots\}$ is $a_n$, denoted
$\partial f = n$. By convention, the zero polynomial has
degree $-\infty$. The coefficient $a_n$ is called the
\emph{leading coefficient}, and if $a_n = 1$, then
$f$ is \emph{monic}. Addition and multiplication work
as expected:
\[
  (a_0 + a_1 X + \dots + a_m X^m) + (b_0 + b_1 X + \dots + b_n X^n)
  = (a_0 + b_0) + (a_1 + b_1) X + \dots
\]
and
\[
  (a_0 + a_1 X + \dots + a_m X^m)(b_0 + b_1 X + \dots + b_n X^n)
  = c_0 + c_1 X + \dots
\]
where
\[
  c_k = \sum_{i + j = k}^k a_i b_{j}.
\]
The ground ring $R$ sits inside of the polynomial ring
$R[X]$. Take the monomorphism $\theta : R \to R[X]$
by $\theta(a) = a$, i.e. an element $a$ maps to the
constant polynomial $a$.

\begin{theorem}
  Let $D$ be an integral domain. Then
  \begin{enumerate}
    \item $D[X]$ is an integral domain.
    \item If $p, q \in D[X]$, then
      $\partial (p + q) \le \max(\partial p, \partial q)$.
    \item If $p, q \in D[X]$, then
      $\partial (p q) = \partial p + \partial q$.
    \item The group of units of $D[X]$ coincides
      with the group of units of $D$.
  \end{enumerate}
\end{theorem}

\begin{proof}
  Statements (2) and (3) are left as exercises.

  (1) We need to show that $D[X]$ has no zero divisors.
  For this, suppose that $p, q$ are nonzero polynomials
  with leading coefficients $a_m$ and $b_n$ respectively.
  Then the leading coefficient of $pq$ is $a_m b_n$,
  which is nonzero since $D$ is an integral domain and
  thus has no zero divisors. So $pq$ is nonzero.

  (4) Let $p, q \in D[X]$ and suppose $pq = 1$.
  Since $\partial (pq) = \partial (1) = 0$, we must
  have $\partial p = \partial q = 0$. Thus $p, q \in D$
  and $pq = 1$ if and only if $p$ and $q$ are in
  the group of units of $D$.
\end{proof}

Since $D[X]$ is a domain, we can consider polynomials
in the variable $Y$ with coefficients in $D[X]$:
\[D[X, Y] = (D[X])[Y].\]
We can repeat this to get polynomials in $n$
variables: $D[X_1, X_2, \dots, X_n]$, which is
an integral domain.

  \chapter{Jan.~17 --- Irreducible Polynomials}

\section{Principal Ideal Domains and Irreducibile Polynomials}
\begin{definition}
  The field of fractions of $D[X]$ consists of
  \emph{rational forms}
  \[
    \frac{a_0 + a_1 X + \dots + a_m X^m}{b_0 + b_1 X + \dots + b_n X^n}
  \]
  where $b_0 + b_1 X + \dots + b_n X^n \ne 0$,
  denoted by $D(X)$.
\end{definition}

\begin{definition}
  A domain $D$ is a \emph{principal ideal domain} (PID)
  if all of its ideals are principal.\footnote{Recall that a principal ideal is one generated by a single element.}
\end{definition}

\begin{example}
  The integers $\Z$ is a PID, since every ideal is of
  the form $\langle n \rangle$.
\end{example}

\begin{definition}
  A non-zero, non-unit element $p$ in a domain $D$
  is \emph{irreducible} if it has no proper factors.
\end{definition}

\begin{definition}
  A domain $D$ is a \emph{unique factorization domain} (UFD)
  if every non-unit $a \ne 0$ in $D$ has an
  essentially unique\footnote{As in, unique up to use of associates or adding in units.} factorization into irreducible
  elements.
\end{definition}

\begin{example}
  Again $\Z$ is a UFD, e.g. $12 = 2 \cdot 2 \cdot 3 = (-2) \cdot 2 \cdot (-3)$.
\end{example}

\begin{theorem}
  Every PID is a UFD.
\end{theorem}

\begin{proof}
  See Howie.
\end{proof}

\begin{theorem}
  If $K$ is a field, then $K[X]$ is a PID.
\end{theorem}

\begin{proof}
  See Howie.
\end{proof}

\begin{theorem}
  Let $p$ be an element in a PID $D$. Then the
  following are equivalent:
  \begin{enumerate}
    \item $p$ is irreducible.
    \item $\langle p \rangle$ is maximal.
    \item $D / \langle p \rangle$ is a field.
  \end{enumerate}
  In particular if $f \in K[X]$, then
  $K[X] / \langle f \rangle$ is a field if and only if $f$ is irreducible.
\end{theorem}

\begin{proof}
  See Howie.
\end{proof}

\begin{definition}
  Let $D$ be a domain and $\alpha \in D$.
  Let $\sigma_{\alpha} : D[X] \to D$ defined by
  \[
    \sigma_{\alpha} (a_0 + a_1 X + \dots + a_n X^n) = a_0 + a_1 \alpha + \dots + a_n \alpha^n.
  \]
  Note that we often write $\sigma_{\alpha} (f)$ as
  $f(\alpha)$. If $f(\alpha) = 0$, we say
  $\alpha$ is a \emph{root} of $f$, or a \emph{zero}.
\end{definition}

\begin{exercise}
  Check that $\sigma_{\alpha}$ is a homomorphism.
\end{exercise}

\begin{theorem}
  Let $K$ be a field, $\beta \in K$ and $f$ a non-zero
  polynomial in $K[X]$. Then $\beta$ is a root of $f$
  if and only if $X - \beta | f$.
\end{theorem}

\begin{proof}
  See Howie.
\end{proof}

\begin{example}
  We have $X^2 + 1$ in $\R[X]$ is irreducible,
  so $\R[X] / \langle X^2 + 1 \rangle$ is a field.
  In fact this field is isomorphic to the complex
  numbers $\C$.
\end{example}

\begin{exercise}
  Do the following:
  \begin{enumerate}
    \item Show that $\varphi : \R[X] \to \C$ given by
      \[
        \varphi (a_0 + a_1 X + \dots + a_n X^n) = a_0 + a_1 i + \dots + a_n i^n
      \]
      is a surjective homomorphism.\footnote{Note that there's some technicality about this $\varphi$ not being a $\sigma_{\alpha}$ since we defined $\sigma_{\alpha}$ for $\alpha$ in the base domain, and $i$ is kind of somewhere else.}
    \item Show that $\ker \varphi = \langle X^2 + 1 \rangle$.
  \end{enumerate}
  So by the first isomorphism theorem we can conclude
  that $\R[X] / \langle X^2 + 1 \rangle = \R / {\ker \varphi} \cong \varphi(\R[X]) = \C$.
\end{exercise}

\begin{theorem}
  Let $K$ be a field and $g \in K[X]$ an irreducible
  polynomial. Then $K[X] / \langle g \rangle$ is a field
  containing $K$ up to isomorphism.
\end{theorem}

\begin{proof}
  Since $g$ is irreducible, $K[X] / \langle g \rangle$
  is a field. Now define $\varphi : K \to K[X] / \langle g \rangle$ by
  \[
    \varphi (a) = a + \langle g \rangle.
  \]
  (Left as an exercise to check that $\varphi$ is a
  homomorphism.) We need to show that $\varphi$ is
  injective. For this, take $a, b \in K$. If
  $a + \langle g \rangle = b + \langle g \rangle$,
  then $a - b \in \langle g \rangle$. But $K$ is a field,
  so this happens precisely when $a = b$. Thus
  $\varphi$ embeds $K$ into $K[X] / \langle g \rangle$,
  as desired.
\end{proof}

\section{\texorpdfstring{Irreducible Polynomials over $\C$, $\R$, $\Q$, and $\Z$}{Irreducible Polynomials over C, R, Q, and Z}}
Our goal now is to study irreducible polynomials. Note
that linear polynomials are irreducible, and recall that
every polynomial in $\C$ factorizes, essentially
uniquely, into linear factors. Furthermore, complex roots
of real polynomials come in conjugate pairs, hence
\[
  g = a_0 + a_1 X + \dots + a_n X^n \in \R[X]
\]
factors as
\[g = a_n (X - \beta_1) \dots (X - \beta_r) (X - \gamma_1)(X - \overline{\gamma}_1) \dots (X - \gamma_3) (X - \overline{\gamma}_s)\]
in $\C[X]$, where $\beta_1, \dots, \beta_r \in \R$ and $\gamma_1, \dots, \gamma_s \in \C \setminus \R$ and
$r + 2s = n$. Thus over $\R[X]$, $g$ factors as
\[
  g = a_n (X - \beta_1) \dots (X - \beta_r) (X^2 - (\gamma_1 + \overline{\gamma}_1) X + \gamma_1 \overline{\gamma}_1) \dots (X^2 - (\gamma_s + \overline{\gamma}_s) X + \gamma_s \overline{\gamma}_s)
\]
in $\R[X]$, where the quadratic factors are irreducible
in $\R[X]$.

\begin{exercise}
  A quadratic $a X^2 + bX + c \in \R[X]$ is irreducible
  if and only if its discriminant $b^2 - 4ac < 0$.
\end{exercise}

Now we have pretty much characterized irreducible
polynomials in $\R[X]$. But what about $\Q[X]$?

\begin{theorem}
  Let $g = a_0 + a_1 X + a_2 X^2 \in \Q[X]$. Then
  \begin{enumerate}
    \item If $g$ is irreducible over $\R$,
      then it is irreducible over $\Q$.
    \item If $g = a_2 (X - \beta_1) (X - \beta)$ with
      $\beta_1, \beta_2 \in \R$, then $g$ is irreducible
      in $\Q[X]$ if and only if $\beta_1$ and $\beta_2$
      are irrational.
  \end{enumerate}
\end{theorem}

\begin{proof}
  (1) We show the contrapositive.
  If $g$ factors as
  \[
    g = a_2 (X - q_1) (X - q_2) \in \Q[X],
  \]
  then $g$ also factors in $\R[X]$.

  (2) If $\beta_1$ and $\beta_2$ are rational, then
  $g$ factors in $\Q[X]$ and is thus not irreducible.
  For the other direction, if $\beta_1$ and $\beta_2$
  are irrational, then $g = a_2 (X - \beta_1) (X - \beta_2)$
  is the only factorization in $\R[X]$ since $\R[X]$
  is a UFD, so there is no factorization in $\Q[X]$
  into linear factors.
\end{proof}

\begin{example}
  Are the following polynomials irreducible in $\R[X]$?
  In $\Q[X]$?
  \begin{enumerate}
    \item $X^2 + X + 1$ is irreducible over $\R$ and $\Q$
      since $b^2 - 4ac = -3$.
    \item $X^2 - X - 1$ has roots
      $(-1 \pm \sqrt{5}) / 2$, so it factors over $\R$
      but is irreducible over $\Q$.
    \item $X^2 + X - 2$ factors as $(X + 2)(X - 1)$
      over $\R$ and $\Q$.
  \end{enumerate}
\end{example}

Now that we have studied irreducible polynomials in
$\R[X]$ and $\Q[X]$, can a
polynomial in $\Z[X]$ be irreducible over $\Z$ but not
$\Q$? The answer is no!

\begin{theorem}[Gauss's lemma]
  Let $f$ be a polynomial in $\Z[X]$, irreducible
  over $\Z$. Then $f$ is irreducible over $\Q$.
\end{theorem}

\begin{proof}
  For sake of contradiction, suppose $f = gh$ with
  $g, h \in \Q[X]$ and $\partial g, \partial h < \partial f$.
  Then there exists $n \in \Z_{> 0}$ such that
  $nf = g' h'$ where $g', h' \in \Z[X]$. Let $n$ be the
  smallest positive integer with this property. Let
  \begin{align*}
    g' &= a_0 + a_1 X + \dots + a_k X^k \\
    h' &= b_0 + b_1 X + \dots + b_l X^l.
  \end{align*}
  If $n = 1$, then $g' = g$ and $h' = h$, a contradiction.
  Now $n \ge 1$, so let $p$ be a prime factor of $n$.\footnote{Lemma: Either $p$ divides all the coefficients of $g'$ or $p$ divides all the coefficients of $h'$. Proof left as an exercise.}
  Without loss of generality, assume $p$ divides $g'$,
  i.e. $g' = p g''$ where $g'' \in \Z[X]$. Then
  \[\frac{n}{p} f = g'' h',\]
  contradicting the minimality of $n$. Hence $f$
  cannot be factored over $\Q$.
\end{proof}

\begin{example}
  Show that $g = X^3 + 2X^2 + 4X - 6$ is irreducible
  over $\Q$.
\end{example}

\begin{proof}
  If $g$ factors over $\Q$, it factors over $\Z$ and
  at least one factor must be linear, i.e.
  \[g = X^3 = 2X^2 + 4X - 6 = (X - a)(X^2 + bX + c)\]
  where $a, b, c \in \Z$. We must have $ac = 6$, so
  $a \in \{\pm 1, \pm 2, \pm 3, \pm 6\}$ and $g(a) = 0$.
  We can check this:
  \begin{center}
    \begin{tabular}{c|cccccccc}
      $a$ & $1$ & $-1$ & $2$ & $-2$ & $3$ & $-3$ & $-6$ & $6$ \\
      \hline
      $g(a)$ & $1$ & $-9$ & $1$ & $-10$ & $51$ & $-27$ & $306$ & $-174$
    \end{tabular}
  \end{center}
  Hence $g$ is irreducible over $\Z$ and thus also
  irreducible over $\Q$.
\end{proof}

We could do this trick since the degree was 3,
forcing a linear factor.
What about degrees higher than 3?

\begin{theorem}[Eisenstein's criterion]
  Let $f = a_0 + a_1 X + \dots + a_n X^n \in \Z[X]$.
  Suppose there exists a prime $p$ such that
  \begin{enumerate}
    \item $p {\nmid} a_n$,
    \item $p | a_i$ for $i = 0, \dots, n - 1$,
    \item $p^2 {\nmid} a_0$.
  \end{enumerate}
  Then $f$ is irreducible over $\Q$.
\end{theorem}

\begin{proof}
  By Gauss's lemma, it suffices to show that $f$ is
  irreducible over $\Z$. Suppose for sake of contradiction
  that $f = gh$ for
  \[
    g = b_0 + b_1 X + \dots + b_r X^r \quad \text{and} \quad
    h = c_0 + c_1 X + \dots + c_s X^s,
  \]
  $r, s < n$, and $r + s = n$. Note that
  $a_0 = b_0 c_0$, so $p | a_0$ from (2) implies that
  $p | b_0$ or $p | c_0$. Since $p^2 {\nmid} a_0$,
  it cannot be both. Without loss of generality, assume
  $p | b_0$ and $p {\nmid} c_0$. Now suppose inductively
  that $p$ divides $b_0, \dots, b_{k - 1}$ where
  $1 \le k \le r$. Then
  \[a_k = b_0 c_k + b_1 c_{k - 1} + \dots + b_{k - 1} c_1 + b_k c_0\]
  and since $p$ divides $a_k$, $b_0 c_k$,
  $b_1 c_{k - 1}$, \dots, $b_{k - 1} c_1$, it follows that
  $p | b_k c_0$. Since $p {\nmid} c_0$ by assumption, we
  must have $p | b_k$. Thus
  $p | b_r$ and since $a_n = b_r c_s$, we have
  $p | a_n$, contradicting (1). Hence is $f$ is
  irreducible.
\end{proof}

\begin{example}
  The polynomial
  \[
    X^5 + 2X^3 + \frac{8}{7} X^2 - \frac{4}{7} X + \frac{2}{7}
  \]
  is irreducible over $\Q$.
\end{example}

\begin{proof}
  Multiply by $7$ and take the integer polynomial
  $7X^5 + 14X^3 + 8X^2 - 4X + 2$. Taking
  $p = 2$ satisfies Eisenstein's criterion, so this
  polynomial is irreducible over $\Z$ and thus also
  irreducible over $\Q$.
\end{proof}

\begin{example}
  If $p > 2$ is prime, then show that
  \[f = 1 + X + X^2 + \dots + X^{p - 1}\]
  is irreducible over $\Q$.
\end{example}

\begin{proof}
  First observe that
  \[f = \frac{X^p - 1}{X - 1}.\]
  Let $g(X) = f(X + 1)$. Then
  \begin{align*}
    g(X)
    &= \frac{(X + 1)^p - 1}{(X + 1) - 1}
    = \frac{1}{X} ((X + 1)^p - 1)
    = \frac{1}{X} \sum_{i = 0}^p \binom{p}{i} X^{p - i} - 1 \\
    &= \frac{1}{X} \sum_{i = 0}^{p - 1} \binom{p}{i} X^{p - i}
    = \sum_{i = 0}^{p - 1} \binom{p}{i} X^{p - i - 1}.
  \end{align*}
  Note that $\binom{p}{1}, \binom{p}{2}, \dots \binom{p}{p - 1}$
  are all divisible by $p$, so $g$ is irreducible by
  Eisenstein's criterion. Now if $f$ factors as
  $f = uv$, then $g(X) = u(X + 1)v(X + 1)$, which
  is a contradiction since $g$ is irreducible.
\end{proof}

  \chapter{Jan.~22 --- Field Extensions}
\section{More on Irreducibility}
The following excerpt is from Howie:
\begin{quote}
  Another device for determining irreducibility over $\Z$ (and consequently over $\Q$) is to map the polynomial onto $\Z_p[X]$ for some suitably chosen prime $p$. Let $g = a_0 +a_1X + \dots +a_nX^n \in \Z[X]$, and let $p$ be a prime not dividing $a_n$. For each $i$ in $\{0, 1, \dots, n\}$, let $\overline{a}_i$ denote the residue class $a_i + \langle p \rangle$ in the field $\Z_p = \Z / \langle p \rangle$, and write the polynomial $\overline{a}_0 + \overline{a}_1X + \dots + \overline{a}_n X^n$ as $\overline{g}$. Our choice of $p$ ensures that $\partial \overline{g} = n$. Suppose that $g = uv$, with $\partial u, \partial v < \partial f$ and $\partial u + \partial v = \partial g$. Then $\overline{g} = \overline{u}\,\overline{v}$. If we can show that $\overline{g}$ is irreducible in $\Z_p[X]$, then we have a contradiction, and we deduce that $g$ is irreducible. The advantage of transferring the problem from $\Z[X]$ to $\Z_p[X]$ is that $\Z_p$ is finite, and the verification of irreducibility is a matter of checking a finite number of cases.
\end{quote}

\begin{example}
  Show that
  \[g = 7X^4 + 10X^3 - 2X^2 + 4X - 5\]
  is irreducible over $\Q$.
\end{example}

\begin{proof}
  Let $p = 3$ and
  \[\overline{g} = X^4 + X^3 + X^2 + 1\]
  This has no linear factors since
  \[\overline{g}(0) = 1, \quad \overline{g}(1) = 2, \quad \overline{g}(-1) = 1.\]
  So suppose
  \[
    \overline{g} = X^4 + X^3 + X^2 + X + 1
    = (X^2 + aX + b)(X^2 + cX + d)
  \]
  in $\Z_3[x]$. Then for some
  $a, b, c, d \in \Z_3 = \{-1, 0, 1\}$, we have
  \[
    \begin{cases}
      X^3 & a + c = 1 \\
      X^2 & b + ac + d = 1 \\
      X & ad + bc = 1 \\
      1 & bd = 1
    \end{cases}
  \]
  The first case is if $b = d = 1$, but this
  implies $ac = -1$, so $a = \pm 1$ and $c = \mp 1$.
  But $a + c = 1$, so this cannot happen. The
  second case is if $b = d = -1$. This implies
  that $ac = 0$ and $a + c = 1$. So if $a = 0$,
  then $c = 1$, so $1 = ad + bc = b$, which is
  a contradiction with $b = -1$. If $c = 0$,
  then $1 = ad + bc = d$, which is a contradiction
  with $d = -1$. Thus $\overline{g}$ is irreducible
  in $\Z_3[x]$, so $g$ is irreducible in $\Z[x]$,
  and by Gauss's lemma, $g$ is irreducible in $\Q[x]$.
\end{proof}

\begin{remark}
  If we had tried $p = 2$, then we have
  $\overline{g} = x^4 + 1 \in \Z_2[x]$, which is
  not in fact irreducible since
  \[\overline{g} = x^4 + 1 = (x + 1)^4 \in \Z_2[x].\]
\end{remark}

\section{Field Extensions}
\begin{definition}
  Let $K, L$ be fields and $\varphi : K \to L$ an
  injective homomorphism. Then $L$ is a
  \emph{field extension} of $K$, denoted $L : K$.
\end{definition}

\begin{example}
  We have $\C : \R$ is a field extension.
\end{example}

\begin{definition}
Recall that $V$ is a \emph{$K$-vector space} if
\begin{enumerate}
  \item $V$ is an abelian group under $+$,
  \item For $a, b \in K$ and $x, y \in V$, we have
    \[\text{(i). } a(x + y) = ax + ay, \quad \text{(ii). }(a + b)x = ax + bx, \quad \text{(iii). } (ab)x = a(bx), \quad \text{(iv). } 1x = x.\]
\end{enumerate}
\end{definition}

\begin{remark}
  If $L : K$ is a field extension, then $L$ is a
  a vector space over $K$.
\end{remark}

\begin{definition}
  A \emph{basis} for a vector space is a linearly
  independent spanning set.
\end{definition}

\begin{example}
  The complex numbers $\C$ is a $\R$-vector space
  with basis $\{1, i\}$. Bases are not unique,
  since $\{1 + i, 1 - i\}$ is another basis for $\C$.
\end{example}

\begin{example}
  If there is a vector space that we know to be a field,
  then it is automatically a field extension of its
  ground field.
\end{example}

\begin{definition}
  The \emph{dimension} of $L$
  is the cardinality of a basis for $L : K$.\footnote{Note that this is well-defined since any two bases of $L$ have the same length.}
  The dimension is also called the \emph{degree} of
  $L : K$, denoted $[L : K]$. We say that $L$ is a
  \emph{finite extension} if $[L : K]$ is finite, and
  an \emph{infinite extension} otherwise.
\end{definition}

\begin{example}
  We have $[\C : \R] = 2$, which is finite. On the
  other hand, $\R : \Q$ is an infinite extension.
\end{example}

\begin{theorem}
  Let $L : K$ be a field extension. Then
  $L = K$ if and only if $[L : K] = 1$.
\end{theorem}

\begin{proof}
  $(\Rightarrow)$ If $L = K$, then $\{1\}$ is a basis
  for $L : K$, and thus $[L : K] = 1$.

  $(\Leftarrow)$ If $[L : K] = 1$, then $\{x\}$ is a
  basis for $L : K$ for some $x \in L$. Then there exists
  some $a \in K$ such that $1 = ax$, so
  $x = a^{-1} \in K$. For every $y \in L$, there exists
  $b \in K$ such that $y = bx$. But then
  \[
    y = bx = b(a^{-1}) \in K,
  \]
  so $y \in K$ as well by closure. Thus $L = K$
  as desired.
\end{proof}

\begin{remark}
Let $L : K$ and $M : L$ be field extensions with
\begin{center}
  \begin{tikzcd}
    K \arrow[r, "\alpha"] & L \arrow[r, "\beta"] & M
  \end{tikzcd}
\end{center}
Then $M : K$ is also a field extension.
\end{remark}

\begin{theorem}
  For field extensions $L : K$ and $M : L$, we have
  $[M : L][L : K] = [M : K]$.
\end{theorem}

\begin{proof}
  Suppose $\{a_1, a_2, \dots a_r\}$ is a linearly
  independent subset of $M$ over $L$ and
  $\{b_1, b_2, \dots, b_s\}$ is a linearly
  independent subset of $L$ over $K$.
  Now we claim that
  \[
    \{a_i b_j \mid 1 \le i \le r, 1 \le j \le s\}
  \]
  is a linearly independent subset of $M$ over $K$.
  To see this, suppose
  \[
    \sum_{i = 1}^r \sum_{j = 1}^s \lambda_{ij} a_i b_i = 0
  \]
  for some $\lambda_{ij} \in K$. We can rewrite this as
  \[
    \sum_{i = 1}^r \left(\sum_{j = 1}^s \lambda_{ij} b_j\right) a_i = 0.
  \]
  Since the $a_i$ are linearly independent over $L$,
  it follows that
  \[
    \sum_{j = 1}^s \lambda_{ij} b_j = 0
  \]
  for each $i = 1, \dots, r$. Since the $b_j$ are
  linearly independent over $K$, it follows that
  $\lambda_{ij} = 0$ for each $i, j$, which proves
  the claim. Returning to the main proof,
  if $[M : L]$ or $[L : K]$ is infinite,
  then $r$ or $s$ can be made arbitrarily large,
  so
  \[
    \{a_i b_j \mid 1 \le i \le r, 1 \le j \le s\}
  \]
  can also be made arbitrarily large, and hence
  $[M : K]$ is infinite. Now suppose
  $[M : L] = r < \infty$ and $[L : K] = s < \infty$. Let
  $\{a_1, a_2, \dots, a_r\}$ be a basis for $M : L$ and
  $\{b_1, b_2, \dots, b_s\}$ be a basis for $L : K$.
  We will show that
  \[
    \{a_i b_j \mid 1 \le i \le r, 1 \le j \le s\}
  \]
  is a basis for $M : K$. Since we already showed that
  $\{a_i b_j\}$ is linearly independent, it only remains
  to show that they span $M$ over $K$. For each $z \in M$,
  there exist $\lambda_1, \dots, \lambda_r \in L$ such
  that
  \[z = \sum_{i = 1}^r \lambda_i a_i.\]
  Then for each $\lambda_i \in L$, there exist
  $\mu_{i 1}, \dots, \mu_{i s} \in K$ such that
  \[
    \lambda_i = \sum_{j = 1}^s \mu_{i j} b_j.
  \]
  Combining this yields
  \[
    z = \sum_{i = 1}^r \sum_{j = 1}^s \mu_{i j} a_i b_j
  \]
  as desired, which finishes the proof.
\end{proof}

\begin{example}
  Consider $\Q(\sqrt{2}) = \Q[\sqrt{2}] = \{a + b\sqrt{2} \mid a, b \in \Q\}$.
\end{example}

\begin{exercise}
  Show that $\Q[\sqrt{2}]$ is a field.
  (Hint: $1 / (a + b\sqrt{2}) = (a - b\sqrt{2}) / (a^2 - 2b^2)$.)
\end{exercise}

\begin{definition}
  Let $K$ be a subfield of $L$ and $S$ a subset of
  $L$. The \emph{subfield of $L$ generated over $K$ by $S$},
  denoted $K(S)$, is the intersection of all subfields
  of $L$ containing $K \cup S$. If
  $S = \{\alpha_1, \dots, \alpha_n\}$ is finite,
  we write $K(\alpha_1, \dots, \alpha_n)$.
\end{definition}

\begin{theorem}
  Let $E$ be the elements in $L$ that can be expressed
  as quotients of finite $K$-linear combinations of
  finite products of elements in $S$. Then $K(S) = E$.
\end{theorem}

\begin{proof}
  To see that $K(S) \subseteq E$, simply check that $E$
  is a subfield of $L$ containing $K \cup S$.

  For $E \subseteq K(S)$, note that any subfield of $L$
  containing $K$ and $S$ must contain all finite products
  of elements in $S$, all linear combinations of
  such products, and all quotients of such linear
  combinations. This is precisely what is means
  to have $E \subseteq K(S)$.
\end{proof}

\begin{definition}
  A \emph{simple extension} of $K$ is $K(\alpha)$,
  i.e. $S$ has a single element $\alpha \notin K$.
\end{definition}

\begin{example}
  The previous example $\Q(\sqrt{2})$ is a simple
  extension.
\end{example}

\begin{theorem}
  Let $L$ be a field, $K$ a subfield, and $\alpha \in L$.
  Then either
  \begin{enumerate}
    \item $K(\alpha)$ is isomorphic to $K(X)$, the
      field of rational forms with coefficients in $K$,
    \item or there exists a unique monic polynomial
      $m \in K[X]$ with the property that for all
      $f \in K[X]$,
      \begin{enumerate}
        \item $f(\alpha) = 0$ if and only if $m | f$,
        \item the field $K(\alpha)$ coincides with
          $K[\alpha]$, the ring of all polynomials
          in $\alpha$ with coefficients in $K$,
        \item and $[K[\alpha] : K] = \partial m$.
      \end{enumerate}
  \end{enumerate}
\end{theorem}

\begin{proof}
  Suppose there does not exist nonzero $f \in K[X]$ such
  that $f(\alpha) = 0$. Then there exists a map
  $\varphi : K(X) \to K(\alpha)$ with
  $f / g \mapsto f(\alpha) / g(\alpha)$, which is
  defined since $g(\alpha) = 0$ only if $g$ is the
  zero polynomial. Note that $\varphi$ is a
  surjective homomorphism,\footnote{Also check that $\varphi$ is well-defined.} which one can check as
  an exercise. Now we show that $\varphi$ is also
  injective. To see this, suppose
  \[
    \varphi(f / g) = \varphi(p / q),
  \]
  which happens if and only if
  \[
    f(\alpha) q(\alpha) - p(\alpha) g(\alpha) = 0.
  \]
  in $L$. This happens if and only if $fq - pg = 0$
  in $K[X]$, which happens if and only if $f / g = p / q$
  in $K(X)$. This completes the first case of the theorem.

  Now suppose there exists nonzero $g \in K[X]$ such
  that $g(\alpha) = 0$. Furthermore, suppose $g$ is a
  polynomial of least degree with this property. Let
  $a$ be the leading coefficient of $g$, and let
  $m = g / a$, so that $m$ is monic and $m(\alpha) = 0$
  still. The reverse implication in (2a) is clear. For
  the forwards implication in (2a), note that by division
  with remainder for polynomials over a field, we can
  write
  \[
    f = qm + r,
  \]
  where $\partial r < \partial m$. By the minimality
  of $\partial m$, we must have $r = 0$, so $m | f$.
  For the uniqueness of $m$, suppose there exists $m'$
  with the same properties. Then
  $m(\alpha) = m'(\alpha) = 0$, so
  $m | m'$ and $m' | m$, which implies that $m = m'$
  since $m$ and $m'$ are monic. For the irreducibility
  of $m$, suppose for the sake of contradiction that
  $m = pq$ with $\partial p, \partial q < \partial m$.
  Then $m(\alpha) = p(\alpha) q(\alpha) = 0$, so
  either $p(\alpha) = 0$ or $q(\alpha) = 0$, which
  contradicts the minimality of $\partial m$.

  Now we show (2b), which says that
  $K(\alpha) = K[\alpha]$. For this, consider
  $p(\alpha) / q(\alpha) \in K(\alpha)$ for
  $q(\alpha) \ne 0$. Then $m {\nmid} q$, and since
  $m$ is irreducible we have $\gcd(m, q) = 1$. Now
  by Theorem 2.15 of Howie (about gcd's in the
  Euclidean domain $K[X]$), there exist polynomials
  $a, b$ such that $aq + bm = 1$. Setting
  $X = \alpha$ yields $a(\alpha) q(\alpha) = 1$, so
  \[
    \frac{p(\alpha)}{q(\alpha)} = p(\alpha) a(\alpha) \in K[\alpha].
  \]
  Thus $K(\alpha) \subseteq K[\alpha]$. Since we already
  know that $K[\alpha] \subseteq K(\alpha)$, we conclude
  that $K(\alpha) = K[\alpha]$.

  Finally we show (2c), which claims that
  $[K[\alpha] : K] = \partial m$. For this, suppose
  $\partial m = n$ and let
  \[
    p(\alpha) \in K[\alpha] = K(\alpha).
  \]
  Then $p = qm + r$ where $\partial r < \partial m = n$.
  We have $p(\alpha) = r(\alpha)$, so if
  \[
    r = c_0 + c_1 X + \dots + c_{n - 1} X^{n - 1}
  \]
  for $c_i \in K$, then
  \[
    p(\alpha) = c_0 + c_1 \alpha + \dots + c_{n - 1} \alpha^{n - 1}.
  \]
  So $\{1, \alpha, \dots, \alpha^{n - 1}\}$ is a spanning
  set for $K[\alpha]$. To see that
  $\{1, \alpha, \dots, \alpha^{n - 1}\}$ is also
  linearly independent, suppose there exists $a_i \in K$
  such that
  \[
    a_0 + a_1 \alpha + \dots + a_{n - 1} \alpha^{n - 1} = 0.
  \]
  Then $a_0 = \dots = a_{n - 1} = 0$ since otherwise
  we would have a polynomial
  \[
    p = a_0 + a_1 X + \dots + a_{n - 1} X^{n - 1}
  \]
  with $\partial p \le n - 1$ and $p(\alpha) = 0$,
  which is a contradiction with the minimality of
  $\partial m = n$. Thus
  $\{1, \alpha, \dots, \alpha^{n - 1}\}$ is a basis, and
  so $[K[\alpha] : K] = n = \partial m$.
\end{proof}

\begin{example}
  Continuing the same example, note that
  \[
    \Q[\sqrt{2}] = \{a + b\sqrt{2} \mid a, b \in \Q\}
    = \{a_0 + a_1 \sqrt{2} + a_2 \sqrt{2}^2 + a_3 \sqrt{2}^3 + \dots + a_n \sqrt{2}^n \mid a_i \in \Q\},
  \]
  which falls in the second case of the previous theorem.
\end{example}

\begin{remark}
  We also have $\Q[\sqrt{2}] = \Q[X] / \langle X^2 - 2 \rangle$.
\end{remark}

  \chapter{Jan.~24 --- Algebraic Extensions}

\section{Minimal Polynomials}
\begin{remark}
  The $m$ in the previous theorem from last class is
  called the \emph{minimal polynomial} of $\alpha$.
\end{remark}

\begin{example}
  Let 
  \[\Q[i\sqrt{3}] = \{a + bi\sqrt{3} \mid a, b \in \Q\} \subseteq \C.\]
  Here $m = X^2 + 3$, so this is a degree 2 extension.
\end{example}

\begin{exercise}
Write $1 / (a + bi\sqrt{3})$ in the form $c + di\sqrt{3}$.
\end{exercise}

\begin{example}
  Is $\Q(\sqrt{2}, \sqrt{3})$ a simple extension?
  In fact it is! Note that certainly
  \[
    \Q(\sqrt{2} + \sqrt{3}) \subseteq \Q(\sqrt{2}, \sqrt{3}).
  \]
  For the reverse inclusion, observe that
  $(\sqrt{3} + \sqrt{2})(\sqrt{3} - \sqrt{2}) = 1$,
  so
  \[1 / (\sqrt{3} + \sqrt{2}) = \sqrt{3} - \sqrt{2} \in \Q(\sqrt{2} + \sqrt{3}).\]
  From this we have
  \[
    (\sqrt{3} + \sqrt{2}) + (\sqrt{3} - \sqrt{2})
    = 2\sqrt{3},
  \]
  which implies that $\sqrt{3} \in \Q(\sqrt{2} + \sqrt{3})$.
  Similarly $\sqrt{2} \in \Q(\sqrt{2} + \sqrt{3})$, so that
  $\Q(\sqrt{2}, \sqrt{3}) \subseteq \Q(\sqrt{2} + \sqrt{3})$.
  Now we can consider
  \[\Q(\sqrt{2}, \sqrt{3}) = \Q[\sqrt{2}, \sqrt{3}] = (\Q[\sqrt{2}])[\sqrt{3}].\]
  First we have $[Q[\sqrt{2}]:\Q] = 2$. Note that
  $X^2 - 3$ is the minimal polynomial of
  $\sqrt{3}$ over $\Q[\sqrt{2}]$, so
  $[\Q[\sqrt{2}, \sqrt{3}]:\Q[\sqrt{2}]] = 2$.
  Hence $[\Q[\sqrt{2}, \sqrt{3}]:\Q] = 4$ with
  basis $\{1, \sqrt{2}, \sqrt{3}, \sqrt{6}\}$.\footnote{Since $\Q[\sqrt{2}, \sqrt{3}] = \Q[\alpha]$ where $\alpha = \sqrt{2} + \sqrt{3}$, we have $\{1, \alpha, \alpha^2, \alpha^3\}$ as another basis.}
  To find the  minimal polynomial of
  $\sqrt{2} + \sqrt{3}$ over $\Q$, we can compute
  \begin{align*}
    (\sqrt{2} + \sqrt{3})^2
    &= 2 + 2\sqrt{6} + 3 = 5 + 2\sqrt{6} \\
    (\sqrt{2} + \sqrt{3})^4
    &= 25 + 20\sqrt{6} + 24 = 49 + 20\sqrt{6}.
  \end{align*}
  Thus $X^4 - 10X^2 + 1$ is the minimal polynomial,
  since $\alpha^4 - 10\alpha^2 + 1 = 0$ for
  $\alpha = \sqrt{2} + \sqrt{3}$.
\end{example}

\section{Algebraic Extensions}
\begin{definition}
  If $\alpha$ has a minimal polynomial over $K$,
  we say $\alpha$ is \emph{algebraic} over $K$, and
  $K[\alpha] = K(\alpha)$ is an \emph{algebraic extension} of $K$.
  A complex number that is algebraic over $\Q$
  is called an \emph{algebraic number}.
  Otherwise, if $K(\alpha) \cong K(X)$, then we say
  $\alpha$ is
  \emph{transcendental} over $K$. A transcendental
  number $\alpha$ is a complex number that is
  transcendental over $\Q$.
\end{definition}

\begin{example}
  We have that $\Q(i\sqrt{3})$, $\Q(\sqrt{2})$, $\Q(\sqrt{3})$, and
  $\Q(\sqrt{2}, \sqrt{3})$ are all simple algebraic
  extensions of $\Q$, whereas $\Q(X)$ is a simple
  transcendental extension of $\Q$.
\end{example}

\begin{theorem}
  Let $K(\alpha)$ be a simple transcendental extension
  of $K$. Then $[K(\alpha) : K] = \infty$.
\end{theorem}

\begin{proof}
  Observe that $1, \alpha, \alpha^2, \dots$ are
  linearly independent over $K$, since no minimal
  polynomial exists.
\end{proof}

\begin{definition}
  An extension $L$ over $K$ is an \emph{algebraic extension}
  if any element of $L$ is algebraic over $K$. Otherwise,
  $L$ is a \emph{transcendental extension}.
\end{definition}

\begin{theorem}
  Every finite extension is algebraic.
\end{theorem}

\begin{proof}
  Let $L : K$ be a finite extension and suppose for
  sake of contradiction that $\alpha \in L$ is
  transcendental over $K$. Then
  $1, \alpha, \alpha^2, \dots$ are linearly independent,
  contradicting the fact that $L : K$ is finite.
\end{proof}

\begin{theorem}
  Let $L : K$ be a field extension and let
  $\mathcal{A}(L)$ be the set of elements in $L$ that
  are algebraic over $K$. Then $\mathcal{A}(L)$ is a
  subfield of $L$.
\end{theorem}

\begin{proof}
  See Howie. Just need to show the closure of
  algebraic elements under usual field operations.
\end{proof}

\begin{example}
  For $L = \C$ and $K = \Q$, we have that
  $\mathcal{A}(\C)$ is
  the field $\mathbb{A}$ of algebraic numbers.
\end{example}

\begin{theorem}
  The set of algebraic numbers $\mathbb{A}$ is countable.
\end{theorem}

\begin{proof}[Proof sketch]
  Note that the set of monic polynomials of degree
  $n$ with coefficients in $\Q$ is countable, and
  each such polynomial has at most $n$ distinct roots
  in $\C$. Hence the number of roots of such polynomials
  is countable. Then $\mathbb{A}$ is the countable
  union of countable sets, so $\mathbb{A}$ is countable.
\end{proof}

\begin{theorem}
  Transcendental numbers exist.
\end{theorem}

\begin{proof}
  Since $|\R| = |\C| = 2^{\aleph_0} > \aleph_0$, we must
  have that $\C \setminus \mathbb{A}$ is nonempty.
\end{proof}

\begin{remark}
  The above proof is very nonconstructive, what about
  actual examples of transcendental numbers? In
  1844, Liouville constructed the following example:
  \[
    \sum_{n = 1}^\infty 10^{-n!},
  \]
  which was shown to be transcendental. In 1873,
  Hermite showed that $e$ is transcendental, and in 1882,
  Lindemann showed that $\pi$ is transcendental.
\end{remark}

\begin{theorem}
  Let $L : K$ be a field extension and
  $\alpha_1, \dots, \alpha_n \in L$ have minimal
  polynomials $m_1, \dots, m_n$, respectively.
  Then
  $[K(\alpha_1, \dots, \alpha_n) : K] \le \partial m_1 \partial m_2 \dots \partial m_n$.
\end{theorem}

\begin{proof}
  See Howie. Uses induction and the fact that
  $[M : L][L : K] = [M : K]$.
\end{proof}

\begin{example}
  Consider
  \[
    [\Q[\sqrt{2}] : \Q] = [\Q[\sqrt{3}] : \Q]
    = [\Q[\sqrt{6}] : \Q] = 2,
  \]
  but $[\Q[\sqrt{2}, \sqrt{3}, \sqrt{6}] : \Q] = 4$.
  So the bound in the previous theorem cannot be made
  into an equality.
\end{example}

\begin{prop}
  A field extension $L : K$ is finite if and only if
  for some $n$, there exist $\alpha_1, \dots, \alpha_n$
  algebraic over $K$ such that
  $L = K(\alpha_1, \dots, \alpha_n)$.
\end{prop}

\begin{proof}
  $(\Leftarrow)$ This is precisely the previous theorem.

  $(\Rightarrow)$ Suppose $L : K$ is finite and
  $\{\alpha_1, \dots, \alpha_n\}$ is a basis for $L$
  over $K$. Since finite extensions are algebraic,
  the $\alpha_i$ must be algebraic.
\end{proof}

\begin{exercise}
  Show that $\varphi : \Q[\sqrt{2}] \to \Q[X] / \langle X^2 - 2 \rangle$
  defined by
  \[a + b\sqrt{2} \mapsto a + bX + \langle X^2 - 2 \rangle\]
  is an isomorphism.
\end{exercise}

\begin{theorem}
  Let $K$ be a field and $m$ a monic irreducible
  polynomial in $K[X]$. Then $L = K[X] / \langle m \rangle$
  is a simple algebraic extension $K[\alpha]$ of $K$,
  and $\alpha = X + \langle m \rangle$ has minimal
  polynomial $m$ over $K$.
\end{theorem}

\begin{proof}
  First note that $L$ is indeed a field since $m$
  is irreducible. Also $L : K$ is indeed a field
  extension since $\varphi : K \to L$ defined by
  $a \mapsto a + \langle m \rangle$ is an injective
  homomorphism. Now let $\alpha = X + \langle m \rangle$.
  For
  \[
    f = a_0 + a_1 X + \dots + a_n X^n \in K[X],
  \]
  we have
  \begin{align*}
    f(\alpha)
    &= a_0 + a_1 \alpha + \dots + a_n \alpha^n
    = a_0 + a_1 (X + \langle m \rangle) + \dots + a_n (X + \langle m \rangle)^n \\
    &= a_0 + a_1 X + \dots + a_n X^n + \langle m \rangle
    = f + \langle m \rangle.
  \end{align*}
  So $f(\alpha) = 0$ if and only if
  $f \in \langle m \rangle$, i.e. $m | f$. Hence
  $m$ is the minimal polynomial of $\alpha$.
\end{proof}

  \chapter{Jan.~29 --- Geometric Constructions}

\section{\texorpdfstring{$K$-Isomorphisms}{K-Isomorphisms}}
Recall from last class that $L = K[X] / \langle m \rangle$
is a simple algebraic extension of $K$. In fact, we
can show that the field $L$ is essentially unique,
i.e. unique up to isomorphism.
\begin{theorem}
  \label{thm:isomorphism-fix}
  Let $K$ be a field and and $f$ and an irreducible
  polynomial in $K[X]$. If $L$ and $L'$ are two extensions
  of $K$ containing roots $\alpha$ and $\alpha'$
  respectively of $f$, then there exists an isomorphism
  $K[\alpha] \to K[\alpha']$ which fixes every element
  of $K$.
\end{theorem}

\begin{proof}[Proof sketch]
  Suppose
  \[
    f = a_0 + a_1 X + \dots + a_n X^n.
  \]
  Then $K[\alpha]$ consists of polynomials of the form
  \[
    b_0 + b_1 \alpha + \dots + b_{n-1} \alpha^{n-1}.
  \]
  This is because multiplication in $K[\alpha]$ relies on
  the observation that
  \[
    \alpha^n = -\frac{1}{\alpha_n} (a_0 + a_1 \alpha + \dots + a_{n-1} \alpha^{n-1})
  \]
  since $\alpha$ is a root of $f$. Define
  $\psi : K[\alpha] \to K[\alpha']$ by
  $\psi(g(a)) = g(\alpha')$ and show that $\psi$
  is an isomorphism.
\end{proof}

\begin{exercise}
  Check the following from the previous proof:
  \begin{enumerate}
    \item $\psi$ is one-to-one and onto,
    \item $\psi$ fixes $K$,
    \item and $\psi$ is a homomorphism.
  \end{enumerate}
  For the last point, the addition is mostly
  straightforward
  but the multiplication is more involved since
  we need to reduce when we get $\alpha^n$ terms
  in the product.
\end{exercise}

\begin{definition}
  A \emph{$K$-isomorphism} is an isomorphism
  $\varphi : L \to L'$
  such that $\varphi(x) = x$ for all $x \in K$.
\end{definition}

\begin{example}
  For $\C : \R$, the complex conjugation map
  $\varphi : \C \to \C$ given by
  $\varphi(a + bi) = a - bi$ is a $\R$-isomorphism.
\end{example}

\begin{example}
  For $\Q[X] / \langle X^2 + 3 \rangle : \Q$,\footnote{Note that $\Q[X] / \langle X^2 + 3 \rangle \cong \Q[i\sqrt{3}]$. The isomorphism is given by $a + bX + \langle X^3 + 3 \rangle \mapsto a + bi\sqrt{3}$.}
  the map $\psi : \Q[X] / \langle X^2 + 3 \rangle \to \Q[X] / \langle X^2 + 3 \rangle$ given by
  \[
    \psi(a + bX + \langle X^2 + 3 \rangle) = a - bX + \langle X^2 + 3 \rangle
  \]
  is a $\Q$-isomorphism. The analogous
  map $\psi : \Q[i\sqrt{3}] \to \Q[i\sqrt{3}]$
  given by $\psi(a + bi\sqrt{3}) = a - bi\sqrt{3}$
  also works, which we can view as a restriction
  of the complex conjugation map to $\Q[i\sqrt{3}]$.
\end{example}

\section{Applications to Geometric Constructions}
Consider the straightedge and compass Constructions
from geometry. Let $B_0$ be a set of points. Then
we have the following operations:
\begin{enumerate}
  \item (straightedge) Draw a straight line through any
    two points in $B_0$.
  \item (compass) Draw a circle whose center is a
    point in $B_0$ passing through another point in $B_0$.
\end{enumerate}
Let $C(B_0)$ be the set of points which are intersections
of lines or circles obtained form $B_0$ by (1) and (2).
Let $B_1 = B_0 \cup C(B_0)$, and proceed
inductively to get $B_n = B_{n-1} \cup C(B_{n-1})$.

\begin{definition}
  A point is \emph{constructible from $B_0$} if it belongs
  to $B_n$ for some $n$. A point is \emph{constructible}
  if it is constructible from $\{O, I\}$ where
  $O = (0, 0)$ and $I = (1, 0)$.
\end{definition}

\begin{example}
  To find the midpoint of the line segment $OI$
  from $B_0 = \{O, I\}$, we can do the following:
  \begin{enumerate}
    \item Draw a circle with center $O$ passing
      through $I$.
    \item Draw a circle with center $I$ passing through
      $O$.
    \item Mark points $P$ and $Q$ where these circles
      intersect. So $B_1 \supseteq \{O, I, P, Q\}$.
    \item Draw a line connecting $P$ and $Q$.
    \item Draw a line connecting $O$ and $I$.
    \item Mark the point $M$ where $PQ$ and $OI$ meet.
      So $B_2 \supseteq \{O, I, P, Q, M\}$.
  \end{enumerate}
  Thus $M$ is constructible from $\{O, I\}$.
\end{example}

The algebraic perspective is the following: Associate
to $B_i$ the subfield of $\R$ generated by coordinates
of points in $B_i$, i.e. view each coordinate of
each point as an element and take the subfield
generated.
\begin{example}
For
$B_0 = \{(0, 0), (1, 0)\}$, we have $\{0, 0, 1, 0\} \subseteq K_0 = \Q$
is the subfield of $\R$ generated by the coordinates
of $B_0$. Next take\footnote{There is some abuse of notation here since we take $B_i$ to be only some subset of all the actual possible points.}
\[
  B_1 = \{O, I, P, Q\}
  = \{(0, 0), (1, 0), (1 / 2, \pm \sqrt{3} / 2)\},
\]
so that $K_1 = \Q[\sqrt{3}]$ is the field
generated by $B_1$. Then
\[
  B_2 = \{O, I, P, Q, M\} = \{(0, 0), (1, 0), (1 / 2, \pm \sqrt{3} / 2), (1 / 2, 0)\},
\]
and the field generated by $B_2$ is still $K_2 = \Q[\sqrt{3}]$.
\end{example}

\begin{theorem}
  Let $P$ be a constructible point belonging to $B_n$,
  where $B_0 = \{(0, 0), (1, 0)\}$, and let $K_n$
  be the field generated over $\Q$ by $B_n$. Then
  $[K_n : \Q]$ is a power of $2$.
\end{theorem}

\begin{proof}[Proof sketch]
  We proceed by induction. The base case is
  $K_0 = \Q$, so $[K_0 : \Q] = 1 = 2^0$. Now suppose
  $[K_{n - 1} : \Q] = 2^k$ for some $k \ge 0$, and
  we want to show that $[K_n : K_{n - 1}]$ is a power
  of $2$. Observe that new points in $B_n$ can be
  obtained by
  \begin{enumerate}
    \item intersection of two lines,
    \item intersection of a line and a circle,
    \item or intersection of two circles.
  \end{enumerate}
  In case (1), the intersection of two lines is
  given by solving a system of two linear equations,
  which only involves rational operations\footnote{By rational operations we mean addition, subtraction, multiplication, division.}.
  In other words, this case takes place entirely in
  $K_{n - 1}$.

  In case (2), the intersection of a line
  and a circle is given by solving of a system of one
  linear equation and one quadratic equation. Solving
  the linear equation for one of the variables and
  substituting into the quadratic equation reduces
  the system down to a single quadratic equation
  in a single variable. The solution involves
  $\sqrt{\Delta}$, where $\Delta$ is the discriminant.
  Then the new points are in $K_{n - 1}[\sqrt{\Delta}]$.

  In case (3), the intersection of two circles is
  given by solving a system of two quadratic equations.
  Subtracting the two quadratic equations yields a
  linear equation, which reduces back to case (2).

  Thus the elements in $K_n$ are either in $K_{n - 1}$
  or $K_{n - 1}[\sqrt{\Delta}]$ for some
  $\Delta \in K_{n - 1}$.\footnote{We can set it up so that we only gain one extra intersection, i.e. only one $\Delta$, at each step.} Hence $[K_n : K_{n - 1}]$
  is either $1$ or $2$, so by induction $[K_n : \Q]$ is
  a power of $2$.
\end{proof}

\section{Classic Problems}
\subsection{Duplicating the Cube}
Consider the problem of taking a cube of volume $1$, and
constructing a cube of volume $2$. We need $\alpha$
such that $\alpha^3 = 2$. But $X^3 - 2$ is irreducible
over $\Q$ by Eisenstein's criterion, so
$[\Q[\alpha] : \Q] = 3$. This is not a power of $2$,
so $\alpha$ is not constructible and thus we cannot
duplicate the cube.

\subsection{Trisecting the Angle}
Recall the triple angle formula:
\[\cos 3\theta = 4\cos^3 \theta - 3\cos \theta.\]
Suppose $\cos 3\theta = c$. So to find $\cos \theta$,
we want a root of $4X^3 - 3X - c = 0$. This depends on
$c$.

\begin{example}
If $3\theta = \pi / 2$, then $c = 0$ and
the polynomial factors into
\[4X^3 - 3X = 4X(4X^2 - 3),\]
so $[\Q[\alpha] : \Q] = [\Q[\sqrt{3}] : \Q] = 2$.
So in fact we can trisect $\pi / 2 = 90^\circ$.
\end{example}

\begin{example}
  If $3\theta = \pi / 3$, then $c = 1 / 2$ and
  we have $4X^3 - 3X - 1 / 2$. Let
  \[
    f(X) = 8X^3 - 6X - 1,
  \]
  so that $g(X) = g(X / 2) = X^3 - 3X - 1$. Note
  that $g$ does not factor over $\Z$ since that
  requires a linear factor of $X \pm 1$ but
  $g(\pm 1) \ne 0$. So $g$ is irreducible over $\Z$
  and by Gauss's lemma, $g$ is irreducible over $\Q$.
  Thus $f$ is irreducible. Hence
  $[\Q[\alpha] : \Q] = 3$, so we cannot
  trisect $\pi / 3$ with a straightedge and compass.
\end{example}

  \chapter{Jan.~31 --- Splitting Fields}

\section{Review of Notation}
Recall that
\begin{align*}
  \Q[X] &= \{
    a_0 + a_1 X + \dotsb + a_n X^n : a_i \in \Q
  \} \\
    \Q(X) &= \{
      f / g : f, g \in \Q[X], g \ne 0
    \} / \sim,
\end{align*}
where $\sim$ is the usual relation on fractions, e.g.
$2f / 2g = f / g$. Next, recall that
\[
  \Q[\sqrt{2}] = \{a_0 + a_1 \sqrt{2} + \dots + a_n \sqrt{2}^n : a_i \in \Q\}
  = \{a + b \sqrt{2} : a, b \in \Q\}
\]
since $\sqrt{2}^2 = 2$. Also
$\Q(\sqrt{2})$ is the smallest subfield of $\R$
containing $\Q \cup \{\sqrt{2}\}$. In this case,
$\Q(\sqrt{2}) = \Q[\sqrt{2}]$ since
\[
  \frac{1}{a + b\sqrt{2}} = \frac{a - b\sqrt{2}}{a^2 - 2b^2}.
\]
Next, we have
\begin{align*}
  \Q[X] / \langle X^2 - 2 \rangle
  &= \{
    a_0 + a_1 X + \dots + a_n X^n + \langle X^2 - 2 \rangle : a_i \in \Q
  \} \\
  &= \{
    a + bX + \langle X^2 - 2 \rangle : a, b \in \Q
  \}
\end{align*}
since $X^2 + \langle X^2 - 2 \rangle = 2 + \langle X^2 - 2\rangle$.
In fact, $\Q[X] / \langle X^2 - 2 \rangle \cong \Q[\sqrt{2}]$.\footnote{Here the isomorphism $\Q[X] / \langle X^2 - 2 \rangle \to \Q[\sqrt{2}]$ is given by $a + bX  +\langle X^2 - 2 \rangle \mapsto a + b\sqrt{2}$.}

\section{Splitting Fields}
The motivating question here is: When can we factor
a polynomial into linear factors?

\begin{definition}
  A polynomial \emph{splits completely} over $K$
  if it can be factored into linear factors over $K$.
\end{definition}

\begin{example}
  The polynomial $X^2 + 2$ splits completely over
  $\Q[i\sqrt{2}]$ since $X^2 + 2 = (X - i\sqrt{2})(X + i\sqrt{2})$.
\end{example}

\begin{example}
  The polynomial $X^3 - 2$ is irreducible
  over $\Q$ by Eisenstein's criterion. However,
  it factors as
  \[
    X^3 - 2 = (X - \alpha) (X^2 + \alpha X + \alpha^2)
  \]
  in $\Q[\alpha]$, where $\alpha = \sqrt[3]{2}$.
  Also $X^2 + \alpha X + \alpha^2$ is irreducible
  over $\Q[\alpha]$, since its discriminant shows that
  it is irreducible even over $\R$. But in $\C$, we
  can factor it as
  \[
    X^3 - 2 = (X - \alpha)(X - \alpha e^{2\pi i / 3})(X - \alpha e^{4\pi i / 3}).
  \]
  A smaller field that $X^3 - 2$ splits completely over
  is $\Q[\sqrt[3]{2}, i \sqrt{3}]$.
\end{example}

\begin{definition}
  Let $K$ be a field and $f \in K[X]$. An extension
  $L$ of $K$ is a \emph{splitting field} for $f$ over $K$
  if
  \begin{enumerate}
    \item $f$ splits completely over $L$,
    \item and $f$ does not split completely over any
      subfield $E$ with $K < E < L$.
  \end{enumerate}
\end{definition}

\begin{example}
  From the last two examples, $\Q[i\sqrt{2}]$ is a
  splitting field over $\Q$ for $X^2 + 2$, and $\Q[\sqrt[3]{2}, i\sqrt{3}]$
  is a splitting field for $X^3 - 2$ over $\Q$.
\end{example}

\begin{theorem}
  Let $K$ be a field and $f \in K[X]$ with
  $\partial f = n$. Then there exists a splitting field
  $L$ for $f$ over $K$ and $[L : K] \le n!$.
\end{theorem}

\begin{proof}
The proof is essentially the process we perform in the
following example. At each step, construct an extension
in which we can split off a linear factor from $f$.
For more details, see Howie.
\end{proof}

\begin{example}
  Let us find a splitting field for
  \[f = X^5 + X^4 - X^3 - 3X^2 - 3X + 3\]
  over $\Q$. Note that $\partial f = n$.
  Stare hard enough and we can see that
  \[
    f = (X^3 - 3)(X^2 + X - 1),
  \]
  where the first factor is irreducible by Eisenstein's criterion and
  the second factor is irreducible by checking the
  discriminant. Now add a root, say $\alpha = \sqrt[3]{3}$,
  and let $E_1 = \Q(\alpha)$. Then
  \[
    f = (X - \alpha) (X^2 + \alpha X + \alpha^2)(X^2 + X - 1).
  \]
  Note that $[E_1 : K] \le n = \partial f$.
  Now let $E_2 = E_1(\alpha e^{2\pi i / 3})$, so that
  \[
    f = (X - \alpha) (X - \alpha e^{2\pi i / 3})
    (X - \alpha e^{-2\pi i / 3})(X^2 + X - 1).
  \]
  Note that $[E_2 : \Q] \le n(n - 1)$. Next
  $E_3 = E_2(\alpha e^{- 2\pi i / 3})$ with
  \[
    f = (X - \alpha) (X - \alpha e^{2\pi i / 3})
    (X - \alpha e^{-2\pi i / 3})(X^2 + X - 1).
  \]
  Note that $[E_3 : K] \le n(n - 1)(n - 2)$. Now let
  \[
    \gamma = \frac{-1 + \sqrt{5}}{2}, \quad \delta = \frac{-1 - \sqrt{5}}{2}.
  \]
  Let $E_4 = E_3(\gamma)$,
  \[
    f = (X - \alpha) (X - \alpha e^{2\pi i / 3})
    (X - \alpha e^{-2\pi i / 3})(X - \gamma)(X - \delta).
  \]
  Finally $E_5 = E_4(\delta)$ is the splitting field
  for $f$ over $\Q$. Note that we did much better than
  $n!$ here, since
  \[
    [E_1 : \Q] = 3, \quad [E_2 : E_1] = 2, \quad
    [E_3 : E_2] = 1, \quad [E_4 : E_3] = 2, \quad
    [E_5 : E_4] = 1,
  \]
  so $[E_5 : \Q] = 12 \le 120$.
\end{example}

\begin{remark}
  Splitting fields are unique (up to isomorphism).
\end{remark}

\begin{theorem}
  Let $L$ and $L'$ be splitting fields of $f$ over $K$.
  Then there exists an isomorphism
  $\varphi : L \to L'$ fixing $K$.
\end{theorem}

\begin{proof}[Proof sketch]
  Induct on the number of roots of $f$ that are not
  in $K$. The induction step uses Theorem \ref{thm:isomorphism-fix}
  from last
  class giving an isomorphism $K[\alpha] \to K[\alpha']$
  for $\alpha, \alpha'$ roots of an irreducible
  polynomial.
\end{proof}

\begin{example}
  Let us find the splitting field of $f = X^4 - 2$ over
  $\Q$ and its degree. Note that $X^4 - 2$ is
  irreducible over $\Q$ by Eisenstein's criterion.
  Note that
  \[
    X^4 - 2 = (X - \alpha)(X + \alpha)(X - i \alpha)(X + i\alpha)
  \]
  where $\alpha = \sqrt[4]{2}$. So the
  splitting field is $\Q(\sqrt[4]{2}, i)$. For the
  degree, note that $[\Q(\sqrt[4]{2}) : \Q] = 4$ since
  the minimal polynomial of $\sqrt[4]{2}$ is $X^4 - 2$.
  A basis for this extension is $\{1, \sqrt[4]{2}, \sqrt[4]{2}^2, \sqrt[4]{2}^3\}$.
  Since $i \notin \Q(\sqrt[4]{2})$, we have
  $[\Q(\sqrt[4]{2}, i) : \Q(\sqrt[4]{2})] = 2$ since
  the minimal polynomial of $i$ over $\Q(\sqrt[4]{2})$
  is $X^2 + 1$. Thus we see that the degree of the
  splitting field
  is $[\Q(\sqrt[4]{2}, i) : \Q] = 8$.
\end{example}

\begin{example}
  Let us look at monic quadratic polynomials
  over $\Z_3 = \{-1, 0, 1\}$.\footnote{Note that as opposite to $\Q$, this field has finite characteristic.}
  These are
  \[
  \begin{array}{ccccc}
    X^2 & & X^2 + 1 & & X^2 - 1 \\
    X^2 + X & & X^2 + X + 1 & & X^2 + X - 1 \\
    X^2 - X & & X^2 - X + 1 & & X^2 - X - 1.
  \end{array}
  \]
  We have $0$ is a root of the polynomials in the
  first column, $1$ is a root of $X^2 - 1$ and
  $X^2 + X + 1$, and $-1$ is a root of $X^2 - X + 1$.
  So the irreducible polynomials over $\Z_3$ are
  \[
    X^2 + 1, \quad X^2 + X - 1, \quad X^2 - X - 1.
  \]
  Let $L = \Z_3[X] / \langle X^2 + 1 \rangle$. Observe
  that $\alpha = X + \langle X^2 + 1 \rangle$ satisfies
  \[
    \alpha^2 = X^2 + \langle X^2 + 1 \rangle
    = -1 + \langle X^2 + 1 \rangle.
  \]
  Hence $L$ is a splitting field for $X^2 + 1$ since
  $(X - \alpha)(X + \alpha) = X^2 + 1$.
  Similarly,
  $\Z_3[X] / \langle X^2 + X - 1 \rangle$ is a
  splitting field for $X^2 + X - 1$ and
  $\Z_3[X] / \langle X^2 - X - 1 \rangle$ is a
  splitting field for $X^2 - X - 1$. Note that each of
  these fields have $9 = 3^2$ elements since they are
  degree $2$ extensions of $\Z_3$.
\end{example}

\begin{remark}
  In $L$, we had $\alpha \in L$ such that
  $\alpha^2 = - 1$ and addition is performed modulo $3$.
  Now observe
  \[
    (\alpha + 1)^2 + (\alpha + 1) - 1
    = (\alpha^2 - \alpha + 1) + (\alpha + 1) - 1
    = \alpha^2 - \alpha + \alpha + 1 + 1 - 1
    = 0
  \]
  since $\alpha^2 = -1$. So $\alpha + 1$ is a root of
  $X^2 + X - 1$ in $L$. By a similar computation, we
  see that $- \alpha + 1$ is a root of $X^2 + X - 1$,
  so $L$ is also a splitting field for $X^2 + X - 1$.
  Additionally, $\alpha - 1$ and $-\alpha - 1$ are
  roots of $X^2 - X - 1$, so $L$ is also a splitting
  field for $X^2 - X - 1$. So by uniqueness of splitting
  fields,
  \[
    \Z_3[X] / \langle X^2 + 1 \rangle
    \cong \Z_3[X] / \langle X^2 + X - 1 \rangle
    \cong \Z_3[X] / \langle X^2 - X - 1 \rangle.
  \]
\end{remark}

\begin{exercise}
  Find explicit isomorphisms between these
  fields.
\end{exercise}

\section{Finite Fields}
\begin{definition}
Let $f = a_0 + a_1 X + \dots + a_n X^n \in K[X]$. Then
the \emph{formal derivative} of $f$ is
\[
  Df = a_1 + 2a_2 X + \dots + n a_n X^{n-1}.
\]
\end{definition}

\begin{exercise}
  The usual formulas for derivatives
  \[
    D(kf) = k Df, \quad D(f + g) = Df + Dg, \quad
    D(fg) = (Df)g + f(Dg)
  \]
  all still hold for $f, g \in K[X]$ and $k \in K$.
\end{exercise}

  \chapter{Feb.~5 --- Finite Fields}

\section{Last Time}

\begin{example}
  The splitting field of $X^4 - 2$ over $\Q$ is
  $\Q(i, \sqrt[4]{2]})$ since
  \[
    X^4 - 2 = (X - \sqrt[4]{2})(X + \sqrt[4]{2})(X - i\sqrt[4]{2})(X + i\sqrt[4]{2}).
  \]
\end{example}

\begin{example}
  The splitting field of $Y^2 + 1$ over $\Z_3$ is
  $\Z_3[X] / \langle X^2 + 1 \rangle$. If
  $\alpha = X + \langle X^2 + 1 \rangle$, then
  \[
    Y^2 + 1 = (Y - \alpha)(Y + \alpha).
  \]
  Also the degree of this extension is
  $[Z_3[X] / \langle X^2 + 1 \rangle : \Z_3] = 2$, and
  a basis for the extension is $\{1, X\}$.
\end{example}

\section{Finite Fields}
\begin{lemma}
  \label{lem:distinct-roots}
  Let $f \in K[X]$, $K$ a field, and $L$ be a splitting
  field for $f$ over $K$. Then the roots of $f$ are
  distinct if and only if $f$ and $Df$ have no
  nonconstant common factor.
\end{lemma}

\begin{proof}
  $(\Leftarrow)$ We show the contrapositive. Suppose
  $f$ has a repeated root $\alpha$ in $L$. Then
  \[
    f = (X - \alpha)^r g
  \]
  for some $r \ge 2$. Then
  \[
    Df = (X - \alpha)^r Dg + r(X - \alpha)^{r - 1} g,
  \]
  so $Df$ and $f$ both have $X - \alpha$ as a factor.

  $(\Rightarrow)$ Suppose the roots of $f$ are
  all distinct. Then for each root $\alpha$ of $f$
  in $L$, we have
  \[
    f = (X - \alpha)g,
  \]
  where $g(\alpha) \ne 0$. Then
  \[
    Df = (X - \alpha)Dg + g,
  \]
  so that
  \[
    (Df)(\alpha) = g(\alpha) \ne 0,
  \]
  i.e. $X - \alpha {\nmid} Df$. This holds for factor
  of $f$ in $L[X]$, so $f$ and $Df$ have no common
  proper factors.
\end{proof}

\begin{theorem}
  Finite fields exist and are unique up to isomorphism.
  In particular,
  \begin{enumerate}
      \item Let $K$ be a finite field. Then $|K| = p^n$ for
    some prime $p$ and integer $n \ge 1$. Every element
    of $K$ is a root of $X^{p^n} - X$ and $K$ is a
    splitting field of $X^{p^n} - X$ over $\Z_p$.
    \item Let $p$ be a prime and $n \in \Z$, $n \ge 1$.
      Then there exists a unique field of
      order $p^n$ up to isomorphism.
  \end{enumerate}
\end{theorem}

\begin{proof}
  (1) Let $\Char K = p$. Then $K$ is a finite extension
  of $\Z_p$. Let $n = [K : \Z_p]$. If
  $\{\delta_1, \dots, \delta_n\}$ is a basis for $K$
  over $\Z_p$, then every element in $K$ can be
  uniquely written as
  \[
    a_1 \delta_1 + \dots + a_n \delta_n
  \]
  for some $a_i \in \Z_p$. There are $p^n$ such elements,
  so $|K| = p^n$. Then $|K^*| = p^n - 1$.\footnote{Recall that $K^*$ is the set of nonzero elements of $K$, which forms a group under multiplication. We also call $K^*$ the group of units of $K$.}
  For
  any $\alpha \in K^*$, the order of $\alpha$ divides
  $p^n - 1$. So $\alpha^{p^n - 1} = 1$, and hence
  $\alpha^{p^n} - \alpha = 0$. We also have
  $0^{p^n} - 0 = 0$ so every element in $K$ is a
  root of $X^{p^n} - X$. Hence $X^{p^n} - X$ splits
  completely over $K$. Since $X - \alpha$ is a factor
  of $X^{p^n} - X$ for each of the $p^n$ elements of $K$,
  $X^{p^n} - X$ does not split over any proper
  subfield of $K$. Thus we conclude that $K$ is a
  splitting field of $X^{p^n} - X$ over $\Z_p$.

  (2) Given a prime $p$ and an integer $n \ge 1$, let
  $L$ be the splitting field of $X^{p^n} - X$ over
  $\Z_p$. Note that
  \[
    Df = p^n X^{p^n - 1} - 1 = -1
  \]
  since $\Char \Z_p = p$. Then $Df$ and $f$ have no
  nonconstant common factors, so by Lemma
  \ref{lem:distinct-roots}, we see that $X^{p^n} - X$ has
  $p^n$ distinct roots in $L$. Let $K$ be the set of
  $p^n$ distinct roots, and we claim that $K$ is a
  subfield of $L$. To check this, let $a, b \in K$.
  Then by an extension of Theorem \ref{thm:freshman-exponentiation},
  \[
    (a - b)^{p^n} = a^{p^n} - b^{p^n} = a - b
  \]
  in $\Z_p$, $a - b \in K$. Also
  \[
    (ab^{-1})^{p^n} = a^{p^n} (b^{p^n})^{-1} = ab^{-1},
  \]
  so $ab^{-1} \in K$. Hence $K$ is a field of order
  $p^n$. In fact, $K = L$ since $K$ contains all the
  roots of $X^{p^n} - X$ and no proper subfield does.
  By uniqueness of splitting fields, $K$ is unique up
  to isomorphism.
\end{proof}

\begin{definition}
  We call the field of order $p^n$ the \emph{Galois field}
  of order $p^n$, denoted $\text{GF}(p^n)$.
\end{definition}

\begin{example}
  We have
  $\GF(3^2) = \Z_3[X] / \langle X^2 + 1 \rangle \cong \Z_3[X] / \langle X^2 + X - 1 \rangle \cong \Z_3[X] / \langle X^2 - X - 1 \rangle$.
\end{example}

\begin{remark}
  Recall that for a finite group $G$ and $a \in G$, the
  \emph{order} of $a$ is
  \[
    \ord(a) = \min\{k \in \N : a^k = 1\}.
  \]
  The \emph{exponent} of $G$ is
  \[
    \exp(G) = \min\{k \in \N : a^k = 1 \text{ for all } a \in G\}.
  \]
  Also recall that $\ord(a)$ divides $|G|$ for all $a \in G$,
  and thus $\exp(G)$ divides $|G|$.
\end{remark}

\begin{exercise}
  Show that $\exp(G) = \lcm\{\ord(a) : a \in G\}$.
\end{exercise}

\begin{example}
  For $S_3 = \{\id, (12), (23), (13), (123), (132)\}$,
  the order of the transpositions is $2$ and the order
  of $3$-cycles is $3$. So we see that $\exp(S_3) = 6$.
\end{example}

\begin{prop}
  If $G$ is a finite abelian group, then there exists
  $a \in G$ such that $\ord(a) = \exp(G)$.
\end{prop}

\begin{proof}
  Suppose that
  \[
    \exp(G) = p_1^{\alpha_1} p_2^{\alpha_2} \dots p_k^{\alpha_k},
  \]
  where the $p_i$ are distinct primes and $\alpha_i \ge 1$
  for all $i$. Since
  \[
    \exp(G) = \lcm\{\ord(a) : a \in G\},
  \]
  there exists $h_1 \in G$ such that
  $p_1^{\alpha_1} | \ord(h_1)$. So
  $\ord(h_1) = p_1^{\alpha_1} q_1$ where
  $q_1 | p_2^{\alpha_2} \dots p_k^{\alpha_k}$.
  Let $g_1 = h_1^{q_1}$. For each
  $m \ge 1$, we have $g_1^m = h_1^{m q_1}$, and
  \[
    h_1^{mq_1} = 1 \iff p_1^{\alpha_1} q_1 | mq_1
    \iff p_1^{\alpha_1} | m.
  \]
  Hence $\ord(g_1) = p_1^{\alpha_1}$. Similarly
  for $i = 2, \dots, k$, we can find elements $g_i$
  of order $p_i^{\alpha_i}$.  Let
  \[
    a = g_1 g_2 \dots g_k
  \]
  and $n = \ord(a)$. Now check as an exercise
  that $\ord(a) = \exp(G)$. This relies on
  \[
    a^n = g_1^n g_2^n \dots g_k^n = 1,
  \]
  which uses the assumption that $G$ is abelian.
\end{proof}

\begin{remark}
  The previous example shows that the abelian condition
  in this theorem is necessary.
\end{remark}

\begin{corollary}
  \label{cor:finite-abelian-cyclic}
  If $G$ is a finite abelian group with $\exp(G) = |G|$,
  then $G$ is cyclic.
\end{corollary}

\begin{theorem}
  The group of units
  $\GF(p^n)^*$
  of a Galois field is cyclic.
\end{theorem}

\begin{proof}
  Let $e = \exp(\GF(p^n)^*)$. Then $a^e = 1$ for
  all $a \in \GF(p^n)^*$, so every element
  $a \in \GF(p^n)^*$ is a root of $X^e - 1$. Since
  $X^e - 1$ has at most $e$ roots, we see that
  $|\GF(p^n)^*| \le e$. But $e \le |\GF(p^n)^*|$ since
  $\exp(\GF(p^n)^*)$ divides $|\GF(p^n)^*|$. Hence
  $|\GF(p^n)^*| = e$, so by Corollary \ref{cor:finite-abelian-cyclic},
  $\GF(p^n)^*$ is cyclic.
\end{proof}

\section{Automorphisms of Fields}

\begin{example}
  The complex conjugation $f : \C \to \C$ given by
  $f(a + bi) = a - bi$ is an automorphism of $\C$.
  Observe that $f(c) = c$ if and only if $c \in \R$.
\end{example}

\begin{theorem}
  Let $K$ be a field. The set $\Aut K$ of automorphisms
  of $K$ forms a group under composition.
\end{theorem}

\begin{proof}
  First observe that composition is associative. The
  identity element in $\Aut K$ is the identity map
  $\id_K$. For inverses, let $\alpha \in \Aut K$. Since
  $\alpha$ is a bijection, there exists an inverse map
  $\alpha^{-1} : K \to K$, where $\alpha^{-1}(x)$
  is the unique element $s$ such that $\alpha(s) = x$. Now
  we check that $\alpha^{-1}$ is also a homomorphism.
  For this, let $x, y \in K$ and suppose that
  $\alpha^{-1}(x) = s$ and $\alpha^{-1}(y) = t$. Then
  $\alpha(s) = x$ and $\alpha(t) = y$, so
  \[
    \alpha(s + t) = \alpha(s) + \alpha(t) = x + y
  \]
  since $\alpha$ is a homomorphism. Then we see that
  \[
    \alpha^{-1}(x + y) = s + t = \alpha^{-1}(x) + \alpha^{-1}(y).
  \]
  Similarly, $\alpha(st) = xy$, so
  \[
    \alpha^{-1}(xy) = st = \alpha^{-1}(x) \alpha^{-1}(y). \]
  Hence $\alpha^{-1} \in \Aut K$ and
  $\alpha \circ \alpha^{-1} = \alpha^{-1} \circ \alpha = \id_K$,
  so $\Aut K$ is indeed a group.
\end{proof}

\begin{definition}
  We call $\Aut K$ the \emph{group of automorphisms}
  of $K$.
\end{definition}

\begin{definition}
  Let $L$ be a field extension of $K$. A
  \emph{$K$-automorphism} is an automorphism
  $\alpha : L \to L$
  such that $\alpha(x) = x$ for all $x \in K$. The
  \emph{Galois group} of $L$ over $K$, denoted
  $\Gal(L : K)$, is the set of $K$-automorphisms of $L$.
  The \emph{Galois group} $\Gal(f)$ of a polynomial
  $f \in K[X]$ is $\Gal(L : K)$ where $L$ is a splitting
  field of $f$ over $K$.
\end{definition}

\begin{theorem}
  The Galois group $\Gal(L : K)$ is a subgroup of
  $\Aut L$.
\end{theorem}

\begin{proof}
  Clearly $\id_L \in \Gal(L : K)$ since it fixes all
  elements of $L$. Now let $\alpha, \beta \in \Gal(L : K)$.
  Then we have $\alpha(x) = x$ and $\beta(x) = x$ for
  all $x \in K$. Then $\beta^{-1}(x) = x$, which gives
  \[
    \alpha \beta^{-1}(x) = \alpha(x) = x,
  \]
  so $\alpha \beta^{-1} \in \Gal(L : K)$. Thus
  $\Gal(L : K)$ is a subgroup of $\Aut L$.
\end{proof}

\begin{remark}
  The big idea here is that there is a correspondence
  between subfields $E$ with $K \subseteq E \subseteq L$
  and subgroups $H$ of $\Gal(L : K)$.
\end{remark}

\begin{remark}
  From a past homework, we identified the subfields
  of $\Q(\sqrt{3}, \sqrt{5})$ as:
  \[
  \begin{tikzcd}
    & \Q(\sqrt{3}, \sqrt{5}) \\
    \Q(\sqrt{3}) \urar & \Q(\sqrt{5}) \uar & \Q(\sqrt{15}) \ular \\
    & \Q \ular \uar \urar
  \end{tikzcd}
  \]
  Compare the subgroups of $\Gal(\Q(\sqrt{3}, \sqrt{5}) : \Q)$
  to subfields of $\Q(\sqrt{3}, \sqrt{5})$ containing
  $\Q$.
\end{remark}

  \chapter{Feb.~7 --- The Galois Correspondence}

\section{Automorphisms of Fields}

\begin{example}
  The complex conjugation $\beta : \C \to \C$ given
  by $\beta(a + bi) = a - bi$ is a nontrivial element
  of the Galois group of $\C : \R$. In fact,
  $\Gal(\C : \R) = \{\id, \beta\}$. Note that
  $\beta$ fixes $\R$, $\id$ fixes $\C$, and
  \[
  \begin{tikzcd}
    \C \\
    \R \uar
  \end{tikzcd}
  \]
\end{example}

\section{The Galois Correspondence}

\begin{definition}
  Define
  \begin{align*}
    \Gamma(E) &= \{
      \alpha \in \Aut L : \alpha(z) = z \text{ for all } z \in E
    \}, \\
    \Phi(H) &= \{
    x \in L : \alpha(x) = x \text{ for all } \alpha \in H
    \},
  \end{align*}
  where $E$ is a subfield of $L$ and $H$ is a subgroup
  of $\Gal(L : K)$. This is called the
  \emph{Galois correspondence}.
\end{definition}

\begin{example}
  In the previous example of $\C : \R$, we have
  $\Gamma(\C) = \{\id\}$ and $\Gamma(\R) = \{\id, \beta\}$.
  We also have $\Phi(\{\id, \beta\}) = \R$ and
  $\Phi(\{\id\}) = \C$.
\end{example}

\begin{remark}
  The goal is to determine: When are $\Gamma$ and
  $\Phi$ inverses of one another?
\end{remark}

\begin{theorem}
  We have the following:
  \begin{enumerate}
    \item For every subfield $E$ of $L$ containing $K$,
      $\Gamma(E)$ is a subgroup of $\Gal(L : K)$.
    \item Conversely, for every subgroup $H$
      of $\Gal(L : K)$, $\Phi(H)$ is a subfield of $L$
      containing $K$.
  \end{enumerate}
\end{theorem}

\begin{proof}
  See Howie.
\end{proof}

\begin{theorem}
  Let $z \in L \setminus K$. If $z$ is a root of
  $f \in K[X]$ and $\alpha \in \Gal(L : K)$, then
  $\alpha(z)$ is also a root of $f$.
\end{theorem}

\begin{proof}
  Let $f = a_0 + a_1X + \cdots + a_nX^n$, where
  $a_i \in K$. Then since $\alpha$ fixes each
  $a_i \in K$, we have
  \begin{align*}
    f(\alpha(z))
    = a_0 + a_1\alpha(z) + \cdots + a_n(\alpha(z))^n
    &= \alpha(a_0) + \alpha(a_1)\alpha(z) + \cdots + \alpha(a_n)(\alpha(z))^n \\
    &= \alpha(a_0 + a_1z + \cdots + a_nz^n)
    = \alpha(0) = 0,
  \end{align*}
  which completes the proof.
\end{proof}

\begin{example}
  Recall this example from homework:
  \[
  \begin{tikzcd}
    & L = \Q(\sqrt{3}, \sqrt{5}) \\
    \Q(\sqrt{3}) \urar & \Q(\sqrt{5}) \uar & \Q(\sqrt{15}) \ular \\
    & K = \Q \ular \uar \urar
  \end{tikzcd}
  \]
  A basis for $L$ over $K$ is $\{1, \sqrt{3}, \sqrt{5}, \sqrt{15}\}$.
  Since $\sqrt{3}$ is a root of $X^2 - 3$,
  by the previous theorem, any element in $\Gal(L : K)$
  must send $\sqrt{3} \mapsto \pm \sqrt{3}$. Similarly,
  any element must send
  $\sqrt{5} \mapsto \pm \sqrt{5}$. So
  the $\Q$-isomorphisms of $\Q(\sqrt{3}, \sqrt{5})$
  are
  \begin{align*}
    \alpha(a + b\sqrt{3} + c\sqrt{5} + d\sqrt{15})
    &= a - b\sqrt{3} + c\sqrt{5} - d\sqrt{15}, \\
    \beta(a + b\sqrt{3} + c\sqrt{5} + d\sqrt{15})
    &= a + b\sqrt{3} - c\sqrt{5} - d\sqrt{15}, \\
    \gamma(a + b\sqrt{3} + c\sqrt{5} + d\sqrt{15})
    &= a - b\sqrt{3} - c\sqrt{5} + d\sqrt{15}, \\
    \id(a + b\sqrt{3} + c\sqrt{5} + d\sqrt{15})
    &= a + b\sqrt{3} + c\sqrt{5} + d\sqrt{15}.
  \end{align*}
  We can write the multiplication table for this
  group as:
  \begin{center}
    \begin{tabular}{c|cccc}
      $\times$ & $\id$ & $\alpha$ & $\beta$ & $\gamma$ \\
      \hline
      $\id$ & $\id$ & $\alpha$ & $\beta$ & $\gamma$ \\
      $\alpha$ & $\alpha$ & $\id$ & $\gamma$ & $\beta$ \\
      $\beta$ & $\beta$ & $\gamma$ & $\id$ & $\alpha$ \\
      $\gamma$ & $\gamma$ & $\beta$ & $\alpha$ & $\id$
    \end{tabular}
  \end{center}
  The proper subgroups are
  $H_1 = \{\id, \alpha\}$, $H_2 = \{\id, \beta\}$, and
  $H_3 = \{\id, \gamma\}$. Also
  $\{id\}$ and $G = \{\id, \alpha, \beta, \gamma\}$
  are subgroups. Then
  \begin{gather*}
    \Phi(H_1) = \Q(\sqrt{5}), \quad
    \Phi(H_2) = \Q(\sqrt{3}), \quad
    \Phi(H_3) = \Q(\sqrt{15}), \\
    \Phi(\{\id\}) = \Q(\sqrt{3}, \sqrt{5}), \quad
    \Phi(G) = \Q.
  \end{gather*}
  Under $\Phi$, this gives the diagram:
  \[
  \begin{tikzcd}
    & G \\
    H_1 \urar & H_2 \uar & H_3 \ular \\
    & \{\id\} \ular \uar \urar
  \end{tikzcd} \quad
  \longrightarrow \quad
  \begin{tikzcd}
    & \Phi(G) = \Q \drar \dar \dlar \\
    \Q(\sqrt{3}) \drar & \Q(\sqrt{5}) \dar & \Q(\sqrt{15}) \dlar \\
    & \Phi(\{\id\}) = \Q(\sqrt{3}, \sqrt{5})
  \end{tikzcd}
  \]
  Also note that
  $\Gamma(\Q(\sqrt{3})) = \{\id, \alpha\}$ since
  \[
    \alpha(a + b\sqrt{3} + c\sqrt{5} + d\sqrt{15})
    = a - b\sqrt{3} + c\sqrt{5} - d\sqrt{15}.
  \]
\end{example}

\begin{exercise}
  Show that $\Gamma$ is the inverse of $\Phi$ in the
  previous example.
\end{exercise}

\begin{theorem}
  Let $L : K$ be a field extension. Then
  \begin{enumerate}
    \item If $E_1, E_2$ are two subfields of $L$
      containing $K$, then
      $E_1 \subseteq E_2$ implies
      $\Gamma(E_1) \supseteq \Gamma(E_2)$.
    \item If $H_1, H_2$ are subgroups of $\Gal(L : K)$,
      then $H_1 \subseteq H_2$ implies
      $\Phi(H_1) \supseteq \Phi(H_2)$.
  \end{enumerate}
\end{theorem}

\begin{proof}
  (1) Suppose $E_1 \subseteq E_2$ and $\alpha \in \Gamma(E_2)$.
  Then $\alpha$ fixes every element in $E_2$, so since
  $E_1 \subseteq E_2$, $\alpha$ also fixes every element
  in $E_1$. Hence $\alpha \in \Gamma(E_1)$ by
  definition.

  (2) Suppose $H_1 \subseteq H_2$ and let $z \in \Phi(H_2)$.
  Then $\alpha(z) = z$ for every $\alpha \in H_2$, and
  since $H_1 \subseteq H_2$, $\alpha(z) = z$ for every
  $\alpha \in H_1$ as well. Hence $z \in \Phi(H_1)$ by
  definition.
\end{proof}

\begin{remark}
  Note that $\Gamma$ and $\Phi$ are not always
  inverses of one another.
\end{remark}

\begin{example}
  Consider the extension $\Q(\sqrt[3]{2}) : \Q$.
  If $\alpha \in \Gal(\Q(\sqrt[3]{2}) : \Q)$, then
  \[
    \alpha(\sqrt[3]{2})^3 = \alpha(2) = 2.
  \]
  Since there is only one cube root of $2$ in this
  field, we must have $\alpha(\sqrt[3]{2}) = \sqrt[3]{2}$.
  So $\Gal(\Q(\sqrt[3]{2}) : \Q) = \{\id\}$. So
  $\Gamma$ cannot be the inverse of $\Phi$ here since
  there are two subfields, namely $\Q(\sqrt[3]{2})$
  and $\Q$. In particular,
  \[
    \Gamma(\Q(\sqrt[3]{2})) = \Gamma(\Q) = \{\id\}
    \quad \text{and} \quad
    \Phi(\{\id\}) = \Q(\sqrt[3]{2}).
  \]
\end{example}

\begin{theorem}
  For any subfield $E$ of $L$ and subgroup $H$ of $\Gal(L : K)$, we have
  \begin{enumerate}
    \item $E \subseteq \Phi(\Gamma(E))$
    \item and $H \subseteq \Gamma(\Phi(H))$.
  \end{enumerate}
\end{theorem}

\begin{proof}
  (1) Let $z \in E$. Then $\Gamma(E)$ is the set of
  all automorphisms fixing every element of $E$, and so
  $z$ is fixed by every element of $\Gamma(E)$.
  Hence $z \in \Phi(\Gamma(E))$.

  (2) Let $\alpha \in H$. Then $\Phi(H)$ is the set
  of elements of $L$ fixed by every element of $H$, and
  so $\alpha$ fixes every element of $\Phi(H)$.
  Hence $\alpha \in \Gamma(\Phi(H))$.
\end{proof}

\begin{remark}
  Now the goal will be to find sufficient conditions
  for $\Gamma$ and $\Phi$ to be inverses of one another.
\end{remark}

\section{Normal Extensions}

\begin{definition}
  A field extension $L : K$ is \emph{normal} if every
  irreducible polynomial in $K[X]$ having at least
  one root in $L$ splits completely over $L$.
\end{definition}

\begin{example}
  An nonexample is $\Q(\sqrt[3]{2}) : \Q$. This is not
  a normal extension since $X^3 - 2$ is irreducible
  and has a root in $\Q(\sqrt[3]{2})$, but does not
  split completely over $\Q(\sqrt[3]{2})$.
\end{example}

\begin{remark}
  Is $\Q(\sqrt{2}) : \Q$ normal?
\end{remark}

\begin{theorem}
  A finite extension $L : K$ is normal if and only if
  it is a splitting field for some polynomial in $K[X]$.
\end{theorem}

\begin{proof}
  $(\Rightarrow)$ Let $L$ be a finite normal extension
  and $\{z_1, \dots, z_n\}$ be a basis for $L : K$.
  let $m_i$ be the minimum polynomial for $z_i$, and
  let
  \[
    m = m_1 m_2 \dots m_n.
  \]
  Each $m_i$ has at least one root $z_i$ in $L$, hence
  $m$ splits completely over $L$ since $L$ is normal.
  Since $L$ is generated by $z_1, \dots, z_n$, it is
  not possible for $m$ to split over a proper subfield
  of $L$, hence $L$ is a splitting field for $m$ over $K$.

  $(\Leftarrow)$ See Howie. Relies on the isomorphism
  $K(\alpha) \to K(\beta)$ for $\alpha, \beta$ roots
  of an irreducible polynomial $f$. We also need
  properties of degrees of field extensions.
\end{proof}

\begin{corollary}
  Let $L$ be a normal extension of $K$ and $E$ a
  subfield of $L$ containing $K$. Then every injective
  $K$-homomorphism $\varphi : E \to L$ can be extended to
  a $K$-automorphism $\varphi^*$ of $L$.
\[
  \begin{tikzcd}
    E \dar[hook, swap, "i"] \rar{\varphi} & L \\
    L \urar[dashed, "\varphi^*"'] &
  \end{tikzcd}
\]
\end{corollary}

\begin{proof}
  By the theorem, there exists $f \in K[X]$ such that
  $L$ is a splitting field for $f$ over $K$. But
  $L$ is also a splitting field for $f$ over $E$ and
  $\varphi(E)$. From here, a slight generalization of the
  proof of uniqueness of splitting fields gives
  the desired $K$-automorphism of $L$ extending $\varphi$.
\end{proof}

\begin{example}
  Let $L = \Q(\sqrt{3}, \sqrt{5})$, $K = \Q$, and
  $E = \Q(\sqrt{3})$. Define $\varphi : E \to L$
  by
  \[
    \varphi(a + b\sqrt{3}) = a - b\sqrt{3},
  \]
  which is an injective $K$-homomorphism. We have the
  following diagram:
  \[
  \begin{tikzcd}
    \Q(\sqrt{3}) \dar[hook, swap, "i"] \rar{\varphi} & \Q(\sqrt{3}, \sqrt{5}) \\
    \Q(\sqrt{3}, \sqrt{5}) \urar[dashed, "\varphi^*"'] &
  \end{tikzcd}
  \]
  Then we can define
  \[
    \varphi^*(a + b\sqrt{3} + c\sqrt{5} + d\sqrt{15})
    = a - b\sqrt{3} + c\sqrt{5} - d\sqrt{15}
  \]
  as an extension of $\varphi$. Note that we could have
  also defined
  \[
    \varphi^*(a + b\sqrt{3} + c\sqrt{5} + d\sqrt{15})
    = a - b\sqrt{3} - c\sqrt{5} + d\sqrt{15}.
  \]
\end{example}

\begin{remark}
  From the previous example we see that
  $\varphi^*$ is not unique.
\end{remark}

  \chapter{Feb.~12 --- Normal Closures}

\section{Normal Closures}

Recall this theorem from last time:
\begin{quote}
  \textbf{Theorem \ref{thm:normal-splitting}.}
  A finite extension $L : K$ is normal if and only
  if it is a splitting field for some polynomial
  in $K[X]$.
\end{quote}
A natural question to ask is: Can we always extend
a finite extension to make it normal?

\begin{definition}
  Let $L : K$ be a finite extension. A field $N$
  containing $L$ is a \emph{normal closure} of $L : K$ if
  \begin{enumerate}
    \item $N$ is a normal extension of $K$,
    \item and if $E$ is a proper subfield of $N$
      containing $L$, then $E$ is not a normal
      extension of $K$.
  \end{enumerate}
\end{definition}

\begin{theorem}
  Let $L : K$ be a finite extension. Then
  \begin{enumerate}
    \item there exists a normal closure $N$ of $L$
      over $K$,
    \item and $N$ is unique up to isomorphism.
  \end{enumerate}
\end{theorem}

\begin{proof}
  Let $\{z_1, \dots, z_n\}$ be a basis for $L : K$. Since
  $L : K$ is finite, each $z_i$ is algebraic over $K$,
  with say minimal polynomial $m_i \in K[X]$. Let
  \[
    m = m_1 \dots m_n,
  \]
  and let $N$ be the splitting field of $m$ over $L$.
  Then $N$ is also a splitting field of $m$ over $K$,
  since $L$ is generated over $K$ by some of
  the roots of $m$ in $N$. Hence $N$ is a normal
  extension of $K$ containing $L$.

  To see that $N$ is the smallest such field, suppose
  $E$ is a subfield of $N$ containing $L$, and suppose
  $E$ is normal. For each $m_i$, $E$ contains a root
  $z_i$, so the normality of $E$ implies that $E$ contains
  all the roots of $m$, so $E = N$. For uniqueness,
  see Howie. The proof relies on the uniqueness of
  splitting fields.
\end{proof}

\begin{definition}
  Let $K_1, \dots, K_n$ be subfields of $L$. The
  \emph{join} of $K_1, \dots, K_n$, denoted
  \[
    K_1 \lor K_2 \lor \dots \lor K_n,
  \]
  is the smallest subfield of $L$ containing
  $K_1 \cup K_2 \cup \dots \cup K_n$.
\end{definition}

\begin{remark}
  The smallest subfield of $L$ containing
  $K_1 \cup K_2$ is $K_1 \lor K_2 = K_1(K_2) = K_2(K_1)$,
  similar to how the smallest subfield of $\R$
  containing $\Q \cup \{\sqrt{3}\}$ is $\Q(\sqrt{3})$.
\end{remark}

\begin{example}
  Let $\Q(\sqrt[3]{2}), \Q(e^{2\pi i / 3} \cdot \sqrt[3]{2}) \subseteq \C$.
  Then
  $\Q(\sqrt[3]{2}) \lor \Q(e^{2\pi i / 3} \cdot \sqrt[3]{2}) = \Q(\sqrt[3]{2}, i\sqrt{3})$, since
  \[
    e^{2\pi i / 3} \cdot \sqrt[3]{2} = -\frac{\sqrt[3]{2}}{2} + \frac{i\sqrt{3}}{2} \sqrt[3]{2}.
  \]
\end{example}

\begin{remark}
  In the above example, we have
  $\Q(\sqrt[3]{2}) \cong \Q(e^{2\pi i / 3} \cdot \sqrt[3]{2}) \cong \Q[X] / \langle X^3 - 2 \rangle$.
\end{remark}

\begin{corollary}
  Let $L : K$ be a finite extension, and $N$ the normal
  closure of $L : K$. Then
  \[
    N = L_1 \lor L_2 \lor \dots \lor L_k,
  \]
  where $L_1, L_2, \dots, L_k$ are subfields of $N$
  containing $K$ isomorphic to $L$.
\end{corollary}

\begin{proof}
  As in the previous proof, suppose $\{z_1, \dots, z_n\}$
  is a basis for $L : K$, so $L = K(z_1, \dots, z_n)$,
  and $m_i$ is a minimal polynomial for $z_i$, and $N$
  a splitting field for $m = m_1 \dots m_n$ over $K$.
  Let $z_i'$ be an arbitrary root of $m_i$. Since
  $z_i$ and $z_i'$ are both roots of $m_i$, there exists
  a $K$-isomorphism $\varphi : K(z_i) \to K(z_i')$,
  which by Corollary \ref{thm:extend-automorphism} implies
  there exists a $K$-automorphism
  $\varphi^* : N \to N$. We have that
  \[
    z_i' \in \varphi^*(L) \cong L,
  \]
  so every root of $m_i$ is contained in a subfield
  $L' = \varphi^*(L)$ of $N$ that contains $K$ and is
  isomorphic to
  $L$, since $\varphi^*$ is a $K$-automorphism. Since
  $N$ is generated over $K$ by the roots of $m$, it is
  generated by finitely many subfields containing $K$
  and isomorphic to $L$.
\end{proof}

\begin{example}
  Find the normal closure of $\Q(\sqrt[3]{2})$ over $\Q$.
  Following the proof of the theorem,
  \[
    \{1, \sqrt[3]{2}, \sqrt[3]{2}^2\}
  \]
  is a basis of $\Q(\sqrt[3]{2}) : \Q$. The minimal
  polynomials of $1, \sqrt[3]{2}, \sqrt[3]{2}^2$ are
  $X - 1, X^3 - 2, X^3 - 4$, respectively. The
  splitting field of
  \[
    (X - 1)(X^3 - 2)(X^3 - 4)
  \]
  over $\Q$ is $\Q(\sqrt[3]{2}, i\sqrt{3})$, since
  \[
    X^3 - 2 = (X - \sqrt[3]{2})(X - e^{2\pi i / 3} \sqrt[3]{2})(X - e^{-2\pi i / 3} \sqrt[3]{2})
  \]
  and
  \[
    X^3 - 4 = (X - \sqrt[3]{2}^2)(X - e^{2\pi i / 3} \sqrt[3]{2}^2)(X - e^{-2\pi i / 3} \sqrt[3]{2}^2).
  \]
  So $\Q(\sqrt[3]{2}, i\sqrt{3}) = L_1 \lor L_2 \lor L_3$,
  where $L_1 = \Q(\sqrt[3]{2})$, $L_2 = \Q(e^{2\pi i / 3} \sqrt[3]{2})$,
  and $L_3 = \Q(e^{-2\pi i / 3} \sqrt[3]{2})$, and
  \[
    L_1 \cong L_2 \cong L_3 \cong \Q[X] / \langle X^3 - 2 \rangle.
  \]
\end{example}

\begin{theorem}
  Let $L : K$ be a finite normal extension and $E$ a
  subfield
  of $L$ containing $K$. Then $E$ is a normal extension
  of $K$ if and only if every $K$-monomorphism of
  $E$ into $L$ is a $K$-automorphism of $E$.
\end{theorem}

\begin{proof}
  $(\Rightarrow)$ Suppose $E : K$ is normal and
  let $\varphi : E \to L$ be a $K$-monomorphism. Now we
  would like to show that $\varphi(E) \subseteq E$. So let
  let $z \in E$ and suppose
  \[
    m = a_0 + a_1 X + \dots + a_n X^n
  \]
  is the minimal polynomial of $z$ over $K$. Then
  \[
    a_0 + a_1z + \dots + a_nz^n = 0,
  \]
  so that
  \[
    a_0 + a_1\varphi(z) + \dots + a_n\varphi(z)^n = 0
  \]
  since $\varphi$ is a homomorphism fixing $K$ pointwise.
  Hence $\varphi(z)$ is also a root of $m$ in $L$.
  Since $E : K$ is normal, the irreducible polynomial
  $m$ splits completely over $E$. Hence
  $\varphi(z) \in E$, so that $\varphi(E) \subseteq E$.
  Then\footnote{We need to make this argument since $E$ may be infinite, so injectivity does not imply bijectivity.}
  \[
    [\varphi(E) : K] = [\varphi(E) : \varphi(K)]
    = [E : K] = [E : \varphi(E)][\varphi(E) : K],
  \]
  so $[E : \varphi(E)] = 1$. Hence $\varphi(E) = E$,
  so $\varphi$ is a $K$-automorphism of $E$.

  $(\Leftarrow)$ Suppose every $K$-monomorphism
  $E \to L$ is a $K$-automorphism of $E$. Let $f$
  be an irreducible polynomial in $K[X]$ having a root
  $z \in E$. We need to show that $f$ splits completely
  over $E$. Since $L$ is normal, $f$ splits completely
  over $L$. Let $z'$ be another root of $f$ in $L$.
  Then there exists a $K$-automorphism
  $K(z) \to K(z')$ which sends $z \mapsto z'$,
  which by Corollary \ref{thm:extend-automorphism}
  extends to a $K$-automorphism $\psi$ of $L$. Let
  $\psi^* = \psi|_E$, i.e. the restriction of $\psi$
  to $E$. By hypothesis, $\psi^*$ is a $K$-automorphism
  of $E$, so
  \[z' = \psi(z) = \psi^*(z) \in E.\]
  That is, $E$ is normal.
\end{proof}

\begin{example}
  Consider $\Q(\sqrt[3]{2}) : \Q$, which is not normal.
  The $\Q$-monomorphism $\varphi : \Q(\sqrt[3]{2}) \to \C$
  given by
  \[
    \varphi(a + b\sqrt[3]{2} + c\sqrt[3]{2}^2) = a + be^{2\pi i / 3} \sqrt[3]{2} + ce^{-2\pi i / 3} \sqrt[3]{2}^2
  \]
  is not an automorphism of $\Q(\sqrt[3]{2})$.
\end{example}

\begin{example}
  Consider $\Q(\sqrt{2}) : \Q$, which is normal. The
  $\Q$-monomorphisms are $\id$ and
  \[\varphi(a + b\sqrt{2}) = a - b\sqrt{2},\]
  which are both $\Q$-automorphisms of $\Q(\sqrt{2})$.
\end{example}

\section{Separable Extensions}

\begin{definition}
  An irreducible polynomial $f \in K[X]$ is
  \emph{separable} over $K$ if it has no repeated roots
  over a splitting field. A polynomial $g \in K[X]$ is
  \emph{separable} over $K$ if its irreducible factors
  are separable over $K$. An algebraic element in $L : K$
  is \emph{separable} over $K$ if its minimal polynomial
  is separable over $K$. An algebraic extension $L : K$
  is \emph{separable} if every $\alpha \in L$ is separable
  over $K$.
\end{definition}

\begin{remark}
  A polynomial like $(X - 2)^2$ actually \emph{is}
  separable
  over $\Q$ since its irreducible factors are
  $X - 2$ and $X - 2$, which are each separable.
\end{remark}

\begin{definition}
  A field $K$ is \emph{perfect} if every polynomial
  in $K[X]$ is separable over $K$.
\end{definition}

\begin{theorem}
  We have the following:
  \begin{enumerate}
    \item Every field of characteristic $0$ is perfect.
    \item Every finite field is perfect.
  \end{enumerate}
\end{theorem}

\begin{proof}
  (1) It suffices to show that if $\Char K = 0$, then
  any irreducible polynomial $f$ is separable. Let
  \[
    f = a_0 + a_1 X + \dots + a_n X^n
  \]
  for $n \ge 1$ and suppose $f$ is not separable. Then
  $f$ and $Df$ have a non-constant common factor $d$.
  Since $f$ is irreducible, $d$ must be a constant
  multiple of $f$, and thus $d$ cannot divide $Df$
  unless
  \[
    Df = a_1 + 2a_2 X + \dots + na_n X^{n-1}
  \]
  is the zero polynomial, by comparing degrees. Then
  \[
    a_1 = 2a_2 = \dots = na_n = 0.
  \]
  Since $\Char K = 0$, this implies
  \[a_1 = a_2 = \dots = a_n = 0,\]
  and so $f = a_0$,
  a constant polynomial.\footnote{Recall that an irreducible polynomial is by definition a non-unit.}
  Contradiction. Hence $f$ is separable.

  (2) The same argument as above implies the only possible
  inseparable irreducible polynomials are of the form\footnote{We can still conclude $ka_k = 0$ implies $a_k = 0$ when $k$ is not a multiple of $p$.}
  \[
    f(X) = b_0 + b_1 X^p + b_2 X^{2p} \dots + b_{m} X^{mp}.
  \]
  Now Theorem 7.24 of Howie implies that if $K$ is finite,
  such a polynomial is reducible. Hence every irreducible
  polynomial is separable, so $K$ is perfect. See Howie
  for details.
\end{proof}

\begin{remark}
  Recall that $\Z_p(X)$ is an example of an infinite
  field with characteristic $p$.
\end{remark}

  \chapter{Feb.~21 --- Galois Extensions}

\section{Example of an Inseparable Extension}

\begin{example}
  The field $K = \Z_p(X)$ is not perfect. Consider
  the polynomial
  \[
    f = Y^p - X \in \Z_p(X)[Y],
  \]
  which is irreducible. Now let $L$ be a splitting
  field of $f$ over $K$ and $\alpha$ a root of $f$,
  i.e. $\alpha^p - X = 0$. Then
  \[
    (Y - \alpha)^p = Y^p - \alpha^p = Y^p - X
  \]
  by freshman exponentiation. In particular,
  $\alpha$ is a repeated root of $f$ in $L$.
\end{example}

\section{Galois Extensions}

\begin{definition}
  A \emph{Galois extension} of $K$ is a finite extension
  that is both normal and separable.
\end{definition}

\begin{remark}
  The main goal here is: For a Galois extension,
  $\Gamma$ and $\Phi$ are inverses of one another.
\end{remark}

\begin{theorem}
  Let $L : K$ be a separable extension of degree
  $n$. Then there are exactly $n$ distinct
  $K$-monomorphisms of $L$ into a normal closure $N$
  of $L$ over $K$.
\end{theorem}

\begin{proof}
  Use strong induction on the degree of $L : K$.
  See Howie for details.
\end{proof}

\begin{corollary}
  \label{cor:galois-size}
  If $L : K$ is Galois, then
  $|\Gal(L : K)| = [L : K]$.
\end{corollary}

\begin{proof}
  If $L : K$ is Galois, then $L : K$ is normal
  and separable. So the previous theorem applies,
  where $L$ is its own normal closure. So we get exactly
  $[L : K]$ distinct $K$-monomorphisms of $L$ into $L$,
  which are precisely the $K$-automorphisms of $L$ and
  thus the elements of the Galois group.
\end{proof}

\begin{example}
  The extension $\Q(\sqrt[3]{2}, i\sqrt{3}) : \Q$ is
  Galois with $[\Q(\sqrt[3]{2}, i\sqrt{3}) : \Q] = 6$.
  We could have
  \[
    \sqrt[3]{2} \mapsto \sqrt[3]{2} \text{ or } e^{2\pi i / 3} \sqrt[3]{2} \text{ or } e^{-2\pi i / 3} \sqrt[3]{2}
    \quad \text{and} \quad
    i\sqrt{3} \mapsto i\sqrt{3} \text{ or } {-i\sqrt{3}}.
  \]
  Combinining these options gives us $6$ distinct
  maps, so these must in fact all be
  $\Q$-automorphisms of $\Q(\sqrt[3]{2}, i\sqrt{3})$,
  since we know the Galois group has size $6$.
  In fact, $\Gal(\Q(\sqrt[3]{2}, i\sqrt{3}) : \Q) \cong S_3 \cong D_3$.
\end{example}

\begin{remark}
  The proper nontrivial subfields of
  $\Q(\sqrt[3]{2}, i\sqrt{3})$ are
  $\Q(\sqrt[3]{2})$, $\Q(e^{2\pi i / 3}\sqrt[3]{2})$,
  $\Q(e^{-2\pi i / 3}\sqrt[3]{2})$, and
  $\Q(i\sqrt{3})$. Maybe draw a pretty diagram with this
  showing the Galois correspondence.
\end{remark}

\begin{exercise}
  Show that $\Z / 6\Z \cong \Z / 2\Z \times \Z / 3\Z$.
\end{exercise}

\begin{exercise}
  Show that $\Z / 4\Z \ncong \Z / 2\Z \times \Z / 2\Z$.
\end{exercise}

\begin{theorem}
  Let $L : K$ be a finite extension.
  Then $\Phi(\Gal(L : K)) = K$ if and only if
  $L : K$ is normal and separable.
\end{theorem}

\begin{proof}
  $(\Leftarrow)$ Let $[L : K] = n$. By Corollary
  \ref{cor:galois-size}, we have $|\Gal(L : K)| = n$.
  Let $K' = \Phi(\Gal(L : K))$. By definition,
  $K \subseteq K'$. By Theorem 7.12 of Howie, we find that
  \[
    [L : K'] = |\Gal(L : K)|.
  \]
  Hence $[L : K'] = [L : K]$ and thus we cocnlude
  that $K = K'$.

  $(\Rightarrow)$ See Howie.
\end{proof}

\begin{exercise}
  Show that if $K \subseteq K'$ and
  $[L : K'] = [L : K]$, then $K = K'$.
\end{exercise}

\begin{theorem}
  Let $L : K$ be Galois and $E$ a subfield of $L$
  containing $K$. If $\delta \in \Gal(L : K)$, then
  \[\Gamma(\delta(E)) = \delta \Gamma(E) \delta^{-1}.\]
\end{theorem}

\begin{proof}
  Next class, see Howie for now.
\end{proof}

\begin{example}
  Consider $\Q(\sqrt[3]{2}, i\sqrt{3}) : \Q$.
  Define the elements of $\Gal(\Q(\sqrt[3]{2}, i\sqrt{3}) : \Q)$ by
  \begin{gather*}
    \mu_1 : \sqrt[3]{2} \mapsto \sqrt[3]{2},\, i\sqrt{3} \mapsto -i\sqrt{3}, \quad
    \mu_2 : \sqrt[3]{2} \mapsto e^{2\pi i / 3} \sqrt[3]{2},\, i\sqrt{3} \mapsto -i\sqrt{3}, \\
    \mu_3 : \sqrt[3]{2} \mapsto e^{-2\pi i / 3}\sqrt[3]{2},\, i\sqrt{3} \mapsto -i\sqrt{3}, \\
    \rho_1 : \sqrt[3]{2} \mapsto e^{2\pi i / 3}\sqrt[3]{2},\, i\sqrt{3} \mapsto i\sqrt{3}, \quad
    \rho_2 : \sqrt[3]{2} \mapsto e^{-2\pi i / 3}\sqrt[3]{2},\, i\sqrt{3} \mapsto i\sqrt{3}.
  \end{gather*}
  Let $\delta = \mu_3$ and $E = \Q(\sqrt[3]{2})$.
  Then $\delta(E) = \Q(e^{-2\pi i / 3}\sqrt[3]{2})$ since
  $\mu_3(\sqrt[3]{2}) = e^{-2\pi i / 3}\sqrt[3]{2}$. Now
  \begin{align*}
    \mu_2(e^{-2\pi i / 3} \sqrt[3]{2})
    &= \mu_2(e^{-2\pi i / 3}) \mu_2(\sqrt[3]{2})
    = \mu_2(-\frac{1}{2} - i\frac{\sqrt{3}}{2}) \mu_2(\sqrt[3]{2}) \\
    &= (-\frac{1}{2} + i\frac{\sqrt{3}}{{2}})(e^{2\pi i / 3} \sqrt[3]{2})
    = e^{2\pi i / 3} e^{2\pi i / 3} \sqrt[3]{2}
    = e^{-2\pi i / 3} \sqrt[3]{2},
  \end{align*}
  so $\Gamma(\delta(E)) = \{\id, \mu_2\}$. Also
  $\Gamma(E) = \{\id, \mu_1\}$, and we find that
  \[
    \delta \Gamma(E) \delta^{-1}
    = \{\delta \id \delta^{-1}, \delta \mu_1 \delta^{-1}\}
    = \{\id, \mu_3 \mu_1 \mu_3^{-1}\} = \{\id, \mu_2\},
  \]
  so indeed we have $\Gamma(\delta(E)) = \delta \Gamma(E) \delta^{-1}$ in this case.
\end{example}

  \chapter{Feb.~26 --- The Fundamental Theorem}

\section{Normal Subgroups}

Recall the following:

\begin{definition}
  A subgroup $H$ of $G$ is \emph{normal} if
  \[
    gHg^{-1} = H
  \]
  for all $g \in G$ (equivalently, $gH = Hg$ for all
  $g \in G$).
\end{definition}

\begin{remark}
  If $G$ is abelian, then every subgroup of $G$ is normal.
\end{remark}

\begin{exercise}
  If $[G : H] = 2$, then $H$ is normal.
\end{exercise}

\begin{remark}
  Normality is a necessary and sufficient condition for
  $G / H$ to be a well-defined group (with operation
  induced by the operation on $G$).
\end{remark}

\begin{theorem}
  Let $\varphi : G \to G'$ be a surjective homomorphism
  with kernel $H$. Then there exists a unique isomorphism
  $\alpha : G / H \to G'$ such that the following diagram
  commutes:
  \[
    \begin{tikzcd}
      G \arrow{r}{\varphi} \arrow[swap]{d}{\pi}
      & G' \\
      G / H \arrow{ur}{\alpha}
    \end{tikzcd}
  \]
  Here $\pi : G \to G / H$ is the canonical projection
  $g \mapsto gH$.
\end{theorem}

\section{Fundamental Theorem of Galois Theory}
\begin{theorem}[Fundamental theorem of Galois theory]
  Let $L : K$ be a separable, normal extension of
  finite degree $n$. Then
  \begin{enumerate}
    \item For all subfields $E$ of $L$ containing
      $K$ and for all subgroups $H$ of $\Gal(L : K)$,
      \begin{enumerate}
        \item $\Phi(\Gamma(E)) = E$ and
          $|\Gamma(E)| = [L : E]$,
        \item $\Gamma(\Phi(H)) = H$ and
          $|{\Gal(L : K)}| / |\Gamma(E)| = [E : K]$.
      \end{enumerate}
    \item A subfield $E$ is a normal extension of $K$
      if and only if $\Gamma(E)$ is a normal subgroup
      of $\Gal(L : K)$. If $E : K$ is normal, then
      \[
        \Gal(E : K) \cong \Gal(L : K) / \Gamma(E).
      \]
  \end{enumerate}
\end{theorem}

\begin{proof}
  (1) By a homework exercise, $L : K$ being normal implies
  that $L : E$ is normal. Also, by Howie's
  Theorem 7.26, $L : K$ being finite and separable
  implies that $L : E$ is separable. Hence $L : E$ is
  Galois, so $|\Gamma(E)| = [L : E]$. Then
  \[
    [E : K] = \frac{[L : K]}{[L : E]}
    = \frac{|{\Gal(L : K)}|}{|\Gamma(E)|}.
  \]
  Now $\Gamma(E) = \Gal(L : E)$, so $L : E$ being
  Galois and Howie's Theorem 7.30 imply that
  $\Phi(\Gamma(E)) = E$. Now let $H$ be a subgroup of
  $\Gal(L : K)$. We showed that
  $H \subseteq \Gamma(\Phi(H))$. Also $\Phi\Gamma\Phi = \Phi$,
  so
  \[
    |H| = [L : \Phi(H)] = [L : \Phi\Gamma\Phi(H)]
    = |\Gamma\Phi(H)|
  \]
  by Howie's Theorem 7.12. Now finiteness and
  $H \subseteq \Gamma(\Phi(H))$ imply that
  $H = \Gamma(\Phi(H))$.

  (2) $(\Rightarrow)$ Suppose $E : K$ is normal and
  let $\delta \in \Gal(L : K)$. Let
  $\delta' = \delta|_E$, the restriction of $\delta$ to
  $E$. Hence $\delta'$ is a monomorphism $E \to L$
  and thus a $K$-automorphism of $E$, by Howie's
  Theorem 7.21. Hence
  \[
    \delta(E) = \delta'(E) = E,
  \]
  and so by Theorem \ref{thm:conjugate},
  \[
    \Gamma(E) = \Gamma(\delta(E)) = \delta \Gamma(E) \delta^{-1},
  \]
  i.e. $\Gamma(E)$ is a normal subgroup of
  $\Gal(L : K)$.

  $(\Leftarrow)$ Suppose $\Gamma(E)$ is a normal
  subgroup of $\Gal(L : K)$. Let $\delta_1$ be
  a $K$-monomorphism from $E$ to $L$. This
  extends (by Howie's Corollary 7.14) to a
  $K$-automorphism $\delta$ of $L$. Since
  $\Gamma(E)$ is normal, $\delta \Gamma(E) \delta^{-1} = \Gamma(E)$.
  Hence by Theorem \ref{thm:conjugate}, we get
  $\Gamma(\delta(E)) = \Gamma(E)$. Since $\Gamma$ is
  injective,
  \[
    \delta_1(E) = \delta(E) = E,
  \]
  so $\delta$ is a $K$-automorphism of $E$. By Howie's
  Theorem 7.21, this implies $E : K$ is normal.

  Now suppose $E : K$ is normal, and we want to show
  that
  \[\Gal(E : K) \cong \Gal(L : K) / \Gamma(E).\]
  Let $\delta \in \Gal(L : K)$ and $\delta' = \delta|_E$.
  By Howie's Theorem 7.21, having $E : K$ be normal
  implies that $\delta'(E) = E$.
  Thus we can define
  $\theta : \Gal(L : K) \to \Gal(E : K)$ by
  $\delta \mapsto \delta'$, i.e. restricting
  $\delta$ to $E$. Clearly $\theta$ is surjective
  onto $\Gal(E : K)$. Also, we see that
  \[
    \ker \theta = \{\delta \in \Gal(L : K) \mid \delta|_E = \id_E\} = \Gamma(E).
  \]
  Hence by the first isomorphism theorem,
  $\Gal(E : K) \cong \Gal(L : K) / {\ker \theta} = \Gal(L : K) / \Gamma(E)$.
\end{proof}

\begin{exercise}
  Show that $\Phi \Gamma \Phi = \Phi$.
\end{exercise}

\begin{exercise}
  Check that $\theta$ is a homomorphism.
\end{exercise}

\begin{example}
  Let $L = \Q(\sqrt[4]{2}, i)$ with $[L : \Q] = 8$.
  Any $\Q$-automorphism in $\Gal(L : \Q)$ must map
  \[
    i \mapsto \pm i, \quad \sqrt[4]{2} \mapsto \pm \sqrt[4]{2}, \pm i\sqrt[4]{2}.
  \]
  So there are only $8$ possible automorphisms, and thus
  each of these must in fact be automorphisms since
  $|{\Gal(L : \Q)}| = [L : \Q] = 8$. We can enumerate these
  automorphisms via
  \begin{gather*}
    \id, \quad \alpha : \sqrt[4]{2} \mapsto i\sqrt[4]{2}, i \mapsto i, \quad
    \beta : \sqrt[4]{2} \mapsto -\sqrt[4]{2}, i \mapsto i,
    \quad \gamma : \sqrt[4]{2} \mapsto -i\sqrt[4]{2}, i \mapsto i, \\
    \lambda : \sqrt[4]{2} \mapsto \sqrt[4]{2}, i \mapsto -i, \quad
    \mu : \sqrt[4]{2} \mapsto i\sqrt[4]{2}, i \mapsto -i, \quad
    \nu : \sqrt[4]{2} \mapsto -\sqrt[4]{2}, i \mapsto -i,  \\
    \rho : \sqrt[4]{2} \mapsto -i\sqrt[4]{2}, i \mapsto -i.
  \end{gather*}
  Note that $\Gal(L : \Q)$ is not abelian, as
  \[
    \lambda\alpha(\sqrt[4]{2}) = \lambda(i\sqrt[4]{2})
    = -i\sqrt[4]{2}, \quad \lambda \alpha(i) = \lambda(i)
    = i,
  \]
  so $\lambda \alpha = \rho$. We can show as an exercise
  that $\alpha \lambda = \mu \ne \rho$, so
  $\lambda \alpha \ne \alpha \lambda$. The subgroups of
  $\Gal(L : \Q)$ are
  \begin{gather*}
    G = \Gal(L : \Q), \quad \{\id\}, \quad
    \{\id, \beta\}, \quad \{\id, \mu\}, \quad
    \{\id, \nu\}, \quad \{\id, \rho\}, \\
    \{\id, \alpha, \beta, \gamma\}, \quad
    \quad \{\id, \beta, \lambda, \nu\},
    \quad \{\id, \beta, \mu, \rho\}.
  \end{gather*}
  Now we could draw a nice subgroup lattice for this
  (identical to $D_4$, the dihedral group of order $8$).
  The normal subgroups of $\Gal(L : \Q)$ are
  \[
    G, \quad \{\id, \beta, \lambda, \nu\},
    \quad \{\id, \alpha, \beta, \gamma\}, \quad
    \{\id, \beta, \mu, \rho\}, \quad \{\id, \beta\},
    \quad \{\id\}.
  \]
  Let $H_1 = \{\id, \alpha, \beta, \gamma\}$. Then
  $\Phi(H_1) = \Q(i)$. Also
  $\Phi(\{\id, \lambda\}) = \Q(\sqrt[4]{2})$ and
  $\Phi(\{\id, \nu\}) = \Q(i\sqrt[4]{2})$. We can also
  see that $\Phi(\{\id, \mu\}) = \Q((1 + i)\sqrt[4]{2})$
  and $\Phi(\{\id, \rho\}) = \Q((1 - i)\sqrt[4]{2})$.
\end{example}

\begin{exercise}
  Write out the multiplication table for $\Gal(L : \Q)$.
\end{exercise}

\end{document}
