\chapter{Mar.~13 --- Symmetric Polynomials}

\section{Transcendental Extensions}

\begin{theorem}
  Let $K, L, m, E, \beta_1, \dots, \beta_m$ be as
  defined in Theorem
  \ref{thm:transcendental-part}.
  If $K \subseteq F \subseteq L$ and
  \begin{enumerate}
    \item $F = K(\gamma_1, \gamma_2, \dots, \gamma_p)$
      where $\{\gamma_1, \gamma_2, \dots, \gamma_p\}$ are
      algebraically independent over $K$,
    \item $[L : F]$ is finite,
  \end{enumerate}
  then $p = m$.
\end{theorem}

\begin{proof}
  Suppose $p > m$. Since $[L : E]$ is finite,
  $\gamma_1$ is algebraic over $E$, so
  $\gamma_1$ is the root of a polynomial with coefficients
  in $E = K(\beta_1, \dots, \beta_m)$. In other words,
  there exists a nonzero polynomial $f$ with coefficients
  in $K$ such that
  \[
    f(\beta_1, \dots, \beta_m, \gamma_1) = 0.
  \]
  Since $\gamma_1$ is transcendental over $K$,
  at least one $\beta_i$ (without loss of generality
  say $\beta_1$) must show up in this polynomial. Hence
  $\beta_1$ is algebraic over
  $K(\beta_2, \beta_3, \dots, \beta_m, \gamma_1)$
  and $[L : K(\beta_2, \dots, \beta_m, \gamma_1)]$
  is finite. Repeat this argument, replacing
  each $\beta_i$ with $\gamma_i$, so
  $[L : K(\gamma_1, \dots, \gamma_m)]$ is finite.
  Recall that $p > m$ by assumption.
  But $\gamma_{m + 1}$ is transcendental over
  $K(\gamma_1, \dots, \gamma_m)$, a contradiction. Thus
  we must have $p \le m$.

  We also get $m \le p$ for free by symmetry, so we
  conclude that $p = m$.
\end{proof}

\begin{definition}
  The $m$ in Theorem \ref{thm:transcendental-part} is
  called the \emph{transcendence degree} of $L : K$.
\end{definition}

\section{Symmetric Polynomials}

\begin{definition}
  Let $L = K(t_1, t_2, \dots, t_n)$ where
  $\{t_1, \dots, t_n\}$ are algebraically independent
  over $K$. For $\sigma \in S_n$, define the
  $K$-automorphism $\varphi_\sigma : L \to L$
  by $\varphi_\sigma(t_i) = t_{\sigma(i)}$, i.e. it
  permutes the $t_i$'s by $\sigma$.
  Let
  \[
    \Aut_n = \{\varphi_\sigma \mid \sigma \in S_n\}.
  \]
\end{definition}

\begin{example}
  If $\sigma = (1\ 2\ 3)$, then we have
  \[
    \varphi_\sigma \left(\frac{t_1 + t_2}{t_3}\right)
    = \frac{t_2 + t_3}{t_1}.
  \]
\end{example}

\begin{exercise}
  Show that the map $\sigma \mapsto \varphi_\sigma$
  is an isomorphism $S_n \to \Aut_n$.
\end{exercise}

\begin{example}
  What is $\Phi(\Aut_n)$, the fixed field of $\Aut_n$?
  Certainly $\Phi(\Aut_n)$ includes all of
  \begin{align*}
    s_1 &= t_1 + t_2 + \dots + t_n, \\
    s_2 &= t_1t_2 + t_1t_3 + \dots + t_{n - 1}t_n, \\
    & \vdots \\
    s_n &= t_1t_2 \dots t_n.
  \end{align*}
  We call these the
  \emph{elementary symmetric polynomials}. All rational
  combinations of the $s_i$ are also fixed.
\end{example}

\begin{exercise}
  Show
  \[
    X^n - s_1 X^{n - 1} + \dots + (-1)^n s_n
    = \prod_{i = 1}^n (X - t_i).
  \]
\end{exercise}

\begin{example}
  The sum of the squares of the $t_i$ is fixed by
  $\Aut_n$. We can also see that
  \[
    t_1^2 + t_2^2 + \dots + t_n^2 = s_1^2 - 2s_2.
  \]
\end{example}

\begin{theorem}
  The fixed field of $\Aut_n$ is precisely
  $\Phi(\Aut_n) = K(s_1, s_2, \dots, s_n)$.
\end{theorem}

\begin{proof}
  We claim $[K(t_1, \dots, t_n) : K(s_1, \dots, s_n)] \le n!$.
  The proof follows since
  $K(s_1, \dots, s_n) \subseteq \Phi_n(\Aut_n)$ and we
  have\footnote{Note that $K(s_1, \dots, s_n) \subseteq \Phi(\Aut_n) \subseteq K(t_1, \dots, t_n)$.}
  \[
    [K(t_1, \dots, t_n) : \Phi_n(\Aut_n)]
    = |{\Aut_n}| = n!.
  \]
  So it suffices to prove the claim to finish.
  
  We show the claim by induction on $n$. The base
  case $n = 1$ is clear. Now for the inductive step,
  suppose we have
  \[
    K(t_1, \dots, t_n) \supseteq K(s_1, \dots, s_n, t_n)
    \supseteq K(s_1, \dots, s_n).
  \]
  Note that
  \[
    f(X) = X^n - s_1X^{n - 1} + \dots + (-1)^n s_n
    = (X - t_1)\dots(X - t_n)
  \]
  over $K(t_1, \dots, t_n)$, so the minimum polynomial
  of $t_n$ over $K(s_1, \dots, s_n)$ divides $f$.
  So we get
  \[K(t_1, \dots, t_n) : K(s_1, \dots, s_n)] \le n. \tag{$\star$}\]
  Now let $s_1', \dots, s_{n - 1}'$ be the elementary
  symmetric polynomials in $t_1, \dots, t_{n - 1}$, and
  notice that
  \begin{align*}
    s_1' &= t_1 + t_2 + \dots + t_{n - 1}, \\
    s_2 &= s_1' + t_n \\
    & \vdots \\
    s_{j} &= s_j' + s_{j - 1}' t_n, \\
           & \vdots \\
    s_{n} &= s_{n - 1}' t_n.
  \end{align*}
  So $K(s_1, \dots, s_n) = K(s_1', \dots, s_{n - 1}', t_n)$
  and so
  \begin{align*}
    [K(t_1, \dots, t_n) : K(s_1, \dots, s_n t_n)]
    &= [K(t_1, \dots, t_n) : K(s_1', \dots, s_{n - 1}', t_n)] \\
    &= [K(t_n)(t_1, \dots, t_{n - 1}) : K(t_n)(s_1', \dots, s_{n - 1}')]
    \le (n - 1)!
  \end{align*}
  by the inductive hypothesis. So this combined
  with $(\star)$ completes the inductive step.
\end{proof}

\begin{theorem}
  The elementary symmetric polynomials $s_1, \dots, s_n$
  are algebraically independent.
\end{theorem}

\begin{proof}
  We have $[K(t_1, \dots, t_n) : K(s_1, \dots, s_n)]$ is
  finite since $t_1, \dots, t_n$ are roots of
  \[
    X^n - s_1X^{n - 1} + \dots + (-1)^n s_n.
  \]
  Hence $K(t_1, \dots, t_n)$ and
  $K(s_1, \dots, s_n)$ have the same transcendence
  degree over $K$, namely $n$, so we get that
  $s_1, \dots, s_n$ must be algebraically independent.
\end{proof}

\begin{definition}
  The \emph{general polynomial} of degree $n$ over $K$ is
  \[
    f = X^n - s_1X^{n - 1} + \dots + (-1)^n s_n.
  \]
\end{definition}

\begin{remark}
  Note that:
  \begin{enumerate}
    \item The coefficients live in $K(s_1, \dots, s_n)$.
    \item For now, $s_i$ are just algebraically
      independent elements.
  \end{enumerate}
\end{remark}

\begin{theorem}
  \label{thm:realize-sn}
  Let $\Char K = 0$ and $f$ as above. Let $L$ be a
  splitting field for $f$ over $K(s_1, \dots, s_n)$.
  Then
  \begin{enumerate}
    \item the zeros $t_1, \dots, t_n$ of $f$ in $L$
      are algebraically independent over $K$,
    \item and $\Gal(L : K(s_1, \dots, s_n)) = S_n$.
  \end{enumerate}
\end{theorem}

\begin{proof}
  Note that $[L : K(s_1, \dots, s_n)]$ is finite, so
  the transcendence degree of $L$ over $K$ is
  the transcendence degree of $K(s_1, \dots, s_n)$ over
  $K$, which is $n$. So $L = K(t_1, \dots, t_n)$, which
  means that $t_1, \dots, t_n$ must be algebraically
  independent. Then we have that
  \[
    X^n - s_1X^{n - 1} + \dots + (-1)^n s_n
    = \prod_{i = 1}^n (X - t_i),
  \]
  so $s_1, \dots, s_n$ are precisely the elementary
  symmetric polynomials in $t_1, \dots, t_n$. So by
  Theorem 10.8 from Howie, we get $\Phi(\Aut_n) = K(s_1, \dots, s_n)$.
  From here we have
  \[
    [L : K(s_1, \dots, s_n)]
    = [L : \Phi(\Aut_n)]
    = |{\Aut_n}| = |S_n| = n!,
  \]
  so $\Gal(L : K(s_1, \dots, s_n)) \cong S_n$.
\end{proof}

\begin{corollary}
  If $\Char K = 0$ and $n \ge 5$, then the general
  polynomial
  \[
    X^n - s_1X^{n - 1} + \dots + (-1)^n s_n
  \]
  is not solvable by radicals.
\end{corollary}

\begin{corollary}
  Every finite group is the Galois group of some
  field extension.
\end{corollary}

\begin{proof}
  Recall that by Cayley's theorem, every finite group
  is a subgroup of $S_n$ for some $n$. By Theorem
  \ref{thm:realize-sn}, we can realize $S_n$ as the
  Galois group of $L : K(s_1, \dots, s_n)$. The
  fundamental theorem of Galois theory then says that
  for every subgroup $G$ of $S_n$, there exists a
  subfield $M$ of $L$ containing $K(s_1, \dots, s_n)$
  such that $G = \Gal(L : M)$.
\end{proof}

\begin{remark}
  In the above theorem, we kind of lost control of the
  ground field, which is just some field $M$. Given a
  finite group $G$, is it the Galois group of a Galois
  extension over $\Q$? Equivalently, does there exist
  $f \in \Q[X]$ such that $G = \Gal(f)$?
  If so, we say
  that $G$ is \emph{realizable} (over $\Q$).
  This is known
  as the \emph{inverse Galois problem}.
\end{remark}

\begin{remark}
  In 1956, Shafarevich showed that every solvable
  group is realizable. An open question is: Is every
  finite simple group realizable?
\end{remark}
