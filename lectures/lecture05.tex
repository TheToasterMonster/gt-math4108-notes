\chapter{Jan.~24 --- Algebraic Extensions}

\section{Minimal Polynomials}
\begin{remark}
  The $m$ in the previous theorem from last class is
  called the \emph{minimal polynomial} of $\alpha$.
\end{remark}

\begin{example}
  Let 
  \[\Q[i\sqrt{3}] = \{a + bi\sqrt{3} \mid a, b \in \Q\} \subseteq \C.\]
  Here $m = X^2 + 3$, so this is a degree 2 extension.
\end{example}

\begin{exercise}
Write $1 / (a + bi\sqrt{3})$ in the form $c + di\sqrt{3}$.
\end{exercise}

\begin{example}
  Is $\Q(\sqrt{2}, \sqrt{3})$ a simple extension?
  In fact it is! Note that certainly
  \[
    \Q(\sqrt{2} + \sqrt{3}) \subseteq \Q(\sqrt{2}, \sqrt{3}).
  \]
  For the reverse inclusion, observe that
  $(\sqrt{3} + \sqrt{2})(\sqrt{3} - \sqrt{2}) = 1$,
  so
  \[1 / (\sqrt{3} + \sqrt{2}) = \sqrt{3} - \sqrt{2} \in \Q(\sqrt{2} + \sqrt{3}).\]
  From this we have
  \[
    (\sqrt{3} + \sqrt{2}) + (\sqrt{3} - \sqrt{2})
    = 2\sqrt{3},
  \]
  which implies that $\sqrt{3} \in \Q(\sqrt{2} + \sqrt{3})$.
  Similarly $\sqrt{2} \in \Q(\sqrt{2} + \sqrt{3})$, so that
  $\Q(\sqrt{2}, \sqrt{3}) \subseteq \Q(\sqrt{2} + \sqrt{3})$.
  Now we can consider
  \[\Q(\sqrt{2}, \sqrt{3}) = \Q[\sqrt{2}, \sqrt{3}] = (\Q[\sqrt{2}])[\sqrt{3}].\]
  First we have $[Q[\sqrt{2}]:\Q] = 2$. Note that
  $X^2 - 3$ is the minimal polynomial of
  $\sqrt{3}$ over $\Q[\sqrt{2}]$, so
  $[\Q[\sqrt{2}, \sqrt{3}]:\Q[\sqrt{2}]] = 2$.
  Hence $[\Q[\sqrt{2}, \sqrt{3}]:\Q] = 4$ with
  basis $\{1, \sqrt{2}, \sqrt{3}, \sqrt{6}\}$.\footnote{Since $\Q[\sqrt{2}, \sqrt{3}] = \Q[\alpha]$ where $\alpha = \sqrt{2} + \sqrt{3}$, we have $\{1, \alpha, \alpha^2, \alpha^3\}$ as another basis.}
  To find the  minimal polynomial of
  $\sqrt{2} + \sqrt{3}$ over $\Q$, we can compute
  \begin{align*}
    (\sqrt{2} + \sqrt{3})^2
    &= 2 + 2\sqrt{6} + 3 = 5 + 2\sqrt{6} \\
    (\sqrt{2} + \sqrt{3})^4
    &= 25 + 20\sqrt{6} + 24 = 49 + 20\sqrt{6}.
  \end{align*}
  Thus $X^4 - 10X^2 + 1$ is the minimal polynomial,
  since $\alpha^4 - 10\alpha^2 + 1 = 0$ for
  $\alpha = \sqrt{2} + \sqrt{3}$.
\end{example}

\section{Algebraic Extensions}
\begin{definition}
  If $\alpha$ has a minimal polynomial over $K$,
  we say $\alpha$ is \emph{algebraic} over $K$, and
  $K[\alpha] = K(\alpha)$ is an \emph{algebraic extension} of $K$.
  A complex number that is algebraic over $\Q$
  is called an \emph{algebraic number}.
  Otherwise, if $K(\alpha) \cong K(X)$, then we say
  $\alpha$ is
  \emph{transcendental} over $K$. A transcendental
  number $\alpha$ is a complex number that is
  transcendental over $\Q$.
\end{definition}

\begin{example}
  We have that $\Q(i\sqrt{3})$, $\Q(\sqrt{2})$, $\Q(\sqrt{3})$, and
  $\Q(\sqrt{2}, \sqrt{3})$ are all simple algebraic
  extensions of $\Q$, whereas $\Q(X)$ is a simple
  transcendental extension of $\Q$.
\end{example}

\begin{theorem}
  Let $K(\alpha)$ be a simple transcendental extension
  of $K$. Then $[K(\alpha) : K] = \infty$.
\end{theorem}

\begin{proof}
  Observe that $1, \alpha, \alpha^2, \dots$ are
  linearly independent over $K$, since no minimal
  polynomial exists.
\end{proof}

\begin{definition}
  An extension $L$ over $K$ is an \emph{algebraic extension}
  if any element of $L$ is algebraic over $K$. Otherwise,
  $L$ is a \emph{transcendental extension}.
\end{definition}

\begin{theorem}
  Every finite extension is algebraic.
\end{theorem}

\begin{proof}
  Let $L : K$ be a finite extension and suppose for
  sake of contradiction that $\alpha \in L$ is
  transcendental over $K$. Then
  $1, \alpha, \alpha^2, \dots$ are linearly independent,
  contradicting the fact that $L : K$ is finite.
\end{proof}

\begin{theorem}
  Let $L : K$ be a field extension and let
  $\mathcal{A}(L)$ be the set of elements in $L$ that
  are algebraic over $K$. Then $\mathcal{A}(L)$ is a
  subfield of $L$.
\end{theorem}

\begin{proof}
  See Howie. Just need to show the closure of
  algebraic elements under usual field operations.
\end{proof}

\begin{example}
  For $L = \C$ and $K = \Q$, we have that
  $\mathcal{A}(\C)$ is
  the field $\mathbb{A}$ of algebraic numbers.
\end{example}

\begin{theorem}
  The set of algebraic numbers $\mathbb{A}$ is countable.
\end{theorem}

\begin{proof}[Proof sketch]
  Note that the set of monic polynomials of degree
  $n$ with coefficients in $\Q$ is countable, and
  each such polynomial has at most $n$ distinct roots
  in $\C$. Hence the number of roots of such polynomials
  is countable. Then $\mathbb{A}$ is the countable
  union of countable sets, so $\mathbb{A}$ is countable.
\end{proof}

\begin{theorem}
  Transcendental numbers exist.
\end{theorem}

\begin{proof}
  Since $|\R| = |\C| = 2^{\aleph_0} > \aleph_0$, we must
  have that $\C \setminus \mathbb{A}$ is nonempty.
\end{proof}

\begin{remark}
  The above proof is very nonconstructive, what about
  actual examples of transcendental numbers? In
  1844, Liouville constructed the following example:
  \[
    \sum_{n = 1}^\infty 10^{-n!},
  \]
  which was shown to be transcendental. In 1873,
  Hermite showed that $e$ is transcendental, and in 1882,
  Lindemann showed that $\pi$ is transcendental.
\end{remark}

\begin{theorem}
  Let $L : K$ be a field extension and
  $\alpha_1, \dots, \alpha_n \in L$ have minimal
  polynomials $m_1, \dots, m_n$, respectively.
  Then
  $[K(\alpha_1, \dots, \alpha_n) : K] \le \partial m_1 \partial m_2 \dots \partial m_n$.
\end{theorem}

\begin{proof}
  See Howie. Uses induction and the fact that
  $[M : L][L : K] = [M : K]$.
\end{proof}

\begin{example}
  Consider
  \[
    [\Q[\sqrt{2}] : \Q] = [\Q[\sqrt{3}] : \Q]
    = [\Q[\sqrt{6}] : \Q] = 2,
  \]
  but $[\Q[\sqrt{2}, \sqrt{3}, \sqrt{6}] : \Q] = 4$.
  So the bound in the previous theorem cannot be made
  into an equality.
\end{example}

\begin{prop}
  A field extension $L : K$ is finite if and only if
  for some $n$, there exist $\alpha_1, \dots, \alpha_n$
  algebraic over $K$ such that
  $L = K(\alpha_1, \dots, \alpha_n)$.
\end{prop}

\begin{proof}
  $(\Leftarrow)$ This is precisely the previous theorem.

  $(\Rightarrow)$ Suppose $L : K$ is finite and
  $\{\alpha_1, \dots, \alpha_n\}$ is a basis for $L$
  over $K$. Since finite extensions are algebraic,
  the $\alpha_i$ must be algebraic.
\end{proof}

\begin{exercise}
  Show that $\varphi : \Q[\sqrt{2}] \to \Q[X] / \langle X^2 - 2 \rangle$
  defined by
  \[a + b\sqrt{2} \mapsto a + bX + \langle X^2 - 2 \rangle\]
  is an isomorphism.
\end{exercise}

\begin{theorem}
  Let $K$ be a field and $m$ a monic irreducible
  polynomial in $K[X]$. Then $L = K[X] / \langle m \rangle$
  is a simple algebraic extension $K[\alpha]$ of $K$,
  and $\alpha = X + \langle m \rangle$ has minimal
  polynomial $m$ over $K$.
\end{theorem}

\begin{proof}
  First note that $L$ is indeed a field since $m$
  is irreducible. Also $L : K$ is indeed a field
  extension since $\varphi : K \to L$ defined by
  $a \mapsto a + \langle m \rangle$ is an injective
  homomorphism. Now let $\alpha = X + \langle m \rangle$.
  For
  \[
    f = a_0 + a_1 X + \dots + a_n X^n \in K[X],
  \]
  we have
  \begin{align*}
    f(\alpha)
    &= a_0 + a_1 \alpha + \dots + a_n \alpha^n
    = a_0 + a_1 (X + \langle m \rangle) + \dots + a_n (X + \langle m \rangle)^n \\
    &= a_0 + a_1 X + \dots + a_n X^n + \langle m \rangle
    = f + \langle m \rangle.
  \end{align*}
  So $f(\alpha) = 0$ if and only if
  $f \in \langle m \rangle$, i.e. $m | f$. Hence
  $m$ is the minimal polynomial of $\alpha$.
\end{proof}
