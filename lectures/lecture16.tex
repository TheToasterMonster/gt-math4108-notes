\chapter{Mar.~11 --- Solvable Polynomials}

\section{More on Cyclotomic Polynomials}
\begin{exercise}
  From Example \ref{ex:splitting-field-of-x5-7-over-q},
  check that
  \[
    (\sigma_{1, 1})^n = \sigma_{n, 1}, \quad
    (\sigma_{0, 2})^n = \sigma_{0, 2^n}, \quad
    \sigma_{2, 1} \sigma_{0, 2} = \sigma_{2, 2} = \sigma_{0, 2} \sigma_{1, 1}.
  \]
  Let $a = \sigma_{1, 1}$ and $b = \sigma_{0, 2}$. Use
  the above to show that
  \[
    \Gal(L : \Q) = \langle a, b \mid a^5 = 1,\, b^4 = 1,\, a^2 b = ba \rangle
  \]
  is a presentation for $\Gal(L : \Q)$
  in terms of generators and relations.
\end{exercise}

\begin{theorem}
  \label{thm:char-zero-cyclic-converse}
  Let $\Char K = 0$ and suppose $X^m - 1$ splits
  completely over $K$. Let $L : K$ be a cyclic extension
  with $[L : K] = m$. Then there exists $a \in K$ such
  that
  \begin{enumerate}
    \item $X^m - a$ is irreducible over $K$,
    \item $L$ is a splitting field for $X^m - a$ over $K$,
    \item and $L = K(\alpha)$ where $\alpha$ is a root of
      $X^m - a$.
  \end{enumerate}
\end{theorem}

\begin{proof}
  See Howie.
\end{proof}

\begin{remark}
  This is a partial converse to Theorem
  \ref{thm:galois-group-of-xm-a}.
\end{remark}

\section{Solvable Polynomials}

\begin{remark}
  For $f \in K[X]$, we define
  $\Gal(f) = \Gal(L : K)$ where $L$ is a splitting
  field for $f$ over $K$.
\end{remark}

\begin{theorem}
  Let $\Char K = 0$ and $f \in K[X]$. If
  $\Gal(f)$ is solvable, then $f$ is solvable
  by radicals.
\end{theorem}

\begin{proof}
  Let $L$ be a splitting field of $f$ over $K$, where
  $\Gal(L : K)$ is solvable by hypothesis, and let
  $m = |{\Gal(L : K)}|$. If $K$ does not contain an
  $m$th root of unity, adjoin one, i.e. let $E$ be the
  splitting field of $X^m - 1$ over $K$. Let $M$ be
  the splitting field of $f$ over $E$. This gives the
  subfield lattice:
  \[
    \begin{tikzcd}[column sep=1em, row sep=1.5em]
      & M \\
      E & & L \\
        & E \cap L \\
        & K
      \arrow[from=2-1, to=1-2]
      \arrow[from=2-3, to=1-2]
      \arrow[from=3-2, to=2-1]
      \arrow[from=3-2, to=2-3]
      \arrow[from=4-2, to=3-2]
    \end{tikzcd}
  \]
  By Theorem 7.36 of Howie, we get
  $G = \Gal(M : E) \cong \Gal(L : E \cap L)$. Now
  $\Gal(L : E \cap L) \subseteq \Gal(L : K)$, i.e. $G$
  is isomorphic to a subgroup of a solvable group, hence
  it is also solvable. So
  \[
    \{1\} = G_0 \triangleleft G_1 \triangleleft \cdots \triangleleft G_{r - 1} \triangleleft G_r = G,
  \]
  with $G_{i + 1} / G_i$ cyclic. By the fundamental
  theorem of Galois theory, we get
  \[
    M_0 = M \supseteq M_1 \supseteq \cdots \subseteq M_{r - 1} \supseteq M_r = E \supseteq K,
  \]
  where $M_i : M_{i + 1}$ is normal. We have
  $\Gal(M : M_i) = G_i$, so
  \[
    \Gal(M_i : M_{i + 1}) \cong \Gal(M : M_i) / \Gamma(M_i)
    \cong G_{i + 1} / G_i,
  \]
  which yields that $M_i : M_{i + 1}$ is cyclic.
  Let $d_i = [M_i : M_{i + 1}]$. Then
  $d_1 \big| [M : E] = \Gal(M : E)$. Now since
  $\Gal(M : E) \cong \Gal(L : E \cap L)$, we have that
  \[
    |{\Gal(L : E \cap L)}| \big| |{\Gal(L : K)}| = m,
  \]
  so $d_1 | m$. Since $M_{i + 1}$ contains $E$, it
  contains every $m$th root of unity, so $E$ also contains
  all $d_i$th roots of unity. By Theorem
  \ref{thm:galois-group-of-xm-a}, there exists
  $\beta_i \in M_i$ such that $M_i = M_{i + 1}(\beta_i)$,
  where $\beta_i$ is a root of $X^{d_i} - c_{i + 1}$ with
  $c_{i + 1} \in M_{i + 1}$. Hence we get that
  $f$ is solvable by radicals.
\end{proof}

\begin{theorem}
  Let $\Char K = 0$ and $K \subseteq L \subseteq M$ where
  $M$ is a radical extension. Then $\Gal(L : K)$ is
  solvable.
\end{theorem}

\begin{proof}
  By hypothesis, there exists a sequence
  \[
    M_r = M \supseteq M_{r - 1} \supseteq \cdots \supseteq M_1 \supseteq M_0 = K,
  \]
  where $M_{i + 1} = M_i(\alpha_i)$ with $\alpha_i$
  a root of $X^{n_i} - a_i \in M_i[X]$.
  The main idea from here is that if $L : K$ and
  $M : K$ are normal, then
  \[
    \Gal(L : K) \cong \Gal(M : K) / \Gal(M : L),
  \]
  so it is sufficient to show that $\Gal(M : K)$ is
  solvable. Now use Theorem 8.18 and Corollary 8.14 from
  Howie to show that $\Gal(M : K)$ is solvable (uses
  induction). See Howie for details.
\end{proof}

\begin{theorem}
  A polynomial $f \in K[X]$ with $\Char K = 0$ is
  solvable by radicals if and only if $\Gal(f)$ is
  solvable.
\end{theorem}

\begin{proof}
  This is summarizing the previous two theorems.
\end{proof}

\section{Insolvability of the Quintic}

\begin{theorem}
  Let $f \in \Q[X]$ be a monic irreducible polynomial
  with $\partial f = p$, $p$ prime. Suppose $f$ has
  exactly two roots in $\C \setminus \R$. Then
  $\Gal(f) = S_p$.
\end{theorem}

\begin{proof}
  Let $L \subseteq \C$ be a splitting field for $f$. Now
  $G = \Gal(L : \Q)$ is a subgroup of $S_p$ since
  $G$ is a group of permutations on the $p$ roots of $f$
  in $L$. Consider $\Q(\alpha)$, where $\alpha$ has
  minimum polynomial $f$. Then $[\Q(\alpha) : \Q] = p$, so
  we get that
  \[
    |G| = |{\Gal(L : \Q)}| = [L : \Q] = [L : \Q(\alpha)] [\Q(\alpha) : \Q]
    = [L : \Q(\alpha) \cdot p.
  \]
  By the Sylow theorems, $G$ has an element of order $p$.\footnote{Cauchy's theorem directly gives this, but also $|S_p| = p!$, so the $p$-Sylow subgroup can only have order $p$.}
  Now $G$ is a subgroup of $S_p$, and the only elements
  in $S_p$ of order $p$ are $p$-cycles, so
  $G$ contains a $p$-cycle. Also complex roots of
  $f$ come in conjugate pairs, so $G$ contains a
  transposition $\tau$ that swaps conjugate roots (there
  are only two complex roots of $f$ in $\C \setminus \R$).
  Then $G$ is a subgroup of $S_p$ that contains a
  $p$-cycle and a transposition, so by Homework 8,
  $G = S_p$.
\end{proof}

\begin{example}
  Consider the polynomial $f = X^5 - 8X + 2$, which
  is irreducible over $\Q$ by Eisenstein's criterion.
  Now we have:
  \[
    \begin{tabular}
      {c|ccccc}
      $X$ & $-2$ & $-1$ & $0$ & $1$ & $2$ \\
      \hline
      $f(X)$ & $-14$ & $9$ & $2$ & $-5$ & $18$
    \end{tabular}
  \]
  So by the intermediate value theorem, $f$ has
  at least $3$ real roots. Then $f'(X) = 5X^4 - 8$, and
  $f'(X) \le 0$ if and only if
  \[
    -\sqrt[4]{\frac{8}{5}} \le X \le \sqrt[4]{\frac{8}{5}}
    \approx 1.12.
  \]
  Rolle's theorem tells us that there exists at least
  one zero of $f'(X)$ between zeroes of $f(X)$.\footnote{The above conditions guarantee that $f'(X)$ has only two zeroes, so $f(X)$ can have at most three.} Thus
  $f$ has exactly $3$ real roots. Then by the previous
  theorem, $\Gal(f) = S_5$, so $f$ is not solvable
  by radicals since $S_5$ is not solvable. So there
  exists a quintic polynomial which is not solvable
  by radicals.
\end{example}

\section{Finitely-Generated Extensions}

\begin{definition}
  A subseteq $\{\alpha_1, \alpha_2, \dots, \alpha_n\} \subseteq L$
  is \emph{algebraically independent} over $K$ if
  for all polynomials $f(X_1, X_2, \dots, X_n)$ with
  coefficients in $K$, we have
  \[
    f(\alpha_1, \alpha_2, \dots, \alpha_n) = 0
    \iff f = 0..
  \]
\end{definition}

\begin{example}
  Notably, this is a stronger condition than linear
  independence. A non-example is
  $\{1, \sqrt{2}, \sqrt{3}, \sqrt{6}\}$, which is linearly
  independent over $\Q$ but not algebraically independent,
  since
  \[
    \sqrt{2} \cdot \sqrt{3} - \sqrt{6} = 0.
  \]
  This means we can take $f(X_1, X_2, X_3, X_4) = X_2 \cdot X_3 - X_4$ to
  get $f(1, \sqrt{2}, \sqrt{3}, \sqrt{6}) = \sqrt{2} \cdot \sqrt{3} - \sqrt{6} = 0$.
\end{example}

\begin{exercise}
  Show that $\{\alpha_1, \dots, \alpha_n\}$  is algebraically
  independent over $K$ if and only if
  $\alpha_1$ is transcendental over $K$ and for each
  $2 \le d \le n$, $\alpha_d$ is transcendental over
  $K(\alpha_1, \dots, \alpha_{d - 1})$. Also show that
  this is if and only if
  \[
    K(\alpha_1, \alpha_2, \dots, \alpha_n) \cong K(X_1, X_2, \dots, X_n).
  \]
\end{exercise}

\begin{definition}
  An extension $L$ of $K$ is \emph{finitely generated}
  if $L = K(\alpha_1, \alpha_2, \dots, \alpha_n)$ for
  some natural number $n$.
\end{definition}

\begin{example}
  Finite extensions are finitely generated.
\end{example}

\begin{example}
  The extension $K(X)$ is finitely generated but not
  a finite extension.
\end{example}

\begin{theorem}
  Let $L = K(\alpha_1, \dots, \alpha_n)$ be a finitely
  generated extension of $K$. Then there exists a field
  $E$ with $K \subseteq E \subseteq L$ such that for
  some $m$ with $0 \le m \le n$,
  \begin{enumerate}
    \item $E = K(\beta_1, \beta_2, \dots, \beta_m)$,
      where $\{\beta_1, \beta_2, \dots, \beta_m\}$ are
      algebraically independent,
    \item and $[L : E]$ is finite.
  \end{enumerate}
\end{theorem}

\begin{proof}
  If all the $\alpha_i$ are algebraic over $K$, then
  $[L : K]$ is finite and we can take $E = K$ with
  $m = 0$. Otherwise, there exists $\alpha_i$ that is
  transcendental over $K$. Let $\beta_1 = \alpha_i$.
  If $[L : K(\beta_1)]$ is not finite, then there exists
  $\alpha_j$ that is transcendental over $K(\beta_1)$.
  Let $\beta_2 = \alpha_j$, and so on. Repeat this
  process, which terminates in at most $n$ steps, so
  \[
    E = K(\beta_1, \beta_2, \dots, \beta_m)
  \]
  with $m \le n$. By construction, $\{\beta_1, \dots, \beta_m\}$
  are algebraically independent over $K$ and
  $[L : E]$ is finite.
\end{proof}

\begin{remark}
  We can think of this theorem as saying that $E$ is
  the ``transcendental part'' of the extension.
\end{remark}

\begin{remark}
  The elements $\beta_i$ are not unique, but the number
  $m$ is determined uniquely by $L$ and $K$.
\end{remark}
