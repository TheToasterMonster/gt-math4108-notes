\chapter{Feb.~26 --- The Fundamental Theorem}

\section{Normal Subgroups}

Recall the following:

\begin{definition}
  A subgroup $H$ of $G$ is \emph{normal} if
  \[
    gHg^{-1} = H
  \]
  for all $g \in G$ (equivalently, $gH = Hg$ for all
  $g \in G$). This is denoted $H \triangleleft G$.
\end{definition}

\begin{remark}
  If $G$ is abelian, then every subgroup of $G$ is normal.
\end{remark}

\begin{exercise}
  If $[G : H] = 2$, then $H$ is normal.
\end{exercise}

\begin{remark}
  Normality is a necessary and sufficient condition for
  $G / H$ to be a well-defined group (with operation
  induced by the operation on $G$).
\end{remark}

\begin{theorem}
  Let $\varphi : G \to G'$ be a surjective homomorphism
  with kernel $H$. Then there exists a unique isomorphism
  $\alpha : G / H \to G'$ such that the following diagram
  commutes:
  \[
    \begin{tikzcd}
      G \arrow{r}{\varphi} \arrow[swap]{d}{\pi}
      & G' \\
      G / H \arrow{ur}{\alpha}
    \end{tikzcd}
  \]
  Here $\pi : G \to G / H$ is the canonical projection
  $g \mapsto gH$.
\end{theorem}

\section{The Fundamental Theorem of Galois Theory}
\begin{theorem}[Fundamental theorem of Galois theory]
  Let $L : K$ be a separable, normal extension of
  finite degree $n$. Then
  \begin{enumerate}
    \item For all subfields $E$ of $L$ containing
      $K$ and for all subgroups $H$ of $\Gal(L : K)$,
      \begin{enumerate}
        \item $\Phi(\Gamma(E)) = E$ and
          $|\Gamma(E)| = [L : E]$,
        \item $\Gamma(\Phi(H)) = H$ and
          $|{\Gal(L : K)}| / |\Gamma(E)| = [E : K]$.
      \end{enumerate}
    \item A subfield $E$ is a normal extension of $K$
      if and only if $\Gamma(E)$ is a normal subgroup
      of $\Gal(L : K)$. If $E : K$ is normal, then
      \[
        \Gal(E : K) \cong \Gal(L : K) / \Gamma(E).
      \]
  \end{enumerate}
\end{theorem}

\begin{proof}
  (1) By a homework exercise, $L : K$ being normal implies
  that $L : E$ is normal. Also, by Howie's
  Theorem 7.26, $L : K$ being finite and separable
  implies that $L : E$ is separable. Hence $L : E$ is
  Galois, so $|\Gamma(E)| = [L : E]$. Then
  \[
    [E : K] = \frac{[L : K]}{[L : E]}
    = \frac{|{\Gal(L : K)}|}{|\Gamma(E)|}.
  \]
  Now $\Gamma(E) = \Gal(L : E)$, so $L : E$ being
  Galois and Howie's Theorem 7.30 imply that
  $\Phi(\Gamma(E)) = E$. Now let $H$ be a subgroup of
  $\Gal(L : K)$. We showed that
  $H \subseteq \Gamma(\Phi(H))$. Also $\Phi\Gamma\Phi = \Phi$,
  so
  \[
    |H| = [L : \Phi(H)] = [L : \Phi\Gamma\Phi(H)]
    = |\Gamma\Phi(H)|
  \]
  by Howie's Theorem 7.12. Now finiteness and
  $H \subseteq \Gamma(\Phi(H))$ imply that
  $H = \Gamma(\Phi(H))$.

  (2) $(\Rightarrow)$ Suppose $E : K$ is normal and
  let $\delta \in \Gal(L : K)$. Let
  $\delta' = \delta|_E$, the restriction of $\delta$ to
  $E$. Hence $\delta'$ is a monomorphism $E \to L$
  and thus a $K$-automorphism of $E$, by Howie's
  Theorem 7.21. Hence
  \[
    \delta(E) = \delta'(E) = E,
  \]
  and so by Theorem \ref{thm:conjugate},
  \[
    \Gamma(E) = \Gamma(\delta(E)) = \delta \Gamma(E) \delta^{-1},
  \]
  i.e. $\Gamma(E)$ is a normal subgroup of
  $\Gal(L : K)$.

  $(\Leftarrow)$ Suppose $\Gamma(E)$ is a normal
  subgroup of $\Gal(L : K)$. Let $\delta_1$ be
  a $K$-monomorphism from $E$ to $L$. This
  extends (by Howie's Corollary 7.14) to a
  $K$-automorphism $\delta$ of $L$. Since
  $\Gamma(E)$ is normal, $\delta \Gamma(E) \delta^{-1} = \Gamma(E)$.
  Hence by Theorem \ref{thm:conjugate}, we get
  $\Gamma(\delta(E)) = \Gamma(E)$. Since $\Gamma$ is
  injective,
  \[
    \delta_1(E) = \delta(E) = E,
  \]
  so $\delta$ is a $K$-automorphism of $E$. By Howie's
  Theorem 7.21, this implies $E : K$ is normal.

  Now suppose $E : K$ is normal, and we want to show
  that
  \[\Gal(E : K) \cong \Gal(L : K) / \Gamma(E).\]
  Let $\delta \in \Gal(L : K)$ and $\delta' = \delta|_E$.
  By Howie's Theorem 7.21, having $E : K$ be normal
  implies that $\delta'(E) = E$.
  Thus we can define
  $\theta : \Gal(L : K) \to \Gal(E : K)$ by
  $\delta \mapsto \delta'$, i.e. restricting
  $\delta$ to $E$. Clearly $\theta$ is surjective
  onto $\Gal(E : K)$. Also, we see that
  \[
    \ker \theta = \{\delta \in \Gal(L : K) \mid \delta|_E = \id_E\} = \Gamma(E).
  \]
  Hence by the first isomorphism theorem,
  $\Gal(E : K) \cong \Gal(L : K) / {\ker \theta} = \Gal(L : K) / \Gamma(E)$.
\end{proof}

\begin{exercise}
  Show that $\Phi \Gamma \Phi = \Phi$.
\end{exercise}

\begin{exercise}
  Check that $\theta$ is a homomorphism.
\end{exercise}

\begin{example}
  \label{ex:big-galois-correspondence}
  Let $L = \Q(\sqrt[4]{2}, i)$ with $[L : \Q] = 8$.
  Any $\Q$-automorphism in $\Gal(L : \Q)$ must map
  \[
    i \mapsto \pm i, \quad \sqrt[4]{2} \mapsto \pm \sqrt[4]{2}, \pm i\sqrt[4]{2}.
  \]
  So there are only $8$ possible automorphisms, and thus
  each of these must in fact be automorphisms since
  $|{\Gal(L : \Q)}| = [L : \Q] = 8$. We can enumerate these
  automorphisms via
  \begin{gather*}
    \id, \quad \alpha : \sqrt[4]{2} \mapsto i\sqrt[4]{2}, i \mapsto i, \quad
    \beta : \sqrt[4]{2} \mapsto -\sqrt[4]{2}, i \mapsto i,
    \quad \gamma : \sqrt[4]{2} \mapsto -i\sqrt[4]{2}, i \mapsto i, \\
    \lambda : \sqrt[4]{2} \mapsto \sqrt[4]{2}, i \mapsto -i, \quad
    \mu : \sqrt[4]{2} \mapsto i\sqrt[4]{2}, i \mapsto -i, \quad
    \nu : \sqrt[4]{2} \mapsto -\sqrt[4]{2}, i \mapsto -i,  \\
    \rho : \sqrt[4]{2} \mapsto -i\sqrt[4]{2}, i \mapsto -i.
  \end{gather*}
  Note that $\Gal(L : \Q)$ is not abelian, as
  \[
    \lambda\alpha(\sqrt[4]{2}) = \lambda(i\sqrt[4]{2})
    = -i\sqrt[4]{2}, \quad \lambda \alpha(i) = \lambda(i)
    = i,
  \]
  so $\lambda \alpha = \rho$. We can show as an exercise
  that $\alpha \lambda = \mu \ne \rho$, so
  $\lambda \alpha \ne \alpha \lambda$. The subgroups of
  $\Gal(L : \Q)$ are
  \begin{gather*}
    G = \Gal(L : \Q), \quad \{\id\}, \quad
    \{\id, \beta\}, \quad \{\id, \mu\}, \quad
    \{\id, \nu\}, \quad \{\id, \rho\}, \\
    \{\id, \alpha, \beta, \gamma\}, \quad
    \quad \{\id, \beta, \lambda, \nu\},
    \quad \{\id, \beta, \mu, \rho\}.
  \end{gather*}
  Now we could draw a nice subgroup lattice for this
  (identical to $D_4$, the dihedral group of order $8$).
  The normal subgroups of $\Gal(L : \Q)$ are
  \[
    G, \quad \{\id, \beta, \lambda, \nu\},
    \quad \{\id, \alpha, \beta, \gamma\}, \quad
    \{\id, \beta, \mu, \rho\}, \quad \{\id, \beta\},
    \quad \{\id\}.
  \]
  Let $H_1 = \{\id, \alpha, \beta, \gamma\}$. Then
  $\Phi(H_1) = \Q(i)$. Also
  $\Phi(\{\id, \lambda\}) = \Q(\sqrt[4]{2})$ and
  $\Phi(\{\id, \nu\}) = \Q(i\sqrt[4]{2})$. We can also
  see that $\Phi(\{\id, \mu\}) = \Q((1 + i)\sqrt[4]{2})$
  and $\Phi(\{\id, \rho\}) = \Q((1 - i)\sqrt[4]{2})$.
\end{example}

\begin{exercise}
  Write out the multiplication table for $\Gal(L : \Q)$.
\end{exercise}
