\chapter{Mar.~4 --- Solvable Groups}

\section{Solvable Groups}
\begin{definition}
  A finite group $G$ is \emph{solvable} if, for some
  $m \ge 0$, it has a finite series
  \[
    \{\id\} = G_0 \subseteq G_1 \subseteq \dots \subseteq G_m = G
  \]
  of subgroups such that for $i = 0, \dots, m - 1$,
  \begin{enumerate}
    \item $G_i \triangleleft G_{i + 1}$,
    \item and $G_{i + 1} / G_i$ is cyclic.
  \end{enumerate}
\end{definition}

\begin{remark}
  We require $G_i \triangleleft G_{i + 1}$, but
  $G_i$ need not be normal in $G$.
\end{remark}

\begin{example}
  Let $G = \Gal(\Q(i, \sqrt[4]{2}), \Q)$ from Example
  \ref{ex:big-galois-correspondence}. We have
  \[
    \{\id\} \subseteq \{\id, \lambda\} \subseteq \{\id, \beta, \lambda, \nu\} \subseteq G,
  \]
  where $G_i \triangleleft G_{i + 1}$ and
  $|G_{i + 1} / G_i| = 2$, so it is cyclic. Observe
  that $\{\id, \lambda\}$ is not normal in G, since
  \[
    \alpha \{\id, \lambda\} = \{\alpha, \mu\}
    \ne \{\alpha, \rho\} = \{\id, \lambda\} \alpha.
  \]
\end{example}

\begin{theorem}
  Every finite abelian group $G$ is solvable.
\end{theorem}

\begin{proof}
  Recall from the structure theorem for finitely generated
  abelian groups that every finite abelian group is
  a direct sum of cyclic groups. Then
  \[
    G = U_1 \oplus U_2 \oplus \dots \oplus U_k
  \]
  where each $U_i$ is cyclic. Let
  \[
    G_i = U_1 \oplus \dots \oplus U_i.
  \]
  Observe that $G_i \triangleleft G_{i + 1}$ since
  $G$ is abelian, and $G_{i + 1} / G_i \cong U_{i + 1}$,
  which is cyclic. So $G$ is solvable.
\end{proof}

\begin{remark}
  Recall that $S_n$ is the symmetric group on $n$
  elements.
\end{remark}

\begin{theorem}
  Every permutation can be expressed as a product of
  transpositions (i.e. 2-cycles).
\end{theorem}

\begin{definition}
  A permutation $\sigma$ is \emph{even} (respectively \emph{odd})
  if $\sigma$ can be expressed as a product of an
  \emph{even} (respectively \emph{odd}) number of transpositions.
  This is well defined. The set
  \[
    A_n = \text{subgroup of even permutations}
  \]
  is called the \emph{alternating group}.
\end{definition}

\begin{example}
  We have $S_3 = \{\id, (12), (23), (13), (123), (132)\}$.
  We can write
  \[
    \{\id\} \subseteq \{\id, (123), (132)\} \subseteq S_3.
  \]
  Call these $G_i$ for $i = 0, 1, 2$.
  Then $G_i \triangleleft G_{i + 1}$, and
  $G_2 / G_1 \cong \Z_2$ and $G_1 / G_0 = G_1 \cong \Z_3$.
  So $S_3$ is solvable.
\end{example}

\begin{example}
  The symmetric group $S_4$ is solvable. We can write
  \[
    \{\id\} \subseteq \{\id, (12)(34)\}
    \subseteq \{\id, (12)(34), (13)(24), (14)(23)\}
    \subseteq A_4 \subseteq S_4.
  \]
  Call the first three subgroups $G_i$ for $i = 0, 1, 2$.
  Then $G_i \triangleleft G_{i + 1}$, and we have
  \[
    S_4 / A_4 \cong \Z_2, \quad A_4 / G_2 \cong \Z_3,
    \quad G_2 / G_1 \cong \Z_2, \quad G_1 / G_0 \cong \Z_2.
  \]
\end{example}

\begin{exercise}
  Show that
  $G_2 = \{\id, (12)(34), (13)(24), (14)(23)\} \triangleleft A_4$.
\end{exercise}

\begin{definition}
  A group is \emph{simple} if it has no proper normal
  subgroups.
\end{definition}

\begin{remark}
  A non-abelian simple group is not solvable.
\end{remark}

\begin{theorem}
  For $n \ge 5$, the alternating group $A_n$ is simple.
\end{theorem}

\begin{proof}
  See Howie.
\end{proof}

\begin{theorem}
  We have the following:
  \begin{enumerate}
    \item If $G$ is solvable, then every subgroup
      of $G$ is solvable.
    \item If $G$ is solvable and $N \triangleleft G$,
      then $G / N$ is solvable.
    \item Let $N \triangleleft G$. Then
      $G$ is solvable if and only if
      $N$ and $G / N$ are solvable.
  \end{enumerate}
\end{theorem}

\begin{proof}
  (1) Since $G$ is solvable, there exists
  \[
    \{\id\} = G_0 \subseteq G_1 \subseteq \dots \subseteq G_m = G
  \]
  where $G_i \triangleleft G_{i + 1}$ and
  $G_{i + 1} / G_i$ is cyclic. Now let $H$ be a subgroup
  of $G$. Let $K_i = G_i \cap H$. Now check as an exercise
  that
  \[
    \{\id\} = K_0 \subseteq K_1 \subseteq \dots \subseteq K_m = H
  \]
  is the desired series of subgroups. In particular,
  check that
  $K_i \triangleleft K_{i + 1}$ and $K_{i + 1} / K_i$
  is cyclic (show that it is a subgroup of the cyclic
  group $G_{i + 1} / G_i$).

  (2) Take
  \[
    N / N = NG_0 / N \subseteq NG_1 / N \subseteq \dots NG_m / N = G / N
  \]
  as the desired series of subgroups. Here
  \[
    NG = \{ng \mid n \in N, g \in G\}.
  \]
  Check as an exercise that this construction works.
  For the cyclic part,
  verify that $(NG_{i + 1} / N) / (NG_i / N)$ is a
  quotient of the cyclic group $G_{i + 1} / G_i$.
  One of the isomorphism theorems may help here.

  (3) $(\Rightarrow)$ this follows from (1) and (2).

  $(\Leftarrow)$ Suppose $N$ and $G / N$ are solvable.
  Then there exists a series
  \[
    \{\id\} \subseteq N_0 \subseteq N_1 \subseteq \dots \subseteq N_p = N
  \]
  such that $N_i \triangleleft N_{i + 1}$ and
  $N_{i + 1} / N_i$ is cyclic, and a series
  \[
    \{\id\} = N / N = G_0 / N \subseteq G_1 / N
    \subseteq \dots \subseteq G_n / N = G / N
  \]
  such that $G_i / N \triangleleft G_{i + 1} / N$ and
  $(G_{i + 1} / N) / (G_i / N) \cong G_{i + 1} / G_i$
  by one of the isomorphism theorems, so it is cyclic as
  well. Now check as an exercise that
  \[
    \{\id\} = N_0 \subseteq N_1 \subseteq \dots \subseteq N_p = N
    = G_0 \subseteq G_1 \subseteq \dots \subseteq G_n = G
  \]
  is the desired series (i.e. check the normal and
  cyclic conditions).
\end{proof}

\begin{corollary}
  For $n \ge 5$, $S_n$ is not solvable.
\end{corollary}

\begin{proof}
  For $n \ge 5$, $A_n$ is simple, hence it is not
  solvable. Now if $S_n$ were solvable, all of its
  subgroups would be solvable, which leads to
  a contradiction since $A_n \subseteq S_n$.
\end{proof}

\section{Solvable Polynomials}

\begin{definition}
  A field extension $L : K$ is a \emph{radical extension}
  if there exists a sequence
  \[
    K = L_0 \subseteq L_1 \subseteq \dots \subseteq L_m = L
  \]
  such that $L_{j + 1} = L_j(\alpha_j)$, where
  $\alpha_j$ is a root of a polynomial in $L_j[X]$ of
  the form $X^{n_j} - c_j$.
\end{definition}

\begin{example}
  For $L_0 = \Q$, we can take
  \begin{align*}
    L_0 &= \Q, \\
    L_1 &= L_0(\alpha_0), \quad \quad \alpha_0^2 = 2, \\
    L_2 &= L_1(\alpha_1), \quad \quad \alpha_1^5 = 3 + \sqrt{2} \in L_1, \\
    L_3 &= L_2(\alpha_2), \quad \quad \alpha_2^2 = 2 + \sqrt[5]{3 + \sqrt{2}} \in L_2.
  \end{align*}
  This is a radical extension of $\Q$.
\end{example}

\begin{definition}
  A polynomial $f \in K[X]$ is \emph{solvable by radicals}
  if there is a splitting field for $f$ contained in a
  radical extension of $K$.
\end{definition}

\begin{example}
  Any quadratic $f = X^2 + bX + c \in \Q[X]$ is
  solvable by radicals, since its roots are
  \[
    \frac{-b \pm \sqrt{b^2 - 4c}}{2}.
  \]
\end{example}

\begin{remark}
  In the 16th and 17th centuries, mathematicians
  proved that cubics
  \[
    X^3 + a_2 X^2 + a_1 X + a_0
  \]
  and quartics
  \[
    X^4 + a_3 X^3 + a_2 X^2 + a_1 X + a_0
  \]
  are solvable by radicals. For cubics, the idea
  is to \emph{depress} the cubic, i.e. make a substitution
  to remove the quadratic term. Then we get
  \[
    Y^3 + 3aY + b = 0.
  \]
  By a lengthy algebra argument, the roots are
  \[
    q + r, \quad q\omega + r\omega^2, \quad q\omega^2 + r\omega,
  \]
  where
  \[
    q = \left(\frac{1}{2} (-b + \sqrt{b^2 + 4a^3})\right)^{1 / 3}
  \]
  and we have similar expressions for $r$ and $\omega$.
  A similar but longer algebraic manipulations can be
  made for quartics. In particular, the expressions for
  the roots of cubics and quartics only involve radicals.
\end{remark}

\begin{theorem}
  Let $L : K$ be a radical extension and $N$ the
  normal closure of $L$ over $K$. Then $N$ is also
  a radical extension of $K$.
\end{theorem}

\begin{proof}
  By Corollary \ref{cor:normal-closure}, we have
  \[
    N = L_1 \lor \dots \lor L_k,
  \]
  where each $L_i \cong L$, hence they are all radical.
  Now it suffices to show that the join of two radical
  extensions is radical. For this, let
  \[
    L_1 = K(\alpha_1, \dots, \alpha_m), \quad
    L_2 = K(\beta_1, \dots, \beta_n),
  \]
  where $\alpha_i^{k_i} \in K(\alpha_1, \dots, \alpha_{i - 1})$
  and $\beta_j^{l_j} \in K(\beta_1, \dots, \beta_{j - 1})$.
  Then
  \[
    L_1 \lor L_2 = K(\alpha_1, \dots, \alpha_m, \beta_1, \dots, \beta_n),
  \]
  where $\alpha_i^{k_i} \in K(\alpha_1, \dots, \alpha_{i - 1})$ and
  $\beta_j^{l_j} \in K(\alpha_1, \dots, \alpha_m, \beta_1, \dots, \beta_{j - 1})$,
  so $L_1 \lor L_2$ is radical.
\end{proof}

\begin{remark}
  Radical extensions involve polynomials of the form
  $X^m - c$. Let us look more closely at $X^m - 1$.
  We focus on fields $K$ of characteristic $0$, so
  that the splitting field $L$ of $X^m - 1$ over $K$
  is normal and separable.
\end{remark}

\begin{lemma}
  The set $R$ of roots of $X^m - 1$ is a cyclic group
  under multiplication.
\end{lemma}

\begin{proof}
  Check as an exercise that $R$ is indeed a subgroup of $L$.
  To see that it is cyclic, recall that
  \[
    \exp(R) = \text{smallest positive integer $e$ such that $a^e = 1$ for all $a \in R$}.
  \]
  Clearly we have $\exp(R) \le |R|$. Now observe that
  $x^e - 1$ has at
  most $e$ roots, so $|R| \le e$. Hence
  $e = |R| = m$, so $(R, \cdot)$ is cyclic.
\end{proof}

\begin{definition}
  A \emph{primitive $m$th root of unity} $\omega$ is a
  generator for $(R, \cdot)$.
\end{definition}

\begin{remark}
  We have
  \[
    R = \{1, \omega, \omega^2, \dots, \omega^{m - 1}\},
  \]
  and $\omega^i$ is a primitive $m$th root of unity if
  $\gcd(m, i) = 1$.
\end{remark}

\begin{definition}
  Let $P_m = \{\text{primitive $m$th roots of unity}\}$.
  The \emph{cyclotomic polynomial} $\Phi_m$ is
  \[
    \Phi_m = \prod_{\varepsilon \in P_m} (X - \varepsilon).
  \]
\end{definition}
