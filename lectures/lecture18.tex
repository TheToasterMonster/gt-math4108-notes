\chapter{Mar.~25 --- Modules}

\section{Introduction to Modules}
\begin{remark}
  Let $(G, +)$ be an abelian group. Recall that given
  $n \in \Z$, we defined
  \[
    ng =
    \begin{cases}
      g + \dots + g & \text{if } n > 0, \\
      0 & \text{if } n = 0, \\
      (-g) + \dots + (-g) & \text{if } n < 0,
    \end{cases}
  \]
  where we add $g$ or its inverse $n$ times. This
  gives a map $\Z \times G \to G$ by
  $(n, g) \mapsto ng$ that satisfies
  \begin{enumerate}
    \item $(n_1 n_2)g = n_1(n_2 g)$,
    \item $(n_1 + n_2) g = n_1 g + n_2 g$,
    \item and $n(g_1 + g_2) = ng_1 + ng_2$.
  \end{enumerate}
  From this, we would say that every abelian group $G$ is
  naturally a \emph{$\Z$-module}.
\end{remark}

\begin{definition}
  A \emph{module} $M$ over a ring $R$ is an abelian
  group $M$ together with a map $R \times M \to M$,
  called the \emph{product}, satisfying
  \begin{enumerate}
    \item $(r_1 r_2) m = r_1 (r_2 m)$,
    \item $(r_1 + r_2) m = r_1 m + r_2 m$,
    \item $r(m_1 + m_2) = rm_1 + rm_2$,
    \item and $1 m = m$.
  \end{enumerate}
\end{definition}

\begin{remark}
  For this class, we will only
  consider modules over commutative rings with unity.
\end{remark}

\begin{exercise}
  Verify the following:
  \begin{enumerate}
    \item $0m = 0$, $r 0 = 0$,
    \item $r(-m) = -(rm) = (-r)m$,
    \item $(-1)m = -m$.
  \end{enumerate}
\end{exercise}

\begin{example}
  A $K$-vector space is a module over $K$, where $K$
  is a field.
\end{example}

\begin{example}
  A ring $R$ is always a module over itself, where
  the product $R \times R \to R$ is the normal ring
  multiplication in $R$.
\end{example}

\begin{example}
  An ideal $I$ in a ring $R$ is a module over $R$.
  The product $R \times I \to I$ is given by
  $(r, m) \mapsto rm$, where $rm \in I$ since $I$ is
  an ideal.
\end{example}

\begin{example}
  The set $R^n = R \times \dots \times R$ is an
  $R$-module, where the product is given by
  \[
    r(r_1, r_2, \dots, r_n) =
    (rr_1, rr_2, \dots, rr_n).
  \]
\end{example}

\section{Submodules}
\begin{definition}
  A \emph{$R$-submodule} of an $R$-module $M$ is a
  subgroup $W$ of $M$ such that for all $r \in R$ and
  $w \in W$, we have $rw \in W$.
\end{definition}

\begin{example}
  Recall that $R$ is a module over itself. Then any
  ideal of $R$ is a submodule, and conversely, any
  submodule is an ideal.
\end{example}

\begin{prop}
  Let $M$ be an $R$-module.
  \begin{enumerate}
    \item If $\{M_\alpha\}$ is a collection of submodules
      of $M$, then $\bigcap_\alpha M_\alpha$ is also
      a submodule.
    \item If $M_1 \subseteq M_2 \subseteq \dots$ is an
      increasing sequence of submodules, then
      $\bigcup_n M_n$ is a submodule.
    \item If $A$ and $B$ are submodules of $M$, then
      $A + B = \{a + b \mid a \in A, b \in B\}$ is a
      submodule of $M$.
  \end{enumerate}
\end{prop}

\begin{proof}
  Left as an exercise.
\end{proof}

\begin{definition}
  Let $M$ be an $R$-module and $S$ a subset of $M$. The
  \emph{submodule of $M$ generated by $S$} is
  \[
    RS = \{r_1 s_1 + r_2 s_2 + \dots + r_n s_n \mid r_i \in R, s_i \in S, n \in \N\}
  .\]
\end{definition}

\begin{exercise}
  Verify that $RS$ is a submodule.
\end{exercise}

\begin{example}
  If $S = \{x\}$ for some $x \in M$, then
  $R\{x\}$ is the \emph{cyclic module} generated by $x$.
\end{example}

\begin{definition}
  If there exists $x \in M$ such that $M = R\{x\}$,
  then we say $M$ is \emph{cyclic}. If there exists a
  finite set $S \subseteq M$ such that $M = RS$, then
  $M$ is \emph{finitely generated}.
\end{definition}

\section{Module Homomorphisms}
\begin{definition}
  Let $M$ and $N$ be $R$-modules. Then an $R$-module
  \emph{homomorphism} $\varphi : M \to N$ is a
  homomorphism of abelian groups such that
  $\varphi(rm) = r\varphi(m)$ for all $r \in R$ and
  $m \in M$.
\end{definition}

\begin{definition}
  An $R$-module \emph{isomorphism} is a bijective
  $R$-module homomorphism. An $R$-module
  \emph{endomorphism}
  is an $R$-module homomorphism from $M$ to itself.
\end{definition}

\begin{remark}
  The set of all $R$-module homomorphisms from $M$ to $N$
  is denoted $\Hom_R(M, N)$, and the set of all
  $R$-module endomorphisms of $M$ is denoted
  $\End_R(M)$.
\end{remark}

\begin{definition}
  The \emph{kernel} of an $R$-module homomorphism
  $\varphi \in \Hom_R(M, N)$ is
  \[
    \ker \varphi = \{x \in M \mid \varphi(x) = 0\}.
  \]
\end{definition}

\begin{example}
  Let $M = R^m$ and $N = R^n$, thought of as column
  vectors. Let $T$ be a fixed $n \times m$ matrix with
  entries in $R$. Then left multiplication by $T$ is an
  $R$-module homomorphism from $M$ to $N$.
\end{example}

\section{Direct Sums of Modules}
\begin{definition}
  The \emph{direct sum} of $R$-modules $M_1, \dots, M_n$,
  denoted
  \[
    M_1 \oplus \dots \oplus M_n,
  \]
  is the product $M_1 \times \dots \times M_n$
  endowed with the operations
  \[
    (x_1, \dots, x_n) + (x_1', \dots, x_n') = (x_1 + x_1', \dots, x_n + x_n')
    \quad \text{and} \quad
    r(x_1, \dots, x_n) = (rx_1, \dots, rx_n).
  \]
\end{definition}

\begin{remark}
  Note that $M_i$ is naturally isomorphic to the
  following submodule of $M_1 \oplus \dots \oplus M_n$:
  \[
    \widetilde{M}_i = \{0\} \oplus \dots \oplus M_i \oplus \dots \oplus \{0\},
  \]
  and $M = \widetilde{M}_1 + \dots + \widetilde{M}_n = \{m_1 + \dots + m_n \mid m_i \in \widetilde{M}_i\}$.
\end{remark}

\begin{prop}
  Let $M$ be an $R$-module with submodules
  $A_1, \dots, A_s$ such that $M = A_1 + \dots + A_s$.
  Then the following are equivalent:
  \begin{enumerate}
    \item $(a_1, \dots, a_s) \mapsto a_1 + \dots + a_s$
      is a group isomorphism
      $A_1 \times \dots \times A_s \to M$.
    \item $(a_1, \dots, a_s) \mapsto a_1 + \dots + a_s$
      is an $R$-module isomorphism
      $A_1 \times \dots \times A_s \to M$.
    \item Each element $x \in M$ can be expressed as a
      sum
      \[
        x = a_1 + \dots + a_s
      \]
      with $a_i \in A_i$ is exactly one way.
    \item If $0 = a_1 + \dots + a_s$ with $a_i \in A_i$,
      then $a_i = 0$ for all $i$.
  \end{enumerate}
\end{prop}

\begin{proof}
  $(2) \Rightarrow (1)$ This is clear since an $R$-module
  isomorphism is also a group isomorphism.

  $(1) \Rightarrow (2)$ Let
  $\varphi : A_1 \times \dots \times A_s \to M$ be the
  given group isomorphism. Then
  \[
    \varphi(r(a_1, \dots, a_s))
    = \varphi(ra_1, \dots, ra_s)
    = ra_1 + \dots + ra_s
    = r(a_1 + \dots + a_s)
    = r\varphi(a_1, \dots, a_s),
  \]
  so $\varphi$ is also an $R$-module isomorphism.

  Now observe that $(1), (3), (4)$ say nothing about the
  module structure of $R$, so
  from here $(1) \Leftrightarrow (3) \Leftrightarrow (4)$
  is just an exercise in group theory.
\end{proof}

\begin{example}
  Let $M = \Z_6$ with $R = \Z$, and let
  $A_1 = \{0, 2, 4\}$ and $A_2 = \{0, 3\}$. Then
  the map $A_1 \oplus A_2 \to M$ given by
  \[
    (a_1, a_2) \mapsto a_1 + a_2.
  \]
  We can see that this is an isomorphism since
  $A_1 \cong \Z_3$ and $A_2 \cong \Z_2$.
\end{example}

\begin{definition}
  A subset $S \subseteq M$ is \emph{linearly independent}
  over $R$ if for any distinct $x_1, \dots, x_n \in S$,
  \[
    r_1 x_1 + \dots + r_n x_n = 0
  \]
  if and only if $r_i = 0$ for all $i$.
\end{definition}

\begin{definition}
  A \emph{basis} for $M$ is a linearly independent
  set $S$ with $RS = M$. An $R$-module $M$ is called
  \emph{free} if it has a basis.
\end{definition}

\begin{example}
  Every vector space over a field $K$ is free as a
  $K$-module.
\end{example}

\begin{example}
  Note that $\Z_n$ is not a free $\Z$-module since
  $na = 0$ for all $a \in \Z_n$. So $\{a\}$ for $a \ne 0$
  is in fact linearly dependent. More generally,
  any finite abelian group $G$ is not a free
  $\Z$-module.
\end{example}

\begin{example}
  However, $\Z$ is a free $\Z$-module. In general,
  $R^n$ is a free $R$-module. The \emph{standard basis}
  for $R^n$ is the set $\{e_1, \dots, e_n\}$ where
  \[
    e_1 = (1, 0, 0, \dots, 0),
    \quad \dots, \quad e_n = (0, 0, 0, \dots, 1).
  \]
\end{example}

\begin{definition}
  Let $M$ be an $R$-module and $B = \{x_1, \dots, x_n\}$
  be distinct nonzero elements in $M$. Then the
  following are equivalent:
  \begin{enumerate}
    \item $B$ is a basis for $M$.
    \item The map
      $\varphi : (r_1, \dots, r_n) \mapsto r_1 x_1 + \dots r_n m_n$
      is an $R$-module isomorphism from $R^n$ to $M$.
    \item For each $i$, the map $R \to M$ given by
      $r \mapsto rx_i$ is injective and
      $M = Rx_1 \oplus \dots \oplus Rx_n$.
  \end{enumerate}
\end{definition}

\begin{proof}
  $(1) \Leftrightarrow (2)$
  Observe that $B$ is linearly independent if
  and only if $\varphi$ is injective and $\mathcal{B}$
  spans $M$ if and only if $\varphi$ is surjective.
  Now check as an exercise that $\varphi$ is an
  $R$-module homomorphism.

  $(1) \Leftrightarrow (3)$ Left as an exercise.
\end{proof}

\begin{prop}
  We have the following:
  \begin{enumerate}
    \item If $\varphi \in \Hom_R(M, N)$, then
      $\ker \varphi$ is a submodule of $M$ and
      $\varphi(M)$ is a submodule of $N$.
    \item If $\varphi \in \Hom_R(M, N)$ and
      $\psi \in \Hom_R(N, P)$, then
      $\psi \circ \varphi \in \Hom_R(M, P)$.
  \end{enumerate}
\end{prop}

\begin{proof}
  (1) We need to show that if $m \in \ker \varphi$ and
  $r \in R$, then $rm \in \ker \varphi$. For this,
  observe that
  \[
    \varphi(rm) = r \varphi(m) = r 0 = 0
  \]
  since $m \in \ker \varphi$, so we have
  $rm \in \ker \varphi$. The rest of the proof is
  left as an exercise.
\end{proof}

\begin{prop}
  We have the following:
  \begin{enumerate}
    \item $\Hom_R(M, N)$ is an abelian group with the
      operation
      \[
        (\varphi + \psi)(m) = \varphi(m) + \psi(m).
      \]
    \item $\End_R(M)$ is a ring with addition as above
      and multiplication given by composition.
  \end{enumerate}
\end{prop}

\begin{proof}
  $(1)$ Clearly the addition is associative and
  commutative. The identity element is the zero map,
  and the inverse of $\varphi$ is $- \varphi$, i.e.
  if $\varphi : m \mapsto n$, then $- \varphi : m \mapsto -n$.
  
  $(2)$ Left as an exercise (the multiplicative
  identity is the identity map $\id_M$).
\end{proof}

\begin{remark}
  Many of the usual facts about group and ring
  homomorphisms have module analogues.
\end{remark}

\begin{prop}
  Let $M$ be an $R$-module and $N$ an $R$-submodule.
  Then the quotient group
  $M / N$ is an $R$-module and the quotient map
  $\pi : M \to M / N$ is an $R$-module homomorphism.
\end{prop}

\begin{proof}
  Define the product $R \times M / N \to M / N$ by
  \[
    r(m + N) = rm + N.
  \]
  To see that this product is well-defined, observe that
  if $m + N = m' + N$, then $m - m' \in N$. Hence
  \[
    rm - rm' = r(m - m') \in N
  \]
  since $N$ is an $R$-submodule. Thus
  $rm + N = rm' + N$ as desired. Now check as an exercise
  that this makes $M / N$ into an $R$-module.

  For the latter part about the quotient map
  $\pi : M \to M / N$, simply observe that
  \[
    \pi(rm) = rm + N = r(m + N) = r \pi(m),
  \]
  so indeed $\pi$ is an $R$-module homomorphism.
\end{proof}
