\chapter{Feb.~28 --- Join of Subgroups and Subfields}

\section{Join of Subgroups}

Let $H_1, H_2$ be subgroups of $G$.

\begin{exercise}
  Show that $H_1 \cap H_2$ is a subgroup of $G$.
\end{exercise}

\begin{remark}
  In general, $H_1 \cup H_2$ is \emph{not} a subgroup
  of $G$.
\end{remark}

\begin{definition}
  The \emph{join} of $H_1$ and $H_2$, denoted
  $H_1 \lor H_2$,
  is the smallest subgroup of $G$ containing
  $H_1 \cup H_2$, i.e. $H_1 \lor H_2$ consists
  of all products of the form
  \[
    a_1 b_1 \dots a_n b_n,
  \]
  where $a_i \in H_1$ and $b_i \in H_2$ for all $n$.
\end{definition}

\begin{remark}
  Recall that if $E_1$ and $E_2$ are subfields of
  $L$, then $E_1 \cap E_2$ is also a subfield of $L$,
  as is the join
  \[E_1 \lor E_2 = E_1(E_2) = E_2(E_1).\]
\end{remark}

\begin{example}
  In Example \ref{ex:big-galois-correspondence},
  we have $\{\id, \beta\} \lor \{\id, \lambda\} = \{\id, \beta, \lambda, \nu\}$.
  Now notice that
  \[
    \Phi(\{\id, \beta\}) = \Q(i, \sqrt{2}), \quad
    \Phi(\{\id, \lambda\}) = \Q(\sqrt[4]{2}), \quad
    \Phi(\{\id, \beta, \lambda, \nu\}) = \Q(\sqrt{2}).
  \]
  Notice that
  $\Q(i, \sqrt{2}) \cap \Q(\sqrt[4]{2}) = \Q(\sqrt{2})$.
\end{example}

\begin{theorem}
  Let $L : K$ be Galois and $E_1, E_2$ subfields
  of $L$ containing $K$. If
  \[
    \Gamma(E_1) = H_1, \quad \Gamma(E_2) = H_2,
  \]
  then $\Gamma(E_1 \cap E_2) = H_1 \lor H_2$ and
  $\Gamma(E_1 \lor E_2) = H_1 \cap H_2$.
\end{theorem}

\begin{proof}
  Certainly $E_1 \cap E_2 \subseteq E_1$, so
  $H_1 = \Gamma(E_1) \subseteq \Gamma(E_1 \cap E_2)$,
  since the Galois correspondence is order
  reversing. Similarly,
  $H_2 = \Gamma(E_2) \subseteq \Gamma(E_1 \cap E_2)$,
  so $H_1 \lor H_2 \subseteq \Gamma(E_1 \cap E_2)$.
  Now $H_1 \subseteq H_1 \lor H_2$, so we get
  $E_1 = \Phi(H_1) \supseteq \Phi(H_1 \lor H_2)$.
  Similarly, $E_2 = \Phi(H_2) \supseteq \Phi(H_1 \lor H_2)$,
  so $\Phi(H_1 \lor H_2) \subseteq E_1 \cap E_2$.
  Since $L : K$ is Galois, we get
  \[
    H_1 \lor H_2 \supseteq \Gamma(E_1 \cap E_2)
  \]
  by applying $\Gamma$ to both sides. So
  $\Gamma(E_1 \cap E_2) = H_1 \lor H_2$.

  The proof for $\Gamma(E_1 \lor E_2) = H_1 \cap H_2$
  is similar, see Howie for details.
\end{proof}
