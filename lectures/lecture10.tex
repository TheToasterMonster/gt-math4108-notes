\chapter{Feb.~12 --- Normal Closures}

\section{Normal Closures}

Recall this theorem from last time:
\begin{quote}
  \textbf{Theorem \ref{thm:normal-splitting}.}
  A finite extension $L : K$ is normal if and only
  if it is a splitting field for some polynomial
  in $K[X]$.
\end{quote}
A natural question to ask is: Can we always extend
a finite extension to make it normal?

\begin{definition}
  Let $L : K$ be a finite extension. A field $N$
  containing $L$ is a \emph{normal closure} of $L : K$ if
  \begin{enumerate}
    \item $N$ is a normal extension of $K$,
    \item and if $E$ is a proper subfield of $N$
      containing $L$, then $E$ is not a normal
      extension of $K$.
  \end{enumerate}
\end{definition}

\begin{theorem}
  Let $L : K$ be a finite extension. Then
  \begin{enumerate}
    \item there exists a normal closure $N$ of $L$
      over $K$,
    \item and $N$ is unique up to isomorphism.
  \end{enumerate}
\end{theorem}

\begin{proof}
  Let $\{z_1, \dots, z_n\}$ be a basis for $L : K$. Since
  $L : K$ is finite, each $z_i$ is algebraic over $K$,
  with say minimal polynomial $m_i \in K[X]$. Let
  \[
    m = m_1 \dots m_n,
  \]
  and let $N$ be the splitting field of $m$ over $L$.
  Then $N$ is also a splitting field of $m$ over $K$,
  since $L$ is generated over $K$ by some of
  the roots of $m$ in $N$. Hence $N$ is a normal
  extension of $K$ containing $L$.

  To see that $N$ is the smallest such field, suppose
  $E$ is a subfield of $N$ containing $L$, and suppose
  $E$ is normal. For each $m_i$, $E$ contains a root
  $z_i$, so the normality of $E$ implies that $E$ contains
  all the roots of $m$, so $E = N$. For uniqueness,
  see Howie. The proof relies on the uniqueness of
  splitting fields.
\end{proof}

\begin{definition}
  Let $K_1, \dots, K_n$ be subfields of $L$. The
  \emph{join} of $K_1, \dots, K_n$, denoted
  \[
    K_1 \lor K_2 \lor \dots \lor K_n,
  \]
  is the smallest subfield of $L$ containing
  $K_1 \cup K_2 \cup \dots \cup K_n$.
\end{definition}

\begin{remark}
  The smallest subfield of $L$ containing
  $K_1 \cup K_2$ is $K_1 \lor K_2 = K_1(K_2) = K_2(K_1)$,
  similar to how the smallest subfield of $\R$
  containing $\Q \cup \{\sqrt{3}\}$ is $\Q(\sqrt{3})$.
\end{remark}

\begin{example}
  Let $\Q(\sqrt[3]{2}), \Q(e^{2\pi i / 3} \cdot \sqrt[3]{2}) \subseteq \C$.
  Then
  $\Q(\sqrt[3]{2}) \lor \Q(e^{2\pi i / 3} \cdot \sqrt[3]{2}) = \Q(\sqrt[3]{2}, i\sqrt{3})$, since
  \[
    e^{2\pi i / 3} \cdot \sqrt[3]{2} = -\frac{\sqrt[3]{2}}{2} + \frac{i\sqrt{3}}{2} \sqrt[3]{2}.
  \]
\end{example}

\begin{remark}
  In the above example, we have
  $\Q(\sqrt[3]{2}) \cong \Q(e^{2\pi i / 3} \cdot \sqrt[3]{2}) \cong \Q[X] / \langle X^3 - 2 \rangle$.
\end{remark}

\begin{corollary}
  Let $L : K$ be a finite extension, and $N$ the normal
  closure of $L : K$. Then
  \[
    N = L_1 \lor L_2 \lor \dots \lor L_k,
  \]
  where $L_1, L_2, \dots, L_k$ are subfields of $N$
  containing $K$ isomorphic to $L$.
\end{corollary}

\begin{proof}
  As in the previous proof, suppose $\{z_1, \dots, z_n\}$
  is a basis for $L : K$, so $L = K(z_1, \dots, z_n)$,
  and $m_i$ is a minimal polynomial for $z_i$, and $N$
  a splitting field for $m = m_1 \dots m_n$ over $K$.
  Let $z_i'$ be an arbitrary root of $m_i$. Since
  $z_i$ and $z_i'$ are both roots of $m_i$, there exists
  a $K$-isomorphism $\varphi : K(z_i) \to K(z_i')$,
  which by Corollary \ref{thm:extend-automorphism} implies
  there exists a $K$-automorphism
  $\varphi^* : N \to N$. We have that
  \[
    z_i' \in \varphi^*(L) \cong L,
  \]
  so every root of $m_i$ is contained in a subfield
  $L' = \varphi^*(L)$ of $N$ that contains $K$ and is
  isomorphic to
  $L$, since $\varphi^*$ is a $K$-automorphism. Since
  $N$ is generated over $K$ by the roots of $m$, it is
  generated by finitely many subfields containing $K$
  and isomorphic to $L$.
\end{proof}

\begin{example}
  Find the normal closure of $\Q(\sqrt[3]{2})$ over $\Q$.
  Following the proof of the theorem,
  \[
    \{1, \sqrt[3]{2}, \sqrt[3]{2}^2\}
  \]
  is a basis of $\Q(\sqrt[3]{2}) : \Q$. The minimal
  polynomials of $1, \sqrt[3]{2}, \sqrt[3]{2}^2$ are
  $X - 1, X^3 - 2, X^3 - 4$, respectively. The
  splitting field of
  \[
    (X - 1)(X^3 - 2)(X^3 - 4)
  \]
  over $\Q$ is $\Q(\sqrt[3]{2}, i\sqrt{3})$, since
  \[
    X^3 - 2 = (X - \sqrt[3]{2})(X - e^{2\pi i / 3} \sqrt[3]{2})(X - e^{-2\pi i / 3} \sqrt[3]{2})
  \]
  and
  \[
    X^3 - 4 = (X - \sqrt[3]{2}^2)(X - e^{2\pi i / 3} \sqrt[3]{2}^2)(X - e^{-2\pi i / 3} \sqrt[3]{2}^2).
  \]
  So $\Q(\sqrt[3]{2}, i\sqrt{3}) = L_1 \lor L_2 \lor L_3$,
  where $L_1 = \Q(\sqrt[3]{2})$, $L_2 = \Q(e^{2\pi i / 3} \sqrt[3]{2})$,
  and $L_3 = \Q(e^{-2\pi i / 3} \sqrt[3]{2})$, and
  \[
    L_1 \cong L_2 \cong L_3 \cong \Q[X] / \langle X^3 - 2 \rangle.
  \]
\end{example}

\begin{theorem}
  Let $L : K$ be a finite normal extension and $E$ a
  subfield
  of $L$ containing $K$. Then $E$ is a normal extension
  of $K$ if and only if every $K$-monomorphism of
  $E$ into $L$ is a $K$-automorphism of $E$.
\end{theorem}

\begin{proof}
  $(\Rightarrow)$ Suppose $E : K$ is normal and
  let $\varphi : E \to L$ be a $K$-monomorphism. Now we
  would like to show that $\varphi(E) \subseteq E$. So let
  let $z \in E$ and suppose
  \[
    m = a_0 + a_1 X + \dots + a_n X^n
  \]
  is the minimal polynomial of $z$ over $K$. Then
  \[
    a_0 + a_1z + \dots + a_nz^n = 0,
  \]
  so that
  \[
    a_0 + a_1\varphi(z) + \dots + a_n\varphi(z)^n = 0
  \]
  since $\varphi$ is a homomorphism fixing $K$ pointwise.
  Hence $\varphi(z)$ is also a root of $m$ in $L$.
  Since $E : K$ is normal, the irreducible polynomial
  $m$ splits completely over $E$. Hence
  $\varphi(z) \in E$, so that $\varphi(E) \subseteq E$.
  Then\footnote{We need to make this argument since $E$ may be infinite, so injectivity does not imply bijectivity.}
  \[
    [\varphi(E) : K] = [\varphi(E) : \varphi(K)]
    = [E : K] = [E : \varphi(E)][\varphi(E) : K],
  \]
  so $[E : \varphi(E)] = 1$. Hence $\varphi(E) = E$,
  so $\varphi$ is a $K$-automorphism of $E$.

  $(\Leftarrow)$ Suppose every $K$-monomorphism
  $E \to L$ is a $K$-automorphism of $E$. Let $f$
  be an irreducible polynomial in $K[X]$ having a root
  $z \in E$. We need to show that $f$ splits completely
  over $E$. Since $L$ is normal, $f$ splits completely
  over $L$. Let $z'$ be another root of $f$ in $L$.
  Then there exists a $K$-automorphism
  $K(z) \to K(z')$ which sends $z \mapsto z'$,
  which by Corollary \ref{thm:extend-automorphism}
  extends to a $K$-automorphism $\psi$ of $L$. Let
  $\psi^* = \psi|_E$, i.e. the restriction of $\psi$
  to $E$. By hypothesis, $\psi^*$ is a $K$-automorphism
  of $E$, so
  \[z' = \psi(z) = \psi^*(z) \in E.\]
  That is, $E$ is normal.
\end{proof}

\begin{example}
  Consider $\Q(\sqrt[3]{2}) : \Q$, which is not normal.
  The $\Q$-monomorphism $\varphi : \Q(\sqrt[3]{2}) \to \C$
  given by
  \[
    \varphi(a + b\sqrt[3]{2} + c\sqrt[3]{2}^2) = a + be^{2\pi i / 3} \sqrt[3]{2} + ce^{-2\pi i / 3} \sqrt[3]{2}^2
  \]
  is not an automorphism of $\Q(\sqrt[3]{2})$.
\end{example}

\begin{example}
  Consider $\Q(\sqrt{2}) : \Q$, which is normal. The
  $\Q$-monomorphisms are $\id$ and
  \[\varphi(a + b\sqrt{2}) = a - b\sqrt{2},\]
  which are both $\Q$-automorphisms of $\Q(\sqrt{2})$.
\end{example}

\section{Separable Extensions}

\begin{definition}
  An irreducible polynomial $f \in K[X]$ is
  \emph{separable} over $K$ if it has no repeated roots
  over a splitting field. A polynomial $g \in K[X]$ is
  \emph{separable} over $K$ if its irreducible factors
  are separable over $K$. An algebraic element in $L : K$
  is \emph{separable} over $K$ if its minimal polynomial
  is separable over $K$. An algebraic extension $L : K$
  is \emph{separable} if every $\alpha \in L$ is separable
  over $K$.
\end{definition}

\begin{remark}
  A polynomial like $(X - 2)^2$ actually \emph{is}
  separable
  over $\Q$ since its irreducible factors are
  $X - 2$ and $X - 2$, which are each separable.
\end{remark}

\begin{definition}
  A field $K$ is \emph{perfect} if every polynomial
  in $K[X]$ is separable over $K$.
\end{definition}

\begin{theorem}
  We have the following:
  \begin{enumerate}
    \item Every field of characteristic $0$ is perfect.
    \item Every finite field is perfect.
  \end{enumerate}
\end{theorem}

\begin{proof}
  (1) It suffices to show that if $\Char K = 0$, then
  any irreducible polynomial $f$ is separable. Let
  \[
    f = a_0 + a_1 X + \dots + a_n X^n
  \]
  for $n \ge 1$ and suppose $f$ is not separable. Then
  $f$ and $Df$ have a non-constant common factor $d$.
  Since $f$ is irreducible, $d$ must be a constant
  multiple of $f$, and thus $d$ cannot divide $Df$
  unless
  \[
    Df = a_1 + 2a_2 X + \dots + na_n X^{n-1}
  \]
  is the zero polynomial, by comparing degrees. Then
  \[
    a_1 = 2a_2 = \dots = na_n = 0.
  \]
  Since $\Char K = 0$, this implies
  \[a_1 = a_2 = \dots = a_n = 0,\]
  and so $f = a_0$,
  a constant polynomial.\footnote{Recall that an irreducible polynomial is by definition a non-unit.}
  Contradiction. Hence $f$ is separable.

  (2) The same argument as above implies the only possible
  inseparable irreducible polynomials are of the form\footnote{We can still conclude $ka_k = 0$ implies $a_k = 0$ when $k$ is not a multiple of $p$.}
  \[
    f(X) = b_0 + b_1 X^p + b_2 X^{2p} \dots + b_{m} X^{mp}.
  \]
  Now Theorem 7.24 of Howie implies that if $K$ is finite,
  such a polynomial is reducible. Hence every irreducible
  polynomial is separable, so $K$ is perfect. See Howie
  for details.
\end{proof}

\begin{remark}
  Recall that $\Z_p(X)$ is an example of an infinite
  field with characteristic $p$.
\end{remark}
