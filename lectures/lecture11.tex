\chapter{Feb.~21 --- Galois Extensions}

\section{Example of an Inseparable Extension}

\begin{example}
  The field $K = \Z_p(X)$ is not perfect. Consider
  the polynomial
  \[
    f = Y^p - X \in \Z_p(X)[Y],
  \]
  which is irreducible. Now let $L$ be a splitting
  field of $f$ over $K$ and $\alpha$ a root of $f$,
  i.e. $\alpha^p - X = 0$. Then
  \[
    (Y - \alpha)^p = Y^p - \alpha^p = Y^p - X
  \]
  by freshman exponentiation. In particular,
  $\alpha$ is a repeated root of $f$ in $L$.
\end{example}

\section{Galois Extensions}

\begin{definition}
  A \emph{Galois extension} of $K$ is a finite extension
  that is both normal and separable.
\end{definition}

\begin{remark}
  The main goal here is: For a Galois extension,
  $\Gamma$ and $\Phi$ are inverses of one another.
\end{remark}

\begin{theorem}
  Let $L : K$ be a separable extension of degree
  $n$. Then there are exactly $n$ distinct
  $K$-monomorphisms of $L$ into a normal closure $N$
  of $L$ over $K$.
\end{theorem}

\begin{proof}
  Use strong induction on the degree of $L : K$.
  See Howie for details.
\end{proof}

\begin{corollary}
  \label{cor:galois-size}
  If $L : K$ is Galois, then
  $|\Gal(L : K)| = [L : K]$.
\end{corollary}

\begin{proof}
  If $L : K$ is Galois, then $L : K$ is normal
  and separable. So the previous theorem applies,
  where $L$ is its own normal closure. So we get exactly
  $[L : K]$ distinct $K$-monomorphisms of $L$ into $L$,
  which are precisely the $K$-automorphisms of $L$ and
  thus the elements of the Galois group.
\end{proof}

\begin{example}
  The extension $\Q(\sqrt[3]{2}, i\sqrt{3}) : \Q$ is
  Galois with $[\Q(\sqrt[3]{2}, i\sqrt{3}) : \Q] = 6$.
  We could have
  \[
    \sqrt[3]{2} \mapsto \sqrt[3]{2} \text{ or } e^{2\pi i / 3} \sqrt[3]{2} \text{ or } e^{-2\pi i / 3} \sqrt[3]{2}
    \quad \text{and} \quad
    i\sqrt{3} \mapsto i\sqrt{3} \text{ or } {-i\sqrt{3}}.
  \]
  Combinining these options gives us $6$ distinct
  maps, so these must in fact all be
  $\Q$-automorphisms of $\Q(\sqrt[3]{2}, i\sqrt{3})$,
  since we know the Galois group has size $6$.
  In fact, $\Gal(\Q(\sqrt[3]{2}, i\sqrt{3}) : \Q) \cong S_3 \cong D_3$.
\end{example}

\begin{remark}
  The proper nontrivial subfields of
  $\Q(\sqrt[3]{2}, i\sqrt{3})$ are
  $\Q(\sqrt[3]{2})$, $\Q(e^{2\pi i / 3}\sqrt[3]{2})$,
  $\Q(e^{-2\pi i / 3}\sqrt[3]{2})$, and
  $\Q(i\sqrt{3})$. Maybe draw a pretty diagram with this
  showing the Galois correspondence.
\end{remark}

\begin{exercise}
  Show that $\Z / 6\Z \cong \Z / 2\Z \times \Z / 3\Z$.
\end{exercise}

\begin{exercise}
  Show that $\Z / 4\Z \ncong \Z / 2\Z \times \Z / 2\Z$.
\end{exercise}

\begin{theorem}
  Let $L : K$ be a finite extension.
  Then $\Phi(\Gal(L : K)) = K$ if and only if
  $L : K$ is normal and separable.
\end{theorem}

\begin{proof}
  $(\Leftarrow)$ Let $[L : K] = n$. By Corollary
  \ref{cor:galois-size}, we have $|\Gal(L : K)| = n$.
  Let $K' = \Phi(\Gal(L : K))$. By definition,
  $K \subseteq K'$. By Theorem 7.12 of Howie, we find that
  \[
    [L : K'] = |\Gal(L : K)|.
  \]
  Hence $[L : K'] = [L : K]$ and thus we cocnlude
  that $K = K'$.

  $(\Rightarrow)$ See Howie.
\end{proof}

\begin{exercise}
  Show that if $K \subseteq K'$ and
  $[L : K'] = [L : K]$, then $K = K'$.
\end{exercise}

\begin{theorem}
  Let $L : K$ be Galois and $E$ a subfield of $L$
  containing $K$. If $\delta \in \Gal(L : K)$, then
  \[\Gamma(\delta(E)) = \delta \Gamma(E) \delta^{-1}.\]
\end{theorem}

\begin{proof}
  Next class, see Howie for now.
\end{proof}

\begin{example}
  Consider $\Q(\sqrt[3]{2}, i\sqrt{3}) : \Q$.
  Define the elements of $\Gal(\Q(\sqrt[3]{2}, i\sqrt{3}) : \Q)$ by
  \begin{gather*}
    \mu_1 : \sqrt[3]{2} \mapsto \sqrt[3]{2},\, i\sqrt{3} \mapsto -i\sqrt{3}, \quad
    \mu_2 : \sqrt[3]{2} \mapsto e^{2\pi i / 3} \sqrt[3]{2},\, i\sqrt{3} \mapsto -i\sqrt{3}, \\
    \mu_3 : \sqrt[3]{2} \mapsto e^{-2\pi i / 3}\sqrt[3]{2},\, i\sqrt{3} \mapsto -i\sqrt{3}, \\
    \rho_1 : \sqrt[3]{2} \mapsto e^{2\pi i / 3}\sqrt[3]{2},\, i\sqrt{3} \mapsto i\sqrt{3}, \quad
    \rho_2 : \sqrt[3]{2} \mapsto e^{-2\pi i / 3}\sqrt[3]{2},\, i\sqrt{3} \mapsto i\sqrt{3}.
  \end{gather*}
  Let $\delta = \mu_3$ and $E = \Q(\sqrt[3]{2})$.
  Then $\delta(E) = \Q(e^{-2\pi i / 3}\sqrt[3]{2})$ since
  $\mu_3(\sqrt[3]{2}) = e^{-2\pi i / 3}\sqrt[3]{2}$. Now
  \begin{align*}
    \mu_2(e^{-2\pi i / 3} \sqrt[3]{2})
    &= \mu_2(e^{-2\pi i / 3}) \mu_2(\sqrt[3]{2})
    = \mu_2(-\frac{1}{2} - i\frac{\sqrt{3}}{2}) \mu_2(\sqrt[3]{2}) \\
    &= (-\frac{1}{2} + i\frac{\sqrt{3}}{{2}})(e^{2\pi i / 3} \sqrt[3]{2})
    = e^{2\pi i / 3} e^{2\pi i / 3} \sqrt[3]{2}
    = e^{-2\pi i / 3} \sqrt[3]{2},
  \end{align*}
  so $\Gamma(\delta(E)) = \{\id, \mu_2\}$. Also
  $\Gamma(E) = \{\id, \mu_1\}$, and we find that
  \[
    \delta \Gamma(E) \delta^{-1}
    = \{\delta \id \delta^{-1}, \delta \mu_1 \delta^{-1}\}
    = \{\id, \mu_3 \mu_1 \mu_3^{-1}\} = \{\id, \mu_2\},
  \]
  so indeed we have $\Gamma(\delta(E)) = \delta \Gamma(E) \delta^{-1}$ in this case.
\end{example}
