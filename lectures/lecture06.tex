\chapter{Jan.~29 --- Geometric Constructions}

\section{\texorpdfstring{$K$-Isomorphisms}{K-Isomorphisms}}
Recall from last class that $L = K[X] / \langle m \rangle$
is a simple algebraic extension of $K$. In fact, we
can show that the field $L$ is essentially unique,
i.e. unique up to isomorphism.
\begin{theorem}
  \label{thm:isomorphism-fix}
  Let $K$ be a field and and $f$ and an irreducible
  polynomial in $K[X]$. If $L$ and $L'$ are two extensions
  of $K$ containing roots $\alpha$ and $\alpha'$
  respectively of $f$, then there exists an isomorphism
  $K[\alpha] \to K[\alpha']$ which fixes every element
  of $K$.
\end{theorem}

\begin{proof}[Proof sketch]
  Suppose
  \[
    f = a_0 + a_1 X + \dots + a_n X^n.
  \]
  Then $K[\alpha]$ consists of polynomials of the form
  \[
    b_0 + b_1 \alpha + \dots + b_{n-1} \alpha^{n-1}.
  \]
  This is because multiplication in $K[\alpha]$ relies on
  the observation that
  \[
    \alpha^n = -\frac{1}{\alpha_n} (a_0 + a_1 \alpha + \dots + a_{n-1} \alpha^{n-1})
  \]
  since $\alpha$ is a root of $f$. Define
  $\psi : K[\alpha] \to K[\alpha']$ by
  $\psi(g(a)) = g(\alpha')$ and show that $\psi$
  is an isomorphism.
\end{proof}

\begin{exercise}
  Check the following from the previous proof:
  \begin{enumerate}
    \item $\psi$ is one-to-one and onto,
    \item $\psi$ fixes $K$,
    \item and $\psi$ is a homomorphism.
  \end{enumerate}
  For the last point, the addition is mostly
  straightforward
  but the multiplication is more involved since
  we need to reduce when we get $\alpha^n$ terms
  in the product.
\end{exercise}

\begin{definition}
  A \emph{$K$-isomorphism} is an isomorphism
  $\varphi : L \to L'$
  such that $\varphi(x) = x$ for all $x \in K$.
\end{definition}

\begin{example}
  For $\C : \R$, the complex conjugation map
  $\varphi : \C \to \C$ given by
  $\varphi(a + bi) = a - bi$ is a $\R$-isomorphism.
\end{example}

\begin{example}
  For $\Q[X] / \langle X^2 + 3 \rangle : \Q$,\footnote{Note that $\Q[X] / \langle X^2 + 3 \rangle \cong \Q[i\sqrt{3}]$. The isomorphism is given by $a + bX + \langle X^3 + 3 \rangle \mapsto a + bi\sqrt{3}$.}
  the map $\psi : \Q[X] / \langle X^2 + 3 \rangle \to \Q[X] / \langle X^2 + 3 \rangle$ given by
  \[
    \psi(a + bX + \langle X^2 + 3 \rangle) = a - bX + \langle X^2 + 3 \rangle
  \]
  is a $\Q$-isomorphism. The analogous
  map $\psi : \Q[i\sqrt{3}] \to \Q[i\sqrt{3}]$
  given by $\psi(a + bi\sqrt{3}) = a - bi\sqrt{3}$
  also works, which we can view as a restriction
  of the complex conjugation map to $\Q[i\sqrt{3}]$.
\end{example}

\section{Applications to Geometric Constructions}
Consider the straightedge and compass Constructions
from geometry. Let $B_0$ be a set of points. Then
we have the following operations:
\begin{enumerate}
  \item (straightedge) Draw a straight line through any
    two points in $B_0$.
  \item (compass) Draw a circle whose center is a
    point in $B_0$ passing through another point in $B_0$.
\end{enumerate}
Let $C(B_0)$ be the set of points which are intersections
of lines or circles obtained form $B_0$ by (1) and (2).
Let $B_1 = B_0 \cup C(B_0)$, and proceed
inductively to get $B_n = B_{n-1} \cup C(B_{n-1})$.

\begin{definition}
  A point is \emph{constructible from $B_0$} if it belongs
  to $B_n$ for some $n$. A point is \emph{constructible}
  if it is constructible from $\{O, I\}$ where
  $O = (0, 0)$ and $I = (1, 0)$.
\end{definition}

\begin{example}
  To find the midpoint of the line segment $OI$
  from $B_0 = \{O, I\}$, we can do the following:
  \begin{enumerate}
    \item Draw a circle with center $O$ passing
      through $I$.
    \item Draw a circle with center $I$ passing through
      $O$.
    \item Mark points $P$ and $Q$ where these circles
      intersect. So $B_1 \supseteq \{O, I, P, Q\}$.
    \item Draw a line connecting $P$ and $Q$.
    \item Draw a line connecting $O$ and $I$.
    \item Mark the point $M$ where $PQ$ and $OI$ meet.
      So $B_2 \supseteq \{O, I, P, Q, M\}$.
  \end{enumerate}
  Thus $M$ is constructible from $\{O, I\}$.
\end{example}

The algebraic perspective is the following: Associate
to $B_i$ the subfield of $\R$ generated by coordinates
of points in $B_i$, i.e. view each coordinate of
each point as an element and take the subfield
generated.
\begin{example}
For
$B_0 = \{(0, 0), (1, 0)\}$, we have $\{0, 0, 1, 0\} \subseteq K_0 = \Q$
is the subfield of $\R$ generated by the coordinates
of $B_0$. Next take\footnote{There is some abuse of notation here since we take $B_i$ to be only some subset of all the actual possible points.}
\[
  B_1 = \{O, I, P, Q\}
  = \{(0, 0), (1, 0), (1 / 2, \pm \sqrt{3} / 2)\},
\]
so that $K_1 = \Q[\sqrt{3}]$ is the field
generated by $B_1$. Then
\[
  B_2 = \{O, I, P, Q, M\} = \{(0, 0), (1, 0), (1 / 2, \pm \sqrt{3} / 2), (1 / 2, 0)\},
\]
and the field generated by $B_2$ is still $K_2 = \Q[\sqrt{3}]$.
\end{example}

\begin{theorem}
  Let $P$ be a constructible point belonging to $B_n$,
  where $B_0 = \{(0, 0), (1, 0)\}$, and let $K_n$
  be the field generated over $\Q$ by $B_n$. Then
  $[K_n : \Q]$ is a power of $2$.
\end{theorem}

\begin{proof}[Proof sketch]
  We proceed by induction. The base case is
  $K_0 = \Q$, so $[K_0 : \Q] = 1 = 2^0$. Now suppose
  $[K_{n - 1} : \Q] = 2^k$ for some $k \ge 0$, and
  we want to show that $[K_n : K_{n - 1}]$ is a power
  of $2$. Observe that new points in $B_n$ can be
  obtained by
  \begin{enumerate}
    \item intersection of two lines,
    \item intersection of a line and a circle,
    \item or intersection of two circles.
  \end{enumerate}
  In case (1), the intersection of two lines is
  given by solving a system of two linear equations,
  which only involves rational operations\footnote{By rational operations we mean addition, subtraction, multiplication, division.}.
  In other words, this case takes place entirely in
  $K_{n - 1}$.

  In case (2), the intersection of a line
  and a circle is given by solving of a system of one
  linear equation and one quadratic equation. Solving
  the linear equation for one of the variables and
  substituting into the quadratic equation reduces
  the system down to a single quadratic equation
  in a single variable. The solution involves
  $\sqrt{\Delta}$, where $\Delta$ is the discriminant.
  Then the new points are in $K_{n - 1}[\sqrt{\Delta}]$.

  In case (3), the intersection of two circles is
  given by solving a system of two quadratic equations.
  Subtracting the two quadratic equations yields a
  linear equation, which reduces back to case (2).

  Thus the elements in $K_n$ are either in $K_{n - 1}$
  or $K_{n - 1}[\sqrt{\Delta}]$ for some
  $\Delta \in K_{n - 1}$.\footnote{We can set it up so that we only gain one extra intersection, i.e. only one $\Delta$, at each step.} Hence $[K_n : K_{n - 1}]$
  is either $1$ or $2$, so by induction $[K_n : \Q]$ is
  a power of $2$.
\end{proof}

\section{Classic Problems}
\subsection{Duplicating the Cube}
Consider the problem of taking a cube of volume $1$, and
constructing a cube of volume $2$. We need $\alpha$
such that $\alpha^3 = 2$. But $X^3 - 2$ is irreducible
over $\Q$ by Eisenstein's criterion, so
$[\Q[\alpha] : \Q] = 3$. This is not a power of $2$,
so $\alpha$ is not constructible and thus we cannot
duplicate the cube.

\subsection{Trisecting the Angle}
Recall the triple angle formula:
\[\cos 3\theta = 4\cos^3 \theta - 3\cos \theta.\]
Suppose $\cos 3\theta = c$. So to find $\cos \theta$,
we want a root of $4X^3 - 3X - c = 0$. This depends on
$c$.

\begin{example}
If $3\theta = \pi / 2$, then $c = 0$ and
the polynomial factors into
\[4X^3 - 3X = 4X(4X^2 - 3),\]
so $[\Q[\alpha] : \Q] = [\Q[\sqrt{3}] : \Q] = 2$.
So in fact we can trisect $\pi / 2 = 90^\circ$.
\end{example}

\begin{example}
  If $3\theta = \pi / 3$, then $c = 1 / 2$ and
  we have $4X^3 - 3X - 1 / 2$. Let
  \[
    f(X) = 8X^3 - 6X - 1,
  \]
  so that $g(X) = g(X / 2) = X^3 - 3X - 1$. Note
  that $g$ does not factor over $\Z$ since that
  requires a linear factor of $X \pm 1$ but
  $g(\pm 1) \ne 0$. So $g$ is irreducible over $\Z$
  and by Gauss's lemma, $g$ is irreducible over $\Q$.
  Thus $f$ is irreducible. Hence
  $[\Q[\alpha] : \Q] = 3$, so we cannot
  trisect $\pi / 3$ with a straightedge and compass.
\end{example}
