\chapter{Feb.~7 --- The Galois Correspondence}

\section{Automorphisms of Fields}

\begin{example}
  The complex conjugation $\beta : \C \to \C$ given
  by $\beta(a + bi) = a - bi$ is a nontrivial element
  of the Galois group of $\C : \R$. In fact,
  $\Gal(\C : \R) = \{\id, \beta\}$. Note that
  $\beta$ fixes $\R$, $\id$ fixes $\C$, and
  \[
  \begin{tikzcd}
    \C \\
    \R \uar
  \end{tikzcd}
  \]
\end{example}

\section{The Galois Correspondence}

\begin{definition}
  Define
  \begin{align*}
    \Gamma(E) &= \{
      \alpha \in \Aut L : \alpha(z) = z \text{ for all } z \in E
    \}, \\
    \Phi(H) &= \{
    x \in L : \alpha(x) = x \text{ for all } \alpha \in H
    \},
  \end{align*}
  where $E$ is a subfield of $L$ and $H$ is a subgroup
  of $\Gal(L : K)$. This is called the
  \emph{Galois correspondence}.
\end{definition}

\begin{example}
  In the previous example of $\C : \R$, we have
  $\Gamma(\C) = \{\id\}$ and $\Gamma(\R) = \{\id, \beta\}$.
  We also have $\Phi(\{\id, \beta\}) = \R$ and
  $\Phi(\{\id\}) = \C$.
\end{example}

\begin{remark}
  The goal is to determine: When are $\Gamma$ and
  $\Phi$ inverses of one another?
\end{remark}

\begin{theorem}
  We have the following:
  \begin{enumerate}
    \item For every subfield $E$ of $L$ containing $K$,
      $\Gamma(E)$ is a subgroup of $\Gal(L : K)$.
    \item Conversely, for every subgroup $H$
      of $\Gal(L : K)$, $\Phi(H)$ is a subfield of $L$
      containing $K$.
  \end{enumerate}
\end{theorem}

\begin{proof}
  See Howie.
\end{proof}

\begin{theorem}
  Let $z \in L \setminus K$. If $z$ is a root of
  $f \in K[X]$ and $\alpha \in \Gal(L : K)$, then
  $\alpha(z)$ is also a root of $f$.
\end{theorem}

\begin{proof}
  Let $f = a_0 + a_1X + \cdots + a_nX^n$, where
  $a_i \in K$. Then since $\alpha$ fixes each
  $a_i \in K$, we have
  \begin{align*}
    f(\alpha(z))
    = a_0 + a_1\alpha(z) + \cdots + a_n(\alpha(z))^n
    &= \alpha(a_0) + \alpha(a_1)\alpha(z) + \cdots + \alpha(a_n)(\alpha(z))^n \\
    &= \alpha(a_0 + a_1z + \cdots + a_nz^n)
    = \alpha(0) = 0,
  \end{align*}
  which completes the proof.
\end{proof}

\begin{example}
  Recall this example from homework:
  \[
  \begin{tikzcd}
    & L = \Q(\sqrt{3}, \sqrt{5}) \\
    \Q(\sqrt{3}) \urar & \Q(\sqrt{5}) \uar & \Q(\sqrt{15}) \ular \\
    & K = \Q \ular \uar \urar
  \end{tikzcd}
  \]
  A basis for $L$ over $K$ is $\{1, \sqrt{3}, \sqrt{5}, \sqrt{15}\}$.
  Since $\sqrt{3}$ is a root of $X^2 - 3$,
  by the previous theorem, any element in $\Gal(L : K)$
  must send $\sqrt{3} \mapsto \pm \sqrt{3}$. Similarly,
  any element must send
  $\sqrt{5} \mapsto \pm \sqrt{5}$. So
  the $\Q$-isomorphisms of $\Q(\sqrt{3}, \sqrt{5})$
  are
  \begin{align*}
    \alpha(a + b\sqrt{3} + c\sqrt{5} + d\sqrt{15})
    &= a - b\sqrt{3} + c\sqrt{5} - d\sqrt{15}, \\
    \beta(a + b\sqrt{3} + c\sqrt{5} + d\sqrt{15})
    &= a + b\sqrt{3} - c\sqrt{5} - d\sqrt{15}, \\
    \gamma(a + b\sqrt{3} + c\sqrt{5} + d\sqrt{15})
    &= a - b\sqrt{3} - c\sqrt{5} + d\sqrt{15}, \\
    \id(a + b\sqrt{3} + c\sqrt{5} + d\sqrt{15})
    &= a + b\sqrt{3} + c\sqrt{5} + d\sqrt{15}.
  \end{align*}
  We can write the multiplication table for this
  group as:
  \begin{center}
    \begin{tabular}{c|cccc}
      $\times$ & $\id$ & $\alpha$ & $\beta$ & $\gamma$ \\
      \hline
      $\id$ & $\id$ & $\alpha$ & $\beta$ & $\gamma$ \\
      $\alpha$ & $\alpha$ & $\id$ & $\gamma$ & $\beta$ \\
      $\beta$ & $\beta$ & $\gamma$ & $\id$ & $\alpha$ \\
      $\gamma$ & $\gamma$ & $\beta$ & $\alpha$ & $\id$
    \end{tabular}
  \end{center}
  The proper subgroups are
  $H_1 = \{\id, \alpha\}$, $H_2 = \{\id, \beta\}$, and
  $H_3 = \{\id, \gamma\}$. Also
  $\{id\}$ and $G = \{\id, \alpha, \beta, \gamma\}$
  are subgroups. Then
  \begin{gather*}
    \Phi(H_1) = \Q(\sqrt{5}), \quad
    \Phi(H_2) = \Q(\sqrt{3}), \quad
    \Phi(H_3) = \Q(\sqrt{15}), \\
    \Phi(\{\id\}) = \Q(\sqrt{3}, \sqrt{5}), \quad
    \Phi(G) = \Q.
  \end{gather*}
  Under $\Phi$, this gives the diagram:
  \[
  \begin{tikzcd}
    & G \\
    H_1 \urar & H_2 \uar & H_3 \ular \\
    & \{\id\} \ular \uar \urar
  \end{tikzcd} \quad
  \longrightarrow \quad
  \begin{tikzcd}
    & \Phi(G) = \Q \drar \dar \dlar \\
    \Q(\sqrt{3}) \drar & \Q(\sqrt{5}) \dar & \Q(\sqrt{15}) \dlar \\
    & \Phi(\{\id\}) = \Q(\sqrt{3}, \sqrt{5})
  \end{tikzcd}
  \]
  Also note that
  $\Gamma(\Q(\sqrt{3})) = \{\id, \alpha\}$ since
  \[
    \alpha(a + b\sqrt{3} + c\sqrt{5} + d\sqrt{15})
    = a - b\sqrt{3} + c\sqrt{5} - d\sqrt{15}.
  \]
\end{example}

\begin{exercise}
  Show that $\Gamma$ is the inverse of $\Phi$ in the
  previous example.
\end{exercise}

\begin{theorem}
  Let $L : K$ be a field extension. Then
  \begin{enumerate}
    \item If $E_1, E_2$ are two subfields of $L$
      containing $K$, then
      $E_1 \subseteq E_2$ implies
      $\Gamma(E_1) \supseteq \Gamma(E_2)$.
    \item If $H_1, H_2$ are subgroups of $\Gal(L : K)$,
      then $H_1 \subseteq H_2$ implies
      $\Phi(H_1) \supseteq \Phi(H_2)$.
  \end{enumerate}
\end{theorem}

\begin{proof}
  (1) Suppose $E_1 \subseteq E_2$ and $\alpha \in \Gamma(E_2)$.
  Then $\alpha$ fixes every element in $E_2$, so since
  $E_1 \subseteq E_2$, $\alpha$ also fixes every element
  in $E_1$. Hence $\alpha \in \Gamma(E_1)$ by
  definition.

  (2) Suppose $H_1 \subseteq H_2$ and let $z \in \Phi(H_2)$.
  Then $\alpha(z) = z$ for every $\alpha \in H_2$, and
  since $H_1 \subseteq H_2$, $\alpha(z) = z$ for every
  $\alpha \in H_1$ as well. Hence $z \in \Phi(H_1)$ by
  definition.
\end{proof}

\begin{remark}
  Note that $\Gamma$ and $\Phi$ are not always
  inverses of one another.
\end{remark}

\begin{example}
  Consider the extension $\Q(\sqrt[3]{2}) : \Q$.
  If $\alpha \in \Gal(\Q(\sqrt[3]{2}) : \Q)$, then
  \[
    \alpha(\sqrt[3]{2})^3 = \alpha(2) = 2.
  \]
  Since there is only one cube root of $2$ in this
  field, we must have $\alpha(\sqrt[3]{2}) = \sqrt[3]{2}$.
  So $\Gal(\Q(\sqrt[3]{2}) : \Q) = \{\id\}$. So
  $\Gamma$ cannot be the inverse of $\Phi$ here since
  there are two subfields, namely $\Q(\sqrt[3]{2})$
  and $\Q$. In particular,
  \[
    \Gamma(\Q(\sqrt[3]{2})) = \Gamma(\Q) = \{\id\}
    \quad \text{and} \quad
    \Phi(\{\id\}) = \Q(\sqrt[3]{2}).
  \]
\end{example}

\begin{theorem}
  For any subfield $E$ of $L$ and subgroup $H$ of $\Gal(L : K)$, we have
  \begin{enumerate}
    \item $E \subseteq \Phi(\Gamma(E))$
    \item and $H \subseteq \Gamma(\Phi(H))$.
  \end{enumerate}
\end{theorem}

\begin{proof}
  (1) Let $z \in E$. Then $\Gamma(E)$ is the set of
  all automorphisms fixing every element of $E$, and so
  $z$ is fixed by every element of $\Gamma(E)$.
  Hence $z \in \Phi(\Gamma(E))$.

  (2) Let $\alpha \in H$. Then $\Phi(H)$ is the set
  of elements of $L$ fixed by every element of $H$, and
  so $\alpha$ fixes every element of $\Phi(H)$.
  Hence $\alpha \in \Gamma(\Phi(H))$.
\end{proof}

\begin{remark}
  Now the goal will be to find sufficient conditions
  for $\Gamma$ and $\Phi$ to be inverses of one another.
\end{remark}

\section{Normal Extensions}

\begin{definition}
  A field extension $L : K$ is \emph{normal} if every
  irreducible polynomial in $K[X]$ having at least
  one root in $L$ splits completely over $L$.
\end{definition}

\begin{example}
  An nonexample is $\Q(\sqrt[3]{2}) : \Q$. This is not
  a normal extension since $X^3 - 2$ is irreducible
  and has a root in $\Q(\sqrt[3]{2})$, but does not
  split completely over $\Q(\sqrt[3]{2})$.
\end{example}

\begin{remark}
  Is $\Q(\sqrt{2}) : \Q$ normal?
\end{remark}

\begin{theorem}
  A finite extension $L : K$ is normal if and only if
  it is a splitting field for some polynomial in $K[X]$.
\end{theorem}

\begin{proof}
  $(\Rightarrow)$ Let $L$ be a finite normal extension
  and $\{z_1, \dots, z_n\}$ be a basis for $L : K$.
  let $m_i$ be the minimum polynomial for $z_i$, and
  let
  \[
    m = m_1 m_2 \dots m_n.
  \]
  Each $m_i$ has at least one root $z_i$ in $L$, hence
  $m$ splits completely over $L$ since $L$ is normal.
  Since $L$ is generated by $z_1, \dots, z_n$, it is
  not possible for $m$ to split over a proper subfield
  of $L$, hence $L$ is a splitting field for $m$ over $K$.

  $(\Leftarrow)$ See Howie. Relies on the isomorphism
  $K(\alpha) \to K(\beta)$ for $\alpha, \beta$ roots
  of an irreducible polynomial $f$. We also need
  properties of degrees of field extensions.
\end{proof}

\begin{corollary}
  Let $L$ be a normal extension of $K$ and $E$ a
  subfield of $L$ containing $K$. Then every injective
  $K$-homomorphism $\varphi : E \to L$ can be extended to
  a $K$-automorphism $\varphi^*$ of $L$.
\[
  \begin{tikzcd}
    E \dar[hook, swap, "i"] \rar{\varphi} & L \\
    L \urar[dashed, "\varphi^*"'] &
  \end{tikzcd}
\]
\end{corollary}

\begin{proof}
  By the theorem, there exists $f \in K[X]$ such that
  $L$ is a splitting field for $f$ over $K$. But
  $L$ is also a splitting field for $f$ over $E$ and
  $\varphi(E)$. From here, a slight generalization of the
  proof of uniqueness of splitting fields gives
  the desired $K$-automorphism of $L$ extending $\varphi$.
\end{proof}

\begin{example}
  Let $L = \Q(\sqrt{3}, \sqrt{5})$, $K = \Q$, and
  $E = \Q(\sqrt{3})$. Define $\varphi : E \to L$
  by
  \[
    \varphi(a + b\sqrt{3}) = a - b\sqrt{3},
  \]
  which is an injective $K$-homomorphism. We have the
  following diagram:
  \[
  \begin{tikzcd}
    \Q(\sqrt{3}) \dar[hook, swap, "i"] \rar{\varphi} & \Q(\sqrt{3}, \sqrt{5}) \\
    \Q(\sqrt{3}, \sqrt{5}) \urar[dashed, "\varphi^*"'] &
  \end{tikzcd}
  \]
  Then we can define
  \[
    \varphi^*(a + b\sqrt{3} + c\sqrt{5} + d\sqrt{15})
    = a - b\sqrt{3} + c\sqrt{5} - d\sqrt{15}
  \]
  as an extension of $\varphi$. Note that we could have
  also defined
  \[
    \varphi^*(a + b\sqrt{3} + c\sqrt{5} + d\sqrt{15})
    = a - b\sqrt{3} - c\sqrt{5} + d\sqrt{15}.
  \]
\end{example}

\begin{remark}
  From the previous example we see that
  $\varphi^*$ is not unique.
\end{remark}
