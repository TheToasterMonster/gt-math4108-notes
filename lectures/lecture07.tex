\chapter{Jan.~31 --- Splitting Fields}

\section{Review of Notation}
Recall that
\begin{align*}
  \Q[X] &= \{
    a_0 + a_1 X + \dotsb + a_n X^n : a_i \in \Q
  \} \\
    \Q(X) &= \{
      f / g : f, g \in \Q[X], g \ne 0
    \} / \sim,
\end{align*}
where $\sim$ is the usual relation on fractions, e.g.
$2f / 2g = f / g$. Next, recall that
\[
  \Q[\sqrt{2}] = \{a_0 + a_1 \sqrt{2} + \dots + a_n \sqrt{2}^n : a_i \in \Q\}
  = \{a + b \sqrt{2} : a, b \in \Q\}
\]
since $\sqrt{2}^2 = 2$. Also
$\Q(\sqrt{2})$ is the smallest subfield of $\R$
containing $\Q \cup \{\sqrt{2}\}$. In this case,
$\Q(\sqrt{2}) = \Q[\sqrt{2}]$ since
\[
  \frac{1}{a + b\sqrt{2}} = \frac{a - b\sqrt{2}}{a^2 - 2b^2}.
\]
Next, we have
\begin{align*}
  \Q[X] / \langle X^2 - 2 \rangle
  &= \{
    a_0 + a_1 X + \dots + a_n X^n + \langle X^2 - 2 \rangle : a_i \in \Q
  \} \\
  &= \{
    a + bX + \langle X^2 - 2 \rangle : a, b \in \Q
  \}
\end{align*}
since $X^2 + \langle X^2 - 2 \rangle = 2 + \langle X^2 - 2\rangle$.
In fact, $\Q[X] / \langle X^2 - 2 \rangle \cong \Q[\sqrt{2}]$.\footnote{Here the isomorphism $\Q[X] / \langle X^2 - 2 \rangle \to \Q[\sqrt{2}]$ is given by $a + bX  +\langle X^2 - 2 \rangle \mapsto a + b\sqrt{2}$.}

\section{Splitting Fields}
The motivating question here is: When can we factor
a polynomial into linear factors?

\begin{definition}
  A polynomial \emph{splits completely} over $K$
  if it can be factored into linear factors over $K$.
\end{definition}

\begin{example}
  The polynomial $X^2 + 2$ splits completely over
  $\Q[i\sqrt{2}]$ since $X^2 + 2 = (X - i\sqrt{2})(X + i\sqrt{2})$.
\end{example}

\begin{example}
  The polynomial $X^3 - 2$ is irreducible
  over $\Q$ by Eisenstein's criterion. However,
  it factors as
  \[
    X^3 - 2 = (X - \alpha) (X^2 + \alpha X + \alpha^2)
  \]
  in $\Q[\alpha]$, where $\alpha = \sqrt[3]{2}$.
  Also $X^2 + \alpha X + \alpha^2$ is irreducible
  over $\Q[\alpha]$, since its discriminant shows that
  it is irreducible even over $\R$. But in $\C$, we
  can factor it as
  \[
    X^3 - 2 = (X - \alpha)(X - \alpha e^{2\pi i / 3})(X - \alpha e^{4\pi i / 3}).
  \]
  A smaller field that $X^3 - 2$ splits completely over
  is $\Q[\sqrt[3]{2}, i \sqrt{3}]$.
\end{example}

\begin{definition}
  Let $K$ be a field and $f \in K[X]$. An extension
  $L$ of $K$ is a \emph{splitting field} for $f$ over $K$
  if
  \begin{enumerate}
    \item $f$ splits completely over $L$,
    \item and $f$ does not split completely over any
      subfield $E$ with $K < E < L$.
  \end{enumerate}
\end{definition}

\begin{example}
  From the last two examples, $\Q[i\sqrt{2}]$ is a
  splitting field over $\Q$ for $X^2 + 2$, and $\Q[\sqrt[3]{2}, i\sqrt{3}]$
  is a splitting field for $X^3 - 2$ over $\Q$.
\end{example}

\begin{theorem}
  Let $K$ be a field and $f \in K[X]$ with
  $\partial f = n$. Then there exists a splitting field
  $L$ for $f$ over $K$ and $[L : K] \le n!$.
\end{theorem}

\begin{proof}
The proof is essentially the process we perform in the
following example. At each step, construct an extension
in which we can split off a linear factor from $f$.
For more details, see Howie.
\end{proof}

\begin{example}
  Let us find a splitting field for
  \[f = X^5 + X^4 - X^3 - 3X^2 - 3X + 3\]
  over $\Q$. Note that $\partial f = n$.
  Stare hard enough and we can see that
  \[
    f = (X^3 - 3)(X^2 + X - 1),
  \]
  where the first factor is irreducible by Eisenstein's criterion and
  the second factor is irreducible by checking the
  discriminant. Now add a root, say $\alpha = \sqrt[3]{3}$,
  and let $E_1 = \Q(\alpha)$. Then
  \[
    f = (X - \alpha) (X^2 + \alpha X + \alpha^2)(X^2 + X - 1).
  \]
  Note that $[E_1 : K] \le n = \partial f$.
  Now let $E_2 = E_1(\alpha e^{2\pi i / 3})$, so that
  \[
    f = (X - \alpha) (X - \alpha e^{2\pi i / 3})
    (X - \alpha e^{-2\pi i / 3})(X^2 + X - 1).
  \]
  Note that $[E_2 : \Q] \le n(n - 1)$. Next
  $E_3 = E_2(\alpha e^{- 2\pi i / 3})$ with
  \[
    f = (X - \alpha) (X - \alpha e^{2\pi i / 3})
    (X - \alpha e^{-2\pi i / 3})(X^2 + X - 1).
  \]
  Note that $[E_3 : K] \le n(n - 1)(n - 2)$. Now let
  \[
    \gamma = \frac{-1 + \sqrt{5}}{2}, \quad \delta = \frac{-1 - \sqrt{5}}{2}.
  \]
  Let $E_4 = E_3(\gamma)$,
  \[
    f = (X - \alpha) (X - \alpha e^{2\pi i / 3})
    (X - \alpha e^{-2\pi i / 3})(X - \gamma)(X - \delta).
  \]
  Finally $E_5 = E_4(\delta)$ is the splitting field
  for $f$ over $\Q$. Note that we did much better than
  $n!$ here, since
  \[
    [E_1 : \Q] = 3, \quad [E_2 : E_1] = 2, \quad
    [E_3 : E_2] = 1, \quad [E_4 : E_3] = 2, \quad
    [E_5 : E_4] = 1,
  \]
  so $[E_5 : \Q] = 12 \le 120$.
\end{example}

\begin{remark}
  Splitting fields are unique (up to isomorphism).
\end{remark}

\begin{theorem}
  Let $L$ and $L'$ be splitting fields of $f$ over $K$.
  Then there exists an isomorphism
  $\varphi : L \to L'$ fixing $K$.
\end{theorem}

\begin{proof}[Proof sketch]
  Induct on the number of roots of $f$ that are not
  in $K$. The induction step uses Theorem \ref{thm:isomorphism-fix}
  from last
  class giving an isomorphism $K[\alpha] \to K[\alpha']$
  for $\alpha, \alpha'$ roots of an irreducible
  polynomial.
\end{proof}

\begin{example}
  Let us find the splitting field of $f = X^4 - 2$ over
  $\Q$ and its degree. Note that $X^4 - 2$ is
  irreducible over $\Q$ by Eisenstein's criterion.
  Note that
  \[
    X^4 - 2 = (X - \alpha)(X + \alpha)(X - i \alpha)(X + i\alpha)
  \]
  where $\alpha = \sqrt[4]{2}$. So the
  splitting field is $\Q(\sqrt[4]{2}, i)$. For the
  degree, note that $[\Q(\sqrt[4]{2}) : \Q] = 4$ since
  the minimal polynomial of $\sqrt[4]{2}$ is $X^4 - 2$.
  A basis for this extension is $\{1, \sqrt[4]{2}, \sqrt[4]{2}^2, \sqrt[4]{2}^3\}$.
  Since $i \notin \Q(\sqrt[4]{2})$, we have
  $[\Q(\sqrt[4]{2}, i) : \Q(\sqrt[4]{2})] = 2$ since
  the minimal polynomial of $i$ over $\Q(\sqrt[4]{2})$
  is $X^2 + 1$. Thus we see that the degree of the
  splitting field
  is $[\Q(\sqrt[4]{2}, i) : \Q] = 8$.
\end{example}

\begin{example}
  Let us look at monic quadratic polynomials
  over $\Z_3 = \{-1, 0, 1\}$.\footnote{Note that as opposite to $\Q$, this field has finite characteristic.}
  These are
  \[
  \begin{array}{ccccc}
    X^2 & & X^2 + 1 & & X^2 - 1 \\
    X^2 + X & & X^2 + X + 1 & & X^2 + X - 1 \\
    X^2 - X & & X^2 - X + 1 & & X^2 - X - 1.
  \end{array}
  \]
  We have $0$ is a root of the polynomials in the
  first column, $1$ is a root of $X^2 - 1$ and
  $X^2 + X + 1$, and $-1$ is a root of $X^2 - X + 1$.
  So the irreducible polynomials over $\Z_3$ are
  \[
    X^2 + 1, \quad X^2 + X - 1, \quad X^2 - X - 1.
  \]
  Let $L = \Z_3[X] / \langle X^2 + 1 \rangle$. Observe
  that $\alpha = X + \langle X^2 + 1 \rangle$ satisfies
  \[
    \alpha^2 = X^2 + \langle X^2 + 1 \rangle
    = -1 + \langle X^2 + 1 \rangle.
  \]
  Hence $L$ is a splitting field for $X^2 + 1$ since
  $(X - \alpha)(X + \alpha) = X^2 + 1$.
  Similarly,
  $\Z_3[X] / \langle X^2 + X - 1 \rangle$ is a
  splitting field for $X^2 + X - 1$ and
  $\Z_3[X] / \langle X^2 - X - 1 \rangle$ is a
  splitting field for $X^2 - X - 1$. Note that each of
  these fields have $9 = 3^2$ elements since they are
  degree $2$ extensions of $\Z_3$.
\end{example}

\begin{remark}
  In $L$, we had $\alpha \in L$ such that
  $\alpha^2 = - 1$ and addition is performed modulo $3$.
  Now observe
  \[
    (\alpha + 1)^2 + (\alpha + 1) - 1
    = (\alpha^2 - \alpha + 1) + (\alpha + 1) - 1
    = \alpha^2 - \alpha + \alpha + 1 + 1 - 1
    = 0
  \]
  since $\alpha^2 = -1$. So $\alpha + 1$ is a root of
  $X^2 + X - 1$ in $L$. By a similar computation, we
  see that $- \alpha + 1$ is a root of $X^2 + X - 1$,
  so $L$ is also a splitting field for $X^2 + X - 1$.
  Additionally, $\alpha - 1$ and $-\alpha - 1$ are
  roots of $X^2 - X - 1$, so $L$ is also a splitting
  field for $X^2 - X - 1$. So by uniqueness of splitting
  fields,
  \[
    \Z_3[X] / \langle X^2 + 1 \rangle
    \cong \Z_3[X] / \langle X^2 + X - 1 \rangle
    \cong \Z_3[X] / \langle X^2 - X - 1 \rangle.
  \]
\end{remark}

\begin{exercise}
  Find explicit isomorphisms between these
  fields.
\end{exercise}

\section{Finite Fields}
\begin{definition}
Let $f = a_0 + a_1 X + \dots + a_n X^n \in K[X]$. Then
the \emph{formal derivative} of $f$ is
\[
  Df = a_1 + 2a_2 X + \dots + n a_n X^{n-1}.
\]
\end{definition}

\begin{exercise}
  The usual formulas for derivatives
  \[
    D(kf) = k Df, \quad D(f + g) = Df + Dg, \quad
    D(fg) = (Df)g + f(Dg)
  \]
  all still hold for $f, g \in K[X]$ and $k \in K$.
\end{exercise}
