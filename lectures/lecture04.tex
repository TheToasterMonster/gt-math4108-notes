\chapter{Jan.~22 --- Field Extensions}
\section{More on Irreducibility}
The following excerpt is from Howie:
\begin{quote}
  Another device for determining irreducibility over $\Z$ (and consequently over $\Q$) is to map the polynomial onto $\Z_p[X]$ for some suitably chosen prime $p$. Let $g = a_0 +a_1X + \dots +a_nX^n \in \Z[X]$, and let $p$ be a prime not dividing $a_n$. For each $i$ in $\{0, 1, \dots, n\}$, let $\overline{a}_i$ denote the residue class $a_i + \langle p \rangle$ in the field $\Z_p = \Z / \langle p \rangle$, and write the polynomial $\overline{a}_0 + \overline{a}_1X + \dots + \overline{a}_n X^n$ as $\overline{g}$. Our choice of $p$ ensures that $\partial \overline{g} = n$. Suppose that $g = uv$, with $\partial u, \partial v < \partial f$ and $\partial u + \partial v = \partial g$. Then $\overline{g} = \overline{u}\,\overline{v}$. If we can show that $\overline{g}$ is irreducible in $\Z_p[X]$, then we have a contradiction, and we deduce that $g$ is irreducible. The advantage of transferring the problem from $\Z[X]$ to $\Z_p[X]$ is that $\Z_p$ is finite, and the verification of irreducibility is a matter of checking a finite number of cases.
\end{quote}

\begin{example}
  Show that
  \[g = 7X^4 + 10X^3 - 2X^2 + 4X - 5\]
  is irreducible over $\Q$.
\end{example}

\begin{proof}
  Let $p = 3$ and
  \[\overline{g} = X^4 + X^3 + X^2 + 1\]
  This has no linear factors since
  \[\overline{g}(0) = 1, \quad \overline{g}(1) = 2, \quad \overline{g}(-1) = 1.\]
  So suppose
  \[
    \overline{g} = X^4 + X^3 + X^2 + X + 1
    = (X^2 + aX + b)(X^2 + cX + d)
  \]
  in $\Z_3[x]$. Then for some
  $a, b, c, d \in \Z_3 = \{-1, 0, 1\}$, we have
  \[
    \begin{cases}
      X^3 & a + c = 1 \\
      X^2 & b + ac + d = 1 \\
      X & ad + bc = 1 \\
      1 & bd = 1
    \end{cases}
  \]
  The first case is if $b = d = 1$, but this
  implies $ac = -1$, so $a = \pm 1$ and $c = \mp 1$.
  But $a + c = 1$, so this cannot happen. The
  second case is if $b = d = -1$. This implies
  that $ac = 0$ and $a + c = 1$. So if $a = 0$,
  then $c = 1$, so $1 = ad + bc = b$, which is
  a contradiction with $b = -1$. If $c = 0$,
  then $1 = ad + bc = d$, which is a contradiction
  with $d = -1$. Thus $\overline{g}$ is irreducible
  in $\Z_3[x]$, so $g$ is irreducible in $\Z[x]$,
  and by Gauss's lemma, $g$ is irreducible in $\Q[x]$.
\end{proof}

\begin{remark}
  If we had tried $p = 2$, then we have
  $\overline{g} = x^4 + 1 \in \Z_2[x]$, which is
  not in fact irreducible since
  \[\overline{g} = x^4 + 1 = (x + 1)^4 \in \Z_2[x].\]
\end{remark}

\section{Field Extensions}
\begin{definition}
  Let $K, L$ be fields and $\varphi : K \to L$ an
  injective homomorphism. Then $L$ is a
  \emph{field extension} of $K$, denoted $L : K$.
\end{definition}

\begin{example}
  We have $\C : \R$ is a field extension.
\end{example}

\begin{definition}
Recall that $V$ is a \emph{$K$-vector space} if
\begin{enumerate}
  \item $V$ is an abelian group under $+$,
  \item For $a, b \in K$ and $x, y \in V$, we have
    \[\text{(i). } a(x + y) = ax + ay, \quad \text{(ii). }(a + b)x = ax + bx, \quad \text{(iii). } (ab)x = a(bx), \quad \text{(iv). } 1x = 1.\]
\end{enumerate}
\end{definition}

\begin{remark}
  If $L : K$ is a field extension, then $L$ is a
  a vector space over $K$.
\end{remark}

\begin{definition}
  A \emph{basis} for a vector space is a linearly
  independent spanning set.
\end{definition}

\begin{example}
  The complex numbers $\C$ is a $\R$-vector space
  with basis $\{1, i\}$. Bases are not unique,
  since $\{1 + i, 1 - i\}$ is another basis for $\C$.
\end{example}

\begin{example}
  If there is a vector space that we know to be a field,
  then it is automatically a field extension of its
  ground field.
\end{example}

\begin{definition}
  The \emph{dimension} of $L$
  is the cardinality of a bsis for $L : K$.\footnote{Note that this is well-defined since any two bases of $L$ have the same length.}
  The dimension is also called the \emph{degree} of
  $L : K$, denoted $[L : K]$. We say that $L$ is a
  \emph{finite extension} if $[L : K]$ is finite, and
  an \emph{infinite extension} otherwise.
\end{definition}

\begin{example}
  We have $[\C : \R] = 2$, which is finite. On the
  other hand, $\R : \Q$ is an infinite extension.
\end{example}

\begin{theorem}
  Let $L : K$ be a field extension. Then
  $L = K$ if and only if $[L : K] = 1$.
\end{theorem}

\begin{proof}
  $(\Rightarrow)$ If $L = K$, then $\{1\}$ is a basis
  for $L : K$, and thus $[L : K] = 1$.

  $(\Leftarrow)$ If $[L : K] = 1$, then $\{x\}$ is a
  basis for $L : K$ for some $x \in L$. Then there exists
  some $a \in K$ such that $1 = ax$, so
  $x = a^{-1} \in K$. For every $y \in L$, there exists
  $b \in K$ such that $y = bx$. But then
  \[
    y = bx = b(a^{-1}) \in K,
  \]
  so $y \in K$ as well by closure. Thus $L = K$
  as desired.
\end{proof}

\begin{remark}
Let $L : K$ and $M : L$ be field extensions with
\begin{center}
  \begin{tikzcd}
    K \arrow[r, "\alpha"] & L \arrow[r, "\beta"] & M
  \end{tikzcd}
\end{center}
Then $M : K$ is also a field extension.
\end{remark}

\begin{theorem}
  For field extensions $L : K$ and $M : L$, we have
  $[M : L][L : K] = [M : K]$.
\end{theorem}

\begin{proof}
  Suppose $\{a_1, a_2, \dots a_r\}$ is a linearly
  independent subset of $M$ over $L$ and
  $\{b_1, b_2, \dots, b_s\}$ is a linearly
  independent subset of $L$ over $K$.
  Now we claim that
  \[
    \{a_i b_j \mid 1 \le i \le r, 1 \le j \le s\}
  \]
  is a linearly independent subset of $M$ over $K$.
  To see this, suppose
  \[
    \sum_{i = 1}^r \sum_{j = 1}^s \lambda_{ij} a_i b_i = 0
  \]
  for some $\lambda_{ij} \in K$. We can rewrite this as
  \[
    \sum_{i = 1}^r \left(\sum_{j = 1}^s \lambda_{ij} b_j\right) a_i = 0.
  \]
  Since the $a_i$ are linearly independent over $L$,
  it follows that
  \[
    \sum_{j = 1}^s \lambda_{ij} b_j = 0
  \]
  for each $i = 1, \dots, r$. Since the $b_j$ are
  linearly independent over $K$, it follows that
  $\lambda_{ij} = 0$ for each $i, j$, which proves
  the claim. Returning to the main proof,
  if $[M : L]$ or $[L : K]$ is infinite,
  then $r$ or $s$ can be made arbitrarily large,
  so
  \[
    \{a_i b_j \mid 1 \le i \le r, 1 \le j \le s\}
  \]
  can also be made arbitrarily large, and hence
  $[M : K]$ is infinite. Now suppose
  $[M : L] = r < \infty$ and $[L : K] = s < \infty$. Let
  $\{a_1, a_2, \dots, a_r\}$ be a basis for $M : L$ and
  $\{b_1, b_2, \dots, b_s\}$ be a basis for $L : K$.
  We will show that
  \[
    \{a_i b_j \mid 1 \le i \le r, 1 \le j \le s\}
  \]
  is a basis for $M : K$. Since we already showed that
  $\{a_i b_j\}$ is linearly independent, it only remains
  to show that they span $M$ over $K$. For each $z \in M$,
  there exist $\lambda_1, \dots, \lambda_r \in L$ such
  that
  \[z = \sum_{i = 1}^r \lambda_i a_i.\]
  Then for each $\lambda_i \in L$, there exist
  $\mu_{i 1}, \dots, \mu_{i s} \in K$ such that
  \[
    \lambda_i = \sum_{j = 1}^s \mu_{i j} b_j.
  \]
  Combining this yields
  \[
    z = \sum_{i = 1}^r \sum_{j = 1}^s \mu_{i j} a_i b_j
  \]
  as desired, which finishes the proof.
\end{proof}

\begin{example}
  Consider $\Q(\sqrt{2}) = \Q[\sqrt{2}] = \{a + b\sqrt{2} \mid a, b \in \Q\}$.
\end{example}

\begin{exercise}
  Show that $\Q[\sqrt{2}]$ is a field.
  (Hint: $1 / (a + b\sqrt{2}) = (a - b\sqrt{2}) / (a^2 - 2b^2)$.)
\end{exercise}

\begin{definition}
  Let $K$ be a subfield of $L$ and $S$ a subset of
  $L$. The \emph{subfield of $L$ generated over $K$ by $S$},
  denoted $K(S)$, is the intersection of all subfields
  of $L$ containing $K \cup S$. If
  $S = \{\alpha_1, \dots, \alpha_n\}$ is finite,
  we write $K(\alpha_1, \dots, \alpha_n)$.
\end{definition}

\begin{theorem}
  Let $E$ be the elements in $L$ that can be expressed
  as quotients of finite $K$-linear combinations of
  finite products of elements in $S$. Then $K(S) = E$.
\end{theorem}

\begin{proof}
  To see that $K(S) \subseteq E$, simply check that $E$
  is a subfield of $L$ containing $K \cup S$.

  For $E \subseteq K(S)$, note that any subfield of $L$
  containing $K$ and $S$ must contain all finite products
  of elements in $S$, all linear combinations of
  such products, and all quotients of such linear
  combinations. This is precisely what is means
  to have $E \subseteq K(S)$.
\end{proof}

\begin{definition}
  A \emph{simple extension} of $K$ is $K(\alpha)$,
  i.e. $S$ has a single element $\alpha \notin K$.
\end{definition}

\begin{example}
  The previous example $\Q(\sqrt{2})$ is a simple
  extension.
\end{example}

\begin{theorem}
  Let $L$ be a field, $K$ a subfield, and $\alpha \in L$.
  Then either
  \begin{enumerate}
    \item $K(\alpha)$ is isomorphic to $K(X)$, the
      field of rational forms with coefficients in $K$,
    \item or there exists a unique monic polynomial
      $m \in K[X]$ with the property that for all
      $f \in K[X]$,
      \begin{enumerate}
        \item $f(\alpha) = 0$ if and only if $m | f$,
        \item the field $K(\alpha)$ coincides with
          $K[\alpha]$, the ring of all polynomials
          in $\alpha$ with coefficients in $K$,
        \item and $[K[\alpha] : K] = \partial m$.
      \end{enumerate}
  \end{enumerate}
\end{theorem}

\begin{proof}
  Suppose there does not exist nonzero $f \in K[X]$ such
  that $f(\alpha) = 0$. Then there exists a map
  $\varphi : K(X) \to K(\alpha)$ with
  $f / g \mapsto f(\alpha) / g(\alpha)$, which is
  defined since $g(\alpha) = 0$ only if $g$ is the
  zero polynomial. Note that $\varphi$ is a
  surjective homomorphism,\footnote{Also check that $\varphi$ is well-defined.} which one can check as
  an exercise. Now we show that $\varphi$ is also
  injective. To see this, suppose
  \[
    \varphi(f / g) = \varphi(p / q),
  \]
  which happens if and only if
  \[
    f(\alpha) q(\alpha) - p(\alpha) g(\alpha) = 0.
  \]
  in $L$. This happens if and only if $fq - pg = 0$
  in $K[X]$, which happens if and only if $f / g = p / q$
  in $K(X)$. This completes the first case of the theorem.

  Now suppose there exists nonzero $g \in K[X]$ such
  that $g(\alpha) = 0$. Furthermore, suppose $g$ is a
  polynomial of least degree with this property. Let
  $a$ be the leading coefficient of $g$, and let
  $m = g / a$, so that $m$ is monic and $m(\alpha) = 0$
  still. The reverse implication in (2a) is clear. For
  the forwards implication in (2a), note that by division
  with remainder for polynomials over a field, we can
  write
  \[
    f = qm + r,
  \]
  where $\partial r < \partial m$. By the minimality
  of $\partial m$, we must have $r = 0$, so $m | f$.
  For the uniqueness of $m$, suppose there exists $m'$
  with the same properties. Then
  $m(\alpha) = m'(\alpha) = 0$, so
  $m | m'$ and $m' | m$, which implies that $m = m'$
  since $m$ and $m'$ are monic. For the irreducibility
  of $m$, suppose for the sake of contradiction that
  $m = pq$ with $\partial p, \partial q < \partial m$.
  Then $m(\alpha) = p(\alpha) q(\alpha) = 0$, so
  either $p(\alpha) = 0$ or $q(\alpha) = 0$, which
  contradicts the minimality of $\partial m$.

  Now we show (2b), which says that
  $K(\alpha) = K[\alpha]$. For this, consider
  $p(\alpha) / q(\alpha) \in K(\alpha)$ for
  $q(\alpha) \ne 0$. Then $m {\nmid} q$, and since
  $m$ is irreducible we have $\gcd(m, q) = 1$. Now
  by Theorem 2.15 of Howie (about gcd's in the
  Euclidean domain $K[X]$), there exist polynomials
  $a, b$ such that $aq + bm = 1$. Setting
  $X = \alpha$ yields $a(\alpha) q(\alpha) = 1$, so
  \[
    \frac{p(\alpha)}{q(\alpha)} = p(\alpha) a(\alpha) \in K[\alpha].
  \]
  Thus $K(\alpha) \subseteq K[\alpha]$. Since we already
  know that $K[\alpha] \subseteq K(\alpha)$, we conclude
  that $K(\alpha) = K[\alpha]$.

  Finally we show (2c), which claims that
  $[K[\alpha] : K] = \partial m$. For this, suppose
  $\partial m = n$ and let
  \[
    p(\alpha) \in K[\alpha] = K(\alpha).
  \]
  Then $p = qm + r$ where $\partial r < \partial m = n$.
  We have $p(\alpha) = r(\alpha)$, so if
  \[
    r = c_0 + c_1 X + \dots + c_{n - 1} X^{n - 1}
  \]
  for $c_i \in K$, then
  \[
    p(\alpha) = c_0 + c_1 \alpha + \dots + c_{n - 1} \alpha^{n - 1}.
  \]
  So $\{1, \alpha, \dots, \alpha^{n - 1}\}$ is a spanning
  set for $K[\alpha]$. To see that
  $\{1, \alpha, \dots, \alpha^{n - 1}\}$ is also
  linearly independent, suppose there exists $a_i \in K$
  such that
  \[
    a_0 + a_1 \alpha + \dots + a_{n - 1} \alpha^{n - 1} = 0.
  \]
  Then $a_0 = \dots = a_{n - 1} = 0$ since otherwise
  we would have a polynomial
  \[
    p = a_0 + a_1 X + \dots + a_{n - 1} X^{n - 1}
  \]
  with $\partial p \le n - 1$ and $p(\alpha) = 0$,
  which is a contradiction with the minimality of
  $\partial m = n$. Thus
  $\{1, \alpha, \dots, \alpha^{n - 1}\}$ is a basis, and
  so $[K[\alpha] : K] = n = \partial m$.
\end{proof}

\begin{example}
  Continuing the same example, note that
  \[
    \Q[\sqrt{2}] = \{a + b\sqrt{2} \mid a, b \in \Q\}
    = \{a_0 + a_1 \sqrt{2} + a_2 \sqrt{2}^2 + a_3 \sqrt{2}^3 + \dots + a_n \sqrt{2}^n \mid a_i \in \Q\},
  \]
  which falls in the second case of the previous theorem.
\end{example}

\begin{remark}
  We also have $\Q[\sqrt{2}] = \Q[X] / \langle X^2 - 2 \rangle$.
\end{remark}
