\chapter{Jan.~8 --- Rings and Fields}

\section{Lots of Definitions}
Recall the definitions of a ring and a field:
\begin{definition}[Ring]
  A \emph{ring} $R = (R, +, \cdot)$ is a non-empty set
  $R$ together with two binary operations
  $+$ and $\cdot$, called addition and multiplication
  respectively, which satisfy:
  \begin{enumerate}
    \item[(R1)] \textit{Associative law for addition}:
      $(a + b) + c = a + (b + c)$ for all $a, b, c \in R$.
    \item[(R2)] \textit{Commutative law for addition}:
      $a + b = b + a$ for all $a, b \in R$.
    \item[(R3)] \textit{Existence of zero}: There exists $0 \in R$ such that
      $a + 0 = a$ for all $a \in R$.
    \item[(R4)] \textit{Existence of additive inverses}:
      For all $a \in R$, there exists $-a \in R$ such that
      $a + (-a) = 0$.\footnote{Note that we'll usually write $a - b$ in place of $a + (-b)$.}
    \item[(R5)] \textit{Associative law for multiplication}:
      $(ab)c = a(bc)$ for all $a, b, c \in R$.
    \item[(R6)] \textit{Distributive laws}:
      $a(b + c) = ab + ac$ and $(a + b)c = ac + bc$ for all $a, b, c \in R$.
  \end{enumerate}
\end{definition}

\begin{definition}[Commutative ring]
In this class, we will mostly be interested in
\emph{commutative rings}, which satisfy the following
additional property for multiplication:
\begin{enumerate}
  \item[(R7)] \textit{Commutative law for multiplication}:
    $ab = ba$ for all $a, b \in R$.
\end{enumerate}
\end{definition}

\begin{definition}[Ring with unity]
A ring \emph{with unity} satisfies the additional
property that
\begin{enumerate}
  \item[(R8)] \emph{Existence of unity}: There exists
    $1 \ne 0 \in R$ such that and
    $a 1 = 1 a = a$ for $a \in R$.
\end{enumerate}
\end{definition}
Note that a ring need not be commutative to have a unity.

\begin{definition}[Domain]
A commutative ring with unity is called a
\emph{(integral) domain} if it has the following
cancellation property:
\begin{enumerate}
  \item[(R9)] \emph{Cancellation}: For all
    $a, b \in R$ and $c \ne 0$, $ca = cb$ implies
    $a = b$.
  \item[(R9')] \textit{No zero divisors}: For all
    $a, b \in R$, $ab = 0$ implies $a = 0$ or $b = 0$.
\end{enumerate}
\end{definition}

The conditions (R9) and (R9') are equivalent.

\begin{definition}[Field]
A commutative ring with unity is called a \emph{field}
if it has the following additional property for
multiplicative inverses:
\begin{enumerate}
  \item[(R10)] \emph{Existence of multiplicative inverses}:
    For all $a \ne 0 \in R$, there exists
    $a^{-1} \in R$ such that $aa^{-1} = 1$.
\end{enumerate}
\end{definition}

\begin{example}
  Some examples of rings are $\Z / 2\Z$, which
  also happens to be a field. The ring $\Z$ is a domain.
  The set $M_{2 \times 2}(\R)$ is a non-commutative ring
  with unity, and has zero divisors. The ring
  $\Q$ is a field.\footnote{In fact, $\Q$ is somehow the smallest field containing $\Z$.} The real polynomials in a single
  variable $\R[x]$ form a ring, which is a domain but
  not a field. The complex numbers $\C$ and the
  real numbers $\R$ both form a field. The
  even integers $2\Z$ form a commutative ring without
  unity. In general, $\Z / n\Z$ is a commutative ring
  with unity, and is a field if and only if $n$ is prime
  (and has zero divisors otherwise, if $n$ is composite).
\end{example}

\begin{remark}
  If $(R, +, \cdot)$ is a ring, then $(R, +)$ is an
  abelian group. If $(K, +, \cdot)$ is a field,
  then $(K^*, \cdot)$ is an abelian group, where
  $K^* = K \setminus \{0\}$.
\end{remark}

\begin{definition}[Group of units]
  Let $R$ be a commutative ring with unity. The
  \emph{group of units} of $R$ is
  \[U = \{u \in R \mid \text{there exists $v \in R$ such that $uv = 1$}\}.\]
\end{definition}

\begin{exercise}
  Show that $U$ is in fact a group under multiplication.
\end{exercise}

\begin{definition}[Associate]
  If $a, b \in R$ such that $a = ub$ for some
  $u \in U$, then $a$ and $b$ are called
  \emph{associates}, denoted by $a \sim b$.
\end{definition}

\begin{exercise}
  Show that $\sim$ is in fact an equivalence relation.
\end{exercise}

\begin{example}
  The group of units of $\Z$ is $\{1, -1\}$. The group
  of units of a field $K$ is $K^* = K \setminus \{0\}$.
\end{example}

\begin{exercise}
  Let $R = \{a + b \sqrt{2} \mid a, b \in \Z\}$. Check
  the following:
  \begin{enumerate}
    \item $R$ is a commutative ring with unity.
    \item The group of units of $R$ is
      $\{a + b \sqrt{2} \mid a, b \in \Z, |a^2 - 2b^2| = 1\}$.
  \end{enumerate}
\end{exercise}

\begin{definition}[Divisor]
  Let $D$ be an integral domain, $a \in D \setminus \{0\}$,
  $b \in D$. Then $a$ divides $b$, or $a$ is a
  \emph{divisor} or \emph{factor} of $b$, denoted by
  $a | b$, if there exists $z \in D$ such that $az = b$.
  We write $a {\nmid} b$ if $a$ does not divide $b$.
  We say that $a$ is a \emph{proper divisor} or
  that $a$ \emph{properly divides} $b$ if $z$ is not
  a unit.
\end{definition}

\begin{remark}
  Equivalent, $a$ is a proper divisor of $b$ if
  and only if $a | b$ and $b {\nmid} a$.
\end{remark}

\begin{definition}[Subring]
  A \emph{subring} $U$ of a ring $R$ is a non-empty
  subset of $R$ with the property that for all
  $a, b \in R$, $a, b \in U$ implies $a + b \in U$
  and $ab \in U$,
  and $a \in U$ implies $-a \in U$.
\end{definition}

\begin{remark}
  Equivalently, $U$ is a subring of $R$ if and only
  if $a, b \in U$ implies $a - b \in U$ and $ab \in U$.
\end{remark}

\begin{remark}
  We automatically have $0 \in U$ since we can pick
  any $a \in U$, and then $0 = a - a \in U$.
\end{remark}

\begin{definition}[Subfield]
  A \emph{subfield} of a field $K$ is a subset $E$
  containing at least two elements such that
  $a, b \in E$ implies $a - b \in E$ and $a \in E, b \in E \setminus \{0\}$ implies $ab^{-1} \in E$. If $E$ is
  a subfield and $E \ne K$, then we say $E$ is a
  \emph{proper} subfield.
\end{definition}

\begin{remark}
  As before, we can replace the last condition with
  the equivalent statement that
  $a, b \in E$ implies $ab \in E$ and
  $a \in E \setminus \{0\}$ implies $a^{-1} \in E$.
\end{remark}

\begin{definition}[Ideal]
  An \emph{ideal} of $R$ is a non-empty subset $I$ of
  $R$ with the properties that $a, b \in I$ implies
  $a - b \in I$ and $a \in I, r \in R$ implies $ra \in I$.
\end{definition}

\begin{remark}
  All ideals are subrings, but the converse is not
  true in general.
\end{remark}

\begin{example}
  The integers $\Z$ form a subring of $\R$ but not an
  ideal.
\end{example}

\begin{remark}
  We trivially have that $\{0\}$ and $R$ are both
  ideals of $R$. An ideal $I$ is called \emph{proper}
  if $\{0\} \subsetneq I \subsetneq R$.
\end{remark}

\begin{theorem}
  Let $A = \{a_1, \dots, a_n\}$ be a finite subset
  of a commutative ring $R$. Then the set
  \[Ra_1 + \dots + Ra_n = \{x_1 a_1 + \dots + x_n a_n \mid x_i \in R\}\]
  is the smallest ideal of $R$ containing $A$.
\end{theorem}

\begin{proof}
  See Howie. Check this is indeed an ideal
  and is contained in any other ideal containing $A$.
\end{proof}

\begin{definition}[Ideals generated by elements of a ring]
  The set $Ra_1 + \dots + Ra_n$ is the
  \emph{ideal generated by} $a_1, \dots, a_n$,
  denoted by $\langle a_1, \dots, a_n \rangle$.
  If the ideal is generated by a single element
  $a \in R$, then we say that $Ra = \langle a \rangle$
  is a \emph{principal ideal}.
\end{definition}

\begin{example}
  In $\Z$, the ideal $\langle 2 \rangle = 2\Z$ are
  the even numbers. We have
  $\langle 2, 3 \rangle = \Z$, but
  $\langle 6, 8 \rangle = \langle 2 \rangle$.
\end{example}

\begin{theorem}
  Let $D$ be an integral domain with group of units
  $U$ and let $a, b \in D \setminus \{0\}$. Then
  \begin{enumerate}
    \item $\langle a \rangle \subseteq \langle b \rangle$
      if and only if $b | a$,
    \item $\langle a \rangle = \langle b \rangle$
      if and only if $a \sim b$,
    \item $\langle a \rangle = D$ if and only if
      $a \in U$.
  \end{enumerate}
\end{theorem}

\begin{proof}
  See Howie.
\end{proof}

\begin{definition}[Homomorphism of rings]
  A \emph{homomorphism} from a ring $R$ to a ring $S$
  is a mapping $\varphi : R \to S$ such that
  $\varphi(a +_R b) = \varphi(a) +_S \varphi(b)$ and
  $\varphi(ab) = \varphi(a)\varphi(b)$ for all
  $a, b \in R$.
\end{definition}

\begin{example}
  The zero mapping $\varphi(a) = 0$ is always a
  homomorphism. The inclusion map
  $\iota : 2\Z \to \Z$ or $\iota : \Z \to \Q$ is
  a homomorphism.
\end{example}

\begin{theorem}
  Let $R, S$ be rings and $\varphi : R \to S$ a
  homomorphism. Then
  \begin{enumerate}
    \item $\varphi(0_R) = 0_S$,
    \item $\varphi(-r) = -\varphi(r)$ for all $r \in R$,
    \item the image $\varphi(R)$ is a subring of $S$.
  \end{enumerate}
\end{theorem}

\begin{proof}
  See Howie.
\end{proof}

\begin{definition}[Monomorphism]
  Let $\varphi : R \to S$ be a homomorphism. If
  $\varphi$ is injective, we say that $\varphi$ is
  a \emph{monomorphism} or an \emph{embedding}.
\end{definition}

\begin{example}
  The inclusion map $\varphi : \Z \to \R$ given by
  $\varphi(n) = n$ is an embedding.
\end{example}
