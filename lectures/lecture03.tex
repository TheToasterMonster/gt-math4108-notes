\chapter{Jan.~17 --- Irreducible Polynomials}

\section{Principal Ideal Domains and Irreducibile Polynomials}
\begin{definition}
  The field of fractions of $D[X]$ consists of
  \emph{rational forms}
  \[
    \frac{a_0 + a_1 X + \dots + a_m X^m}{b_0 + b_1 X + \dots + b_n X^n}
  \]
  where $b_0 + b_1 X + \dots + b_n X^n \ne 0$,
  denoted by $D(X)$.
\end{definition}

\begin{definition}
  A domain $D$ is a \emph{principal ideal domain} (PID)
  if all of its ideals are principal.\footnote{Recall that a principal ideal is one generated by a single element.}
\end{definition}

\begin{example}
  The integers $\Z$ is a PID, since every ideal is of
  the form $\langle n \rangle$.
\end{example}

\begin{definition}
  A non-zero, non-unit element $p$ in a domain $D$
  is \emph{irreducible} if it has no proper factors.
\end{definition}

\begin{definition}
  A domain $D$ is a \emph{unique factorization domain} (UFD)
  if every non-unit $a \ne 0$ in $D$ has an
  essentially unique\footnote{As in, unique up to use of associates or adding in units.} factorization into irreducible
  elements.
\end{definition}

\begin{example}
  Again $\Z$ is a UFD, e.g. $12 = 2 \cdot 2 \cdot 3 = (-2) \cdot 2 \cdot (-3)$.
\end{example}

\begin{theorem}
  Every PID is a UFD.
\end{theorem}

\begin{proof}
  See Howie.
\end{proof}

\begin{theorem}
  If $K$ is a field, then $K[X]$ is a PID.
\end{theorem}

\begin{proof}
  See Howie.
\end{proof}

\begin{theorem}
  Let $p$ be an element in a PID $D$. Then the
  following are equivalent:
  \begin{enumerate}
    \item $p$ is irreducible.
    \item $\langle p \rangle$ is maximal.
    \item $D / \langle p \rangle$ is a field.
  \end{enumerate}
  In particular if $f \in K[X]$, then
  $K[X] / \langle f \rangle$ is a field if and only if $f$ is irreducible.
\end{theorem}

\begin{proof}
  See Howie.
\end{proof}

\begin{definition}
  Let $D$ be a domain and $\alpha \in D$.
  Let $\sigma_{\alpha} : D[X] \to D$ defined by
  \[
    \sigma_{\alpha} (a_0 + a_1 X + \dots + a_n X^n) = a_0 + a_1 \alpha + \dots + a_n \alpha^n.
  \]
  Note that we often write $\sigma_{\alpha} (f)$ as
  $f(\alpha)$. If $f(\alpha) = 0$, we say
  $\alpha$ is a \emph{root} of $f$, or a \emph{zero}.
\end{definition}

\begin{exercise}
  Check that $\sigma_{\alpha}$ is a homomorphism.
\end{exercise}

\begin{theorem}
  Let $K$ be a field, $\beta \in K$ and $f$ a non-zero
  polynomial in $K[X]$. Then $\beta$ is a root of $f$
  if and only if $X - \beta | f$.
\end{theorem}

\begin{proof}
  See Howie.
\end{proof}

\begin{example}
  We have $X^2 + 1$ in $\R[X]$ is irreducible,
  so $\R[X] / \langle X^2 + 1 \rangle$ is a field.
  In fact this field is isomorphic to the complex
  numbers $\C$.
\end{example}

\begin{exercise}
  Do the following:
  \begin{enumerate}
    \item Show that $\varphi : \R[X] \to \C$ given by
      \[
        \varphi (a_0 + a_1 X + \dots + a_n X^n) = a_0 + a_1 i + \dots + a_n i^n
      \]
      is a surjective homomorphism.\footnote{Note that there's some technicality about this $\varphi$ not being a $\sigma_{\alpha}$ since we defined $\sigma_{\alpha}$ for $\alpha$ in the base domain, and $i$ is kind of somewhere else.}
    \item Show that $\ker \varphi = \langle X^2 + 1 \rangle$.
  \end{enumerate}
  So by the first isomorphism theorem we can conclude
  that $\R[X] / \langle X^2 + 1 \rangle = \R / {\ker \varphi} \cong \varphi(\R[X]) = \C$.
\end{exercise}

\begin{theorem}
  Let $K$ be a field and $g \in K[X]$ an irreducible
  polynomial. Then $K[X] / \langle g \rangle$ is a field
  containing $K$ up to isomorphism.
\end{theorem}

\begin{proof}
  Since $g$ is irreducible, $K[X] / \langle g \rangle$
  is a field. Now define $\varphi : K \to K[X] / \langle g \rangle$ by
  \[
    \varphi (a) = a + \langle g \rangle.
  \]
  (Left as an exercise to check that $\varphi$ is a
  homomorphism.) We need to show that $\varphi$ is
  injective. For this, take $a, b \in K$. If
  $a + \langle g \rangle = b + \langle g \rangle$,
  then $a - b \in \langle g \rangle$. But $K$ is a field,
  so this happens precisely when $a = b$. Thus
  $\varphi$ embeds $K$ into $K[X] / \langle g \rangle$,
  as desired.
\end{proof}

\section{\texorpdfstring{Irreducible Polynomials over $\C$, $\R$, $\Q$, and $\Z$}{Irreducible Polynomials over C, R, Q, and Z}}
Our goal now is to study irreducible polynomials. Note
that linear polynomials are irreducible, and recall that
every polynomial in $\C$ factorizes, essentially
uniquely, into linear factors. Furthermore, complex roots
of real polynomials come in conjugate pairs, hence
\[
  g = a_0 + a_1 X + \dots + a_n X^n \in \R[X]
\]
factors as
\[g = a_n (X - \beta_1) \dots (X - \beta_r) (X - \gamma_1)(X - \overline{\gamma}_1) \dots (X - \gamma_3) (X - \overline{\gamma}_s)\]
in $\C[X]$, where $\beta_1, \dots, \beta_r \in \R$ and $\gamma_1, \dots, \gamma_s \in \C \setminus \R$ and
$r + 2s = n$. Thus over $\R[X]$, $g$ factors as
\[
  g = a_n (X - \beta_1) \dots (X - \beta_r) (X^2 - (\gamma_1 + \overline{\gamma}_1) X + \gamma_1 \overline{\gamma}_1) \dots (X^2 - (\gamma_s + \overline{\gamma}_s) X + \gamma_s \overline{\gamma}_s)
\]
in $\R[X]$, where the quadratic factors are irreducible
in $\R[X]$.

\begin{exercise}
  A quadratic $a X^2 + bX + c \in \R[X]$ is irreducible
  if and only if its discriminant $b^2 - 4ac < 0$.
\end{exercise}

Now we have pretty much characterized irreducible
polynomials in $\R[X]$. But what about $\Q[X]$?

\begin{theorem}
  Let $g = a_0 + a_1 X + a_2 X^2 \in \Q[X]$. Then
  \begin{enumerate}
    \item If $g$ is irreducible over $\R$,
      then it is irreducible over $\Q$.
    \item If $g = a_2 (X - \beta_1) (X - \beta)$ with
      $\beta_1, \beta_2 \in \R$, then $g$ is irreducible
      in $\Q[X]$ if and only if $\beta_1$ and $\beta_2$
      are irrational.
  \end{enumerate}
\end{theorem}

\begin{proof}
  (1) We show the contrapositive.
  If $g$ factors as
  \[
    g = a_2 (X - q_1) (X - q_2) \in \Q[X],
  \]
  then $g$ also factors in $\R[X]$.

  (2) If $\beta_1$ and $\beta_2$ are rational, then
  $g$ factors in $\Q[X]$ and is thus not irreducible.
  For the other direction, if $\beta_1$ and $\beta_2$
  are irrational, then $g = a_2 (X - \beta_1) (X - \beta_2)$
  is the only factorization in $\R[X]$ since $\R[X]$
  is a UFD, so there is no factorization in $\Q[X]$
  into linear factors.
\end{proof}

\begin{example}
  Are the following polynomials irreducible in $\R[X]$?
  In $\Q[X]$?
  \begin{enumerate}
    \item $X^2 + X + 1$ is irreducible over $\R$ and $\Q$
      since $b^2 - 4ac = -3$.
    \item $X^2 - X - 1$ has roots
      $(-1 \pm \sqrt{5}) / 2$, so it factors over $\R$
      but is irreducible over $\Q$.
    \item $X^2 + X - 2$ factors as $(X + 2)(X - 1)$
      over $\R$ and $\Q$.
  \end{enumerate}
\end{example}

Now that we have studied irreducible polynomials in
$\R[X]$ and $\Q[X]$, can a
polynomial in $\Z[X]$ be irreducible over $\Z$ but not
$\Q$? The answer is no!

\begin{theorem}[Gauss's lemma]
  Let $f$ be a polynomial in $\Z[X]$, irreducible
  over $\Z$. Then $f$ is irreducible over $\Q$.
\end{theorem}

\begin{proof}
  For sake of contradiction, suppose $f = gh$ with
  $g, h \in \Q[X]$ and $\partial g, \partial h < \partial f$.
  Then there exists $n \in \Z_{> 0}$ such that
  $nf = g' h'$ where $g', h' \in \Z[X]$. Let $n$ be the
  smallest positive integer with this property. Let
  \begin{align*}
    g' &= a_0 + a_1 X + \dots + a_k X^k \\
    h' &= b_0 + b_1 X + \dots + b_l X^l.
  \end{align*}
  If $n = 1$, then $g' = g$ and $h' = h$, a contradiction.
  Now $n \ge 1$, so let $p$ be a prime factor of $n$.\footnote{Lemma: Either $p$ divides all the coefficients of $g'$ or $p$ divides all the coefficients of $h'$. Proof left as an exercise.}
  Without loss of generality, assume $p$ divides $g'$,
  i.e. $g' = p g''$ where $g'' \in \Z[X]$. Then
  \[\frac{n}{p} f = g'' h',\]
  contradicting the minimality of $n$. Hence $f$
  cannot be factored over $\Q$.
\end{proof}

\begin{example}
  Show that $g = X^3 + 2X^2 + 4X - 6$ is irreducible
  over $\Q$.
\end{example}

\begin{proof}
  If $g$ factors over $\Q$, it factors over $\Z$ and
  at least one factor must be linear, i.e.
  \[g = X^3 = 2X^2 + 4X - 6 = (X - a)(X^2 + bX + c)\]
  where $a, b, c \in \Z$. We must have $ac = 6$, so
  $a \in \{\pm 1, \pm 2, \pm 3, \pm 6\}$ and $g(a) = 0$.
  We can check this:
  \begin{center}
    \begin{tabular}{c|cccccccc}
      $a$ & $1$ & $-1$ & $2$ & $-2$ & $3$ & $-3$ & $-6$ & $6$ \\
      \hline
      $g(a)$ & $1$ & $-9$ & $1$ & $-10$ & $51$ & $-27$ & $306$ & $-174$
    \end{tabular}
  \end{center}
  Hence $g$ is irreducible over $\Z$ and thus also
  irreducible over $\Q$.
\end{proof}

We could do this trick since the degree was 3,
forcing a linear factor.
What about degrees higher than 3?

\begin{theorem}[Eisenstein's criterion]
  Let $f = a_0 + a_1 X + \dots + a_n X^n \in \Z[X]$.
  Suppose there exists a prime $p$ such that
  \begin{enumerate}
    \item $p {\nmid} a_n$,
    \item $p | a_i$ for $i = 0, \dots, n - 1$,
    \item $p^2 {\nmid} a_0$.
  \end{enumerate}
  Then $f$ is irreducible over $\Q$.
\end{theorem}

\begin{proof}
  By Gauss's lemma, it suffices to show that $f$ is
  irreducible over $\Z$. Suppose for sake of contradiction
  that $f = gh$ for
  \[
    g = b_0 + b_1 X + \dots + b_r X^r \quad \text{and} \quad
    h = c_0 + c_1 X + \dots + c_s X^s,
  \]
  $r, s < n$, and $r + s = n$. Note that
  $a_0 = b_0 c_0$, so $p | a_0$ from (2) implies that
  $p | b_0$ or $p | c_0$. Since $p^2 {\nmid} a_0$,
  it cannot be both. Without loss of generality, assume
  $p | b_0$ and $p {\nmid} c_0$. Now suppose inductively
  that $p$ divides $b_0, \dots, b_{k - 1}$ where
  $1 \le k \le r$. Then
  \[a_k = b_0 c_k + b_1 c_{k - 1} + \dots + b_{k - 1} c_1 + b_k c_0\]
  and since $p$ divides $a_k$, $b_0 c_k$,
  $b_1 c_{k - 1}$, \dots, $b_{k - 1} c_1$, it follows that
  $p | b_k c_0$. Since $p {\nmid} c_0$ by assumption, we
  must have $p | b_k$. Thus
  $p | b_r$ and since $a_n = b_r c_s$, we have
  $p | a_n$, contradicting (1). Hence is $f$ is
  irreducible.
\end{proof}

\begin{example}
  The polynomial
  \[
    X^5 + 2X^3 + \frac{8}{7} X^2 - \frac{4}{7} X + \frac{2}{7}
  \]
  is irreducible over $\Q$.
\end{example}

\begin{proof}
  Multiply by $7$ and take the integer polynomial
  $7X^5 + 14X^3 + 8X^2 - 4X + 2$. Taking
  $p = 2$ satisfies Eisenstein's criterion, so this
  polynomial is irreducible over $\Z$ and thus also
  irreducible over $\Q$.
\end{proof}

\begin{example}
  If $p > 2$ is prime, then show that
  \[f = 1 + X + X^2 + \dots + X^{p - 1}\]
  is irreducible over $\Q$.
\end{example}

\begin{proof}
  First observe that
  \[f = \frac{X^p - 1}{X - 1}.\]
  Let $g(X) = f(X + 1)$. Then
  \begin{align*}
    g(X)
    &= \frac{(X + 1)^p - 1}{(X + 1) - 1}
    = \frac{1}{X} ((X + 1)^p - 1)
    = \frac{1}{X} \sum_{i = 0}^p \binom{p}{i} X^{p - i} - 1 \\
    &= \frac{1}{X} \sum_{i = 0}^{p - 1} \binom{p}{i} X^{p - i}
    = \sum_{i = 0}^{p - 1} \binom{p}{i} X^{p - i - 1}.
  \end{align*}
  Note that $\binom{p}{1}, \binom{p}{2}, \dots \binom{p}{p - 1}$
  are all divisible by $p$, so $g$ is irreducible by
  Eisenstein's criterion. Now if $f$ factors as
  $f = uv$, then $g(X) = u(X + 1)v(X + 1)$, which
  is a contradiction since $g$ is irreducible.
\end{proof}
